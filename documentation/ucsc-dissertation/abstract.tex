
Statistical models of price volatility most commonly use low-frequency
(daily, weekly, or monthly) returns. However, despite their
availability, two types of financial data have not been extensively
studied: high-frequency data where sampling periods are on the order
of seconds; and open, close, high, and low (OCHL) data which
incorporate intraperiod extremes.

The first part of this dissertation focuses on the development of a
filtering-based method for the estimation of volatility in high-frequency
returns, which contrasts currently popular averaging-based
approaches. The second part of this dissertation develops a
foundational and novel method for likelihood-based estimation for
bivariate OCHL, an approach unfeasible until now.

In Chapter 2, we formulate a discrete-time Bayesian stochastic
volatility model for high-frequency stock-market data that directly
accounts for microstructure noise, and outline a Markov chain Monte
Carlo algorithm for parameter estimation. The methods described in
this paper are designed to be coherent across all sampling timescales,
with the goal of estimating the latent log-volatility signal from data
collected at arbitrarily short sampling periods. In keeping with this
goal, we carefully develop a method for eliciting priors. The
empirical results derived from both simulated and real data show that
directly accounting for microstructure noise in a state-space formulation
allows for well-calibrated estimates of the log-volatility process
driving prices.

In Chapter 3, we present and motivate the bivariate OCHL problem,
enumerate the fundamental limitations of some common out-of-the-box
approaches, and present a semidiscrete Galerkin numerical solver for
computing the likelihood of the observed data. In addition, we prove
the consistency of maximum likelihood estimates under the approximate
density given by the solver.

Chapter 4 develops a closed-form, analytic solution to the OCHL
likelihood problem in parameter ranges where the Galerkin solver
requires near-infinite compute time and memory to produce numerically
accurate results. A matching solution is also proposed to interpolate
between parameter regions where neither the Galerkin nor analytic
solutions are applicable. Thus, we present a method for producing
likelihoods based on OCHL data over all model parameter ranges, which
is a key requirement in statistical estimation algorithms. We use
numerical experiments in both Chapters 3 and 4 to show the validity of
our methods and demonstrate the increase in statistical power in
estimating price volatility and correlation when using bivariate OCHL
data.