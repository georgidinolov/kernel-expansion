\documentclass[12pt]{article}
%
\usepackage{amssymb}
\usepackage{graphics}
\usepackage{epsfig}
\usepackage{epstopdf}
\usepackage{color}
%--- double-spaced -----
\def\dsp{\def\baselinestretch{1.5}\large\normalsize}
\dsp
%
\oddsidemargin   0.0 in
\evensidemargin  0.0 in
\topmargin          -0.6 in
\textwidth       6.5 in
\textheight      9.0 in
%\parindent       0.0 in
\parskip         0.1 in
%
\begin{document}

\begin{itemize}

\item In equation (1), change the boundary conditions to 
$$ \begin{array}{cc}
a_x \le X(t) \le b_x \\
a_y \le Y(t) \le b_y
\end{array}$$
Or change the boundary conditions in equation (3) to make the two equations 
consistent with each other. 

\item Delete $0 < t' \le t$ below equation (2). 

\item Change equation (3) to 
$$\frac{1}{dx dy} \mbox{Pr}\left( \left. \begin{array}{ll}
X(t) \in [x, x+dx), Y(t) \in [y, y+dy),  \\
\displaystyle \min_{t'\in[0, t]} X(t') \ge a_x, \max_{t'\in[0, t]} X(t') \le b_x, \\
\displaystyle \min_{t'\in[0, t]} Y(t') \ge a_y, \max_{t'\in[0, t]} Y(t') \le b_y
\end{array} \right | X(0)=x_0, Y(0)=y_0, \theta
\right) $$

\item In equation (4), change to $\displaystyle \frac{\partial}{\partial t'} $ 

\item Change equation (6) to 
$$ \frac{\partial^4}{\partial a_x \partial b_x \partial a_y \partial b_y} q(x,y,t) = 
\mbox{density} \left( \left. \begin{array}{ll}
X(t) =x, Y(t) =y,  \\
\displaystyle \min_{t'\in[0, t]} X(t') = a_x, \max_{t'\in[0, t]} X(t') = b_x, \\
\displaystyle \min_{t'\in[0, t]} Y(t') = a_y, \max_{t'\in[0, t]} Y(t') = b_y
\end{array} \right | X(0)=x_0, Y(0)=y_0, \theta
\right) $$

\item At the end of paragraph after equation (6), change to \\
`` ... method necessary for carrying out inferential procedures with ... ''

\item In the last paragraph of page 1, change to \\
`` ... each eigenfunction of the differential operator is a product of two sine functions, one in each dimension ...'' \\

\item In the last paragraph of page 1 \\
Use $a_x, a_y$ instead of $a_1, a_2$ \\
Use $b_x, b_y$ instead of $b_1, b_2$ 

\item In the last paragraph of page 1, change to  \\
``This, however, requires one to solve Fokker-Planck equation (4)-(5) accurately for at least 16
slightly different sets of boundaries and combine the results with numerical differentiation 
to evaluate the density function for a just single observation ... ''

\item At the beginning of first paragraph on page 2, add \\
``for non-zero correlation, '' \\

Change to \\
``... where the eigenfunctions for the differential operator are approximated
by the eigenvectors of a linear system obtained using a truncated expansion based on 
a set of separable basis function, each of which is a product of two sine functions (one in each dimension) satisfying the boundary conditions ...'' \\

\item In the middle of first paragraph on page , change to \\
 ``... makes the expansion a slow, if not unfeasible, solution.'' \\
 Change to \\
 ``... from either using a separable representation for the differential operator that is intrinsically 
 correlated in the two dimensions (trigonometric series) or ... ''

\item At the beginning of second paragraph on page , change to \\
``In this paper, we propose a robust and efficient solution to the general problem (4)-(5). The solution is obtained by combining a small-time analytic solution with a Galerkin discretization
based on basis functions that are correlated. ''

\item Change equation (9) to 
$$p(x,y,t) = \sum_v h_v \phi_v (x, y) e^{-\lambda_v t} $$
where $h_v$ is the coefficient of $\phi_v (x, y) $ in the eigenfunction expansion of $p(x, y, 0)$. 

\item 4 lines below equation (9) on page 3, change to \\
``... we approximate the eigenfunction using a finite set of orthogonal basis functions satisfying boundary conditions, i.e., a finite sequence of sines, ... ''

\item The equation below equation (9) on page 3, make changes \\
1. Each index should start at 1, not 0. \\ 
2. Make the associated changes in the definition of $\Psi(x, y)$ and $c_v$. 

\item In the paragraph below equation (10) on page 3, change to \\ 
`` ... Applying $L$ to the basis function expansion of $\phi_v $ and again approximating 
the result using the finite set of basis functions yields ''

\item 4 lines below equation (10) on page 3,  {\bf what is $\tilde{\theta} $?} 
This is the first time this parameter/quantity is mentioned.  

\item After $\tilde{\theta} $ on page 3, add \\
``In the last part of the equation above, we have truncated the infinite sine series 
expansion of $L \psi_{l, m}(x, y)$ '' 

\item 4 lines below equation (10) on page 3, delete the subscript outside the curly bracket \\
it should be $ \left\{ \psi_{l, m}(x, y) \right\} $. 

\item 5 lines below equation (10) on page 3, change to \\
`` The dense structure of matrix $A$ is caused by the mixing terms ... ''

\item 9 lines below equation (10) on page 3, change to \\
`` ... , we arrive at the matrix eigenvalue problem ''

\item Equation at the bottom of page 3, move $\Psi(x,y)^T $ outside the summation
and add the coefficient $h_v$ \\
It should be
$$ p(x,y,t) \approx \Psi(x,y)^T \sum_v h_v c_v e^{-\lambda_v t}  $$

\item Page 4, line 7, change``such that ...'' to \\
``The 4-th derivative of $p$ with respect to the 4 boundaries is approximated as '' \\
\begin{eqnarray} 
& & \frac{\partial^4}{\partial a_x \partial b_x \partial a_y \partial b_y} p(x, y, t) 
\nonumber \\
& & \hspace*{0.5cm} \approx \frac{\displaystyle \sum_{k_1, k_2, k_3, k_4 = \pm 1}
c_{\{ k_1, k_2, k_3, k_4 \} } p(\left. x, y, t \right| a_x+k_1 \varepsilon, b_x+k_2 \varepsilon, 
a_y+k_3 \varepsilon, b_y+k_4 \varepsilon) }{(2 \varepsilon )^4} 
\nonumber
\end{eqnarray}

\item Page 4, line 9, change to \\
`` ... requires many terms in the basis function expansions of the eigenfunctions, ...''

\item Paragraph after equation (11) on page 4, change to \\
`` ... Here, c(t) is a vector consisting of values of the solution in (8) on a set of grid points
 over $\Omega$ at time $t$, ... ''

\item Line 6 below equation (11) on page 4, change to \\
`` For example, using $c_{l, m}(t) $ 
to denote the approximation of the solution at grid point $(x_l, y_m)$, and assembling 
vector $c(t)$ using the index scheme $c_{l, m}(t) \rightarrow c_k(t)$ with  
$k=  (l-1) M + m$, we can approximate the operator $\frac{\partial^2 }{\partial x^2} $ as 

\item Line 12 below equation (11) on page 4, change to \\
`` ... with a constant h independent of parameters is appealing ...''

\item last sentence of page 4, delete the sentence since we are not discussing 
the approach in details. If we want to keep the sentence, we need to describe the approach
in some details, including i) keeping $\tau_x =1$ and 
$\tau_y = 1$; ii) keeping the 3 boundaries on grid points; iii) having to work with 
$\Omega \ne \mbox{a square}$; and iv) one side of boundaries not falling on grid points. 

\item Page 5, line 3, change to \\
`` ... , $h$ cannot be made too small because round-off error and the irregular part of 
the truncation error become an issue relatively quickly. 

\item Equations and text from line 6 of page 5 to the end of section 2, change to \\
$$p_{FD}(\left. x, y, t \right| b) = p(\left. x, y, t \right| b) 
+ h^2 \mbox{F}_{reg}(b) + h^2 \mbox{F}_{irreg}(b) + 
\varepsilon_{mach} \mbox{F}_{round}(b) $$
where we have included parameter $b $ explicitly as a simplified notation for 
$[a_x, b_x]$ and $[a_y, b_y]$ after shifting $a_x$ and$a_y$ to 0. 

Note that when expressed using the chain rule, both 
$\displaystyle \frac{\partial}{\partial a_x}$ and  $\displaystyle \frac{\partial}{\partial b_x}$
contain $\displaystyle \frac{\partial}{\partial b}$. As a result, 
$\displaystyle \frac{\partial^2 }{\partial a_x \partial b_x} $ leads to 
$\displaystyle \frac{\partial^2 }{\partial b^2} $. Although in the discussion below, 
for simplicity, we only illustrate the numerical differentiation on the first derivative, 
keep in mind that it is the second derivative that is more relevant in the calculation of 
 $\displaystyle \frac{\partial^4}{\partial a_x \partial b_x \partial a_y \partial b_y}
p(x, y, t) $. 

In the expression of finite difference solution above, 
$h^2 \mbox{F}_{reg}(b) $ is the regular part of 
the truncation error from discretizing the differential operator on the grid; 
$h^2 \mbox{F}_{irreg}(b) $ is the irregular part of the truncation error 
from the interpolations invoked by $\{x_0, y_0, x, y \}$ not being on grid points; 
and $\varepsilon_{mach} \mbox{F}_{round}(b) $ is the effect of round-off errors
with $\varepsilon_{mach} \sim 10^{-16}$ denoting the machine epsilon for 
IEEE double precision system. 
The coefficient, $\mbox{F}_{reg}(b) $, of the regular part of truncation error 
is a smooth function of $b$ with derivative $= O(1)$. 
The coefficient, $\mbox{F}_{round}(b) $, in the effect of round-off errors, 
behaves virtually like a random variable, discontinuous in $b$. 
For the irregular part of truncation error, the coefficient $\mbox{F}_{irreg}(b) $ 
in general is continuous in $b$ but not smooth in $b$ where the derivative has 
discontinuities of magnitude $O\left( \frac{1}{h} \right)$. 
For example, the error in 
a piece-wise linear interpolation of a smooth function at position $b$ using step $h$ 
has the general form of 
$$\mbox{Interpolation error } = O(h^2) (1-\mbox{rem}(b/h,1)) \mbox{rem}(b/h,1)$$ 
The coefficient part $\mbox{F}_{irreg}(b) = (1-\mbox{rem}(b/h,1)) \mbox{rem}(b/h,1) $ 
is continuous in $b$ but not differentiable. Its first derivative 
has the behavior of 
$$ \frac{\partial }{\partial b} \mbox{F}_{irreg}(b) =  \frac{1}{h} \left(1-2\mbox{rem}(b/h,1) \right)$$

Based on the expression we wrote out above for the finite difference solution, 
applying the numerical differentiation on t with step $\varepsilon$ yields: 
\begin{eqnarray}
& & \frac{p_{FD}(\left. x, y, t \right| b+\varepsilon)-p_{FD}(\left. x, y, t \right| b -\varepsilon)}{2 \varepsilon} \nonumber \\
& & \hspace*{1cm} =  \frac{\partial }{\partial b} p(\left. x, y, t \right| b) + O(\varepsilon^2)
 + h^2 \frac{\mbox{F}_{reg}(b+\varepsilon)-\mbox{F}_{reg}(b-\varepsilon)}{2 \varepsilon}  \nonumber \\
& & \hspace*{1cm} + h^2 \frac{\mbox{F}_{irreg}(b+\varepsilon)-\mbox{F}_{irreg}(b-\varepsilon)}{2 \varepsilon}
 + \varepsilon_{mach} \frac{\mbox{F}_{round}(b+\varepsilon)-\mbox{F}_{round}(b-\varepsilon)}{2 \varepsilon}  \nonumber
\end{eqnarray}
In the equation above, as the step in the numerical differentiation is refined, the first line 
of the RHS is well behaved, converging to the true value 
$\frac{\partial }{\partial b} p(\left. x, y, t \right| b)$ as $\varepsilon \rightarrow 0$. 
The second line of RHS, however, is problematic. As $\varepsilon \rightarrow 0$, 
the contribution from round-off error blows up to infinity
$$ \varepsilon_{mach} \frac{\mbox{F}_{round}(b+\varepsilon)-\mbox{F}_{round}(b-\varepsilon)}{2 \varepsilon}  = O\left(\frac{\varepsilon_{mach}}{\varepsilon} \right) 
\longrightarrow \infty \;\;\;\; \mbox{as } \varepsilon \rightarrow 0 $$
The contribution from the irregular part of truncation error is 
$$ h^2 \frac{\mbox{F}_{irreg}(b+\varepsilon)-\mbox{F}_{irreg}(b-\varepsilon)}{2 \varepsilon}  = 
O\left(\frac{h^2}{\max(\varepsilon, h)} \right) $$ 
In the second order numerical differentiation, however, the contribution from 
the irregular part of truncation error behaves like 
$$ h^2 \frac{\mbox{F}_{irreg}(b+\varepsilon)-2\mbox{F}_{irreg}(b)
+\mbox{F}_{irreg}(b-\varepsilon)}{\varepsilon^2 }  = 
O\left(\frac{h^2}{\max(\varepsilon^2, h^2)} \right) $$

\item {\Large \bf Below are new in version 2}

\item The first equation in section 3, change to 
$$p(x,y,t) = \sum_v h_v \phi_v (x, y) e^{-\lambda_v t} $$
where $h_v$ is the coefficient of $\phi_v (x, y) $ in the eigenfunction expansion of $p(x, y, 0)$. \textcolor{red}{DONE}

\item Line 2 on page 6, change to \\
`` ... i.e., $\psi_i(x, y) \in W_2^2(\Omega) $ where $\Omega = (0, 1) \times (0, 1)$ is 
the domain of the normalized problem. '' \\
{\bf Question:} How about the derivatives with respect to parameters 
$(\tau_x, \tau_y)$? These derivatives along with the derivatives with respect to $(x, y)$
will affect the derivatives with respect to $(a_x, b_x, a_y, b_y)$ in the original 
(unnormalized) problem. \textcolor{red}{DONE}

\item Two lines below equation (14) on page 6, \\
`` ... span of $S_k$ ... '' \\
Question: what is $S_k$? You need to define it. 
Is it the set of basis functions $\{\psi_i(x, y), 1 \le i \le k \}$? 
If it is already the space spanned by the set of basis functions 
$\{\psi_i(x, y), 1 \le i \le k \}$, then we just say \\
`` ... projection of $p(x, y, 0)$ onto subspace $S_k$. '' \textcolor{red}{DONE}

\item Two lines below equation (14) on page 6, replace `` ... with elements'' by \\
  ``. Matrix $M$, matrix $S$ and vector $\bf{p}(0)$ are expressed in elements as: ''
  \textcolor{red}{DONE}

\item Line 2 from bottom of page 6, \\
  `` We can then find a small enough $t_{\varepsilon}$ such that ... '' \\
  We also need to mention that we select the largest among all the
  time instances satisfying this condition.  \textcolor{red}{DONE}

\item Line 3 on page 7, \\
  `` so that the problem obeys the diffusion equation. '' \\
  Do you mean the diffusion equation with $\tau_x=1$, $\tau_y=1$ and
  $\rho=0$, obtained by a scaling and rotating transformation on
  equation (8)? \textcolor{red}{DONE}

\item Line 8 on page 7, \\
  `` We define distance between ... '' \\
  There are some minor problems with this definition when we look at
  the distances between the image of $(\xi_0, \eta_0)$ and the 4
  boundaries.  How about defining the distance to each boundary as the
  shortest distance between $(\xi_0, \eta_0)$ and the boundary
  segment? Note that the perpendicular intersection may not fall
  within the line segment when the 4 line segments form a very skewed
  diamond.  When the 4 line segments do form a very skewed diamond and
  when $(\xi_0, \eta_0)$ is close to a corner with angle larger than
  $90^o$, the image with respect to one side may well be very close in
  terms of perpendicular distance to the adjacent side. But the
  shortest distance to the line segment will be well behaved and the
  shorted distance is what matters since we only need to enforce the
  boundary condition over the line segment, not over the whole
  infinite line.  \textcolor{red}{DONE}


\item Line 4 in section 3.2 on page 7, change to \\
  $S_k = \{\psi_i(x, y), 0 \le i \le k\}$. \\
  This is the first time $S_k$ is defined. It needs to be defined
  earlier.  \textcolor{red}{DONE}

\item Line 4 of page 8, change to \\
  `` ... $\{\psi_i(x, y | \tilde{\rho}, \sigma) \}_{i=0}^k $ ... '' \\
  {\bf Question:} What about the derivatives of $c_i(t)$ with respect
  to parameters $(\tau_x, \tau_y)$?  Since the basis functions
  $\{\psi_i(x, y | \tilde{\rho}, \sigma) \}_{i=0}^k $ are not
  explicitly dependent on $(\tau_x, \tau_y)$, derivatives of
  $p^{(k)}(x,y,t) $ with respect to
  parameters $(\tau_x, \tau_y)$ will mainly affected by those of $c_i(t)$. Correct? \\
  \textcolor{red}{DONE} {\bf Question:} How do you select the value of
  $\sigma $?

\item Second equation in section 3.3, page 8 \\
  {\bf Question:} Should it be
$$ ... \lambda_j^2 \langle w, \phi_j \rangle^2 $$
\textcolor{red}{DONE}

\item The line below Equation (17) on page 8 \\
  {\bf Question/Comments:} Do you mean `` ... $h(k)$ is a decreasing
  function of $h$ ... ''? What is $h(k)$?  \textcolor{red}{DONE}

\item Equation (18) on page 8 \\
  {\bf Question/Comments:} Do you mean
  `` ... for $t > t_{\varepsilon}$ ...''? \\
  In equation (18), is coefficient $C$ the same as that in equation (17)? \\
  In equation (18), is $h(k)$ the same as that in equation (17)? \\
  If so, what/where is the contribution to global error from the
  discretization when the initial value at $t=t_{\varepsilon} $ is
  represented exactly with no error?  \textcolor{red}{DONE}

\item First equation in section 4 and the line above, change to  \\
`` ... an i.i.d set of samples $(Z_1, \ldots, Z_n)$ from random variable $Z(t)$ at a given 
value of $t$ in the process in (1)-(2). Specifically, $Z(t)$ is formed as 
$$ Z(t) =( X(t), Y(t), m_x(t), M_x(t), m_y(t), M_y(t)) $$
where 
$$m_x(t)= \min_{0 \le t' \le t} X(t), \;\; 
M_x(t)= \max_{0 \le t' \le t} X(t), \;\; 
m_y(t)= \min_{0 \le t' \le t} Y(t), \;\; 
M_y(t)= \max_{0 \le t' \le t} Y(t) $$
\textcolor{red}{DONE}

\item First equation in section 4 \\
  {\bf Question:} Do we need to specify $x0_0=0, y_0=0$?
  \textcolor{red}{DONE}

\item Second and third equations in section 4 \\
{\bf Question:} What is $F$? density function? cumulative distribution function? \\
The RHS of third equation looks like CDF but the LHS looks like PDF.
  \textcolor{red}{DONE; I think this is standard notation} 

\item Line 4 of page 9, change to \\
  `` ... or use only some of it. ''
    \textcolor{red}{DONE}

\item The equation above section 4.1 \\
{\bf Question:} What is the exact meaning of this equation? 
Is $Pr()$ density function? CDF? 
Or do you mean the convergence is the sense of probability?
    \textcolor{red}{DONE}

\item Equation (19) on page 9, \\
Revise it according to the revision suggested for equation (6). 

\item Line 2 below equation (20) on page 9, change it to \\
  `` ... $f^{(k)}(z)$ corresponds to the approximate solution
  $q^{(k)}$ of the unnormalized equation obtained from the Galerkin
  approximation $p^{(k)}$ of the normalized equation.
  \textcolor{red}{DONE}

\item Line 1 of page 10,  \\
{\bf Question:} What about the dependence of coefficients $c_i(t)$ on 
$(\tau_x, \tau_y)$, and in turn, their dependence on $(a_x, b_x, a_y, b_y)$) ?

\item Line 3 from the bottom of page 10, change to \\
  ``The second-order finite difference ... '' \\
  {\bf Note:} This is the second order numerical differentiation. The
  first order one has 9 terms instead of 16.
    \textcolor{red}{DONE}

\item Line 1 of page 11, change to \\
  `` ... in the second-order finite difference ...''
    \textcolor{red}{DONE}

  \item Page 11, equation (24),  change the last term to \\
    $O(\varepsilon^6; a_x, b_x, a_y, b_y) $ \textcolor{red}{DONE}

  \item In the two equations after equation (24),  \\
    All integration limits should be multiplied by $\varepsilon$.
    \textcolor{red}{DONE}

\item In the third equation after equation (24),  line 3 of RHS, change to  \\
  $ O(\varepsilon^6; a_x, b_x, a_y, b_y) $ \textcolor{red}{DONE}

  \item In the fourth equation after equation (24),  line 1 of RHS, change to  \\
    $ O(\varepsilon^2; a_x, b_x, a_y, b_y) $ \textcolor{red}{DONE}

\end{itemize}

\end{document}

