\documentclass[10pt]{article}

\usepackage{amsthm}
\usepackage[toc,page]{appendix}
\usepackage{amssymb}
\usepackage{bm}
\usepackage{pslatex,palatino,avant,graphicx,color}
\usepackage{colortbl}
\usepackage{fullpage}
\usepackage{mathtools}
\usepackage{natbib}
\usepackage{amsmath}
\usepackage{caption}
%\usepackage{subfigure}
\usepackage{subcaption}
\usepackage{bm}
\usepackage{wrapfig}
\usepackage{enumerate}
\usepackage{rotating}
\usepackage{multirow}
\usepackage{tabularx}
\usepackage{tikz}
\usepackage{pgfplots}
\usepackage{bbm}
\usepackage[titles,subfigure]{tocloft} % Alter the style of the Table of Contents

\newtheorem{lemma}{Lemma}
%\pdfminorversion=4
% NOTE: To produce blinded version, replace "0" with "1" below.
\newcommand{\blind}{0}

% DON'T change margins - should be 1 inch all around.
\addtolength{\oddsidemargin}{0in}%
\addtolength{\evensidemargin}{0in}%
\addtolength{\textwidth}{0in}%
\addtolength{\textheight}{0in}%
\addtolength{\topmargin}{0in}%


\renewcommand{\cftsecfont}{\rmfamily\mdseries\upshape}
\renewcommand{\cftsecpagefont}{\rmfamily\mdseries\upshape} % No bold!

\newcommand{\indicator}[1]{\mathbbm{1}\left( #1 \right) }
\newcommand{\hb}{\hat{b}}
\newcommand{\ha}{\hat{a}}
\newcommand{\htheta}{\hat{\theta}}
\newcommand{\halpha}{\hat{\alpha}}
\newcommand{\hmu}{\hat{\mu}}
\newcommand{\hsigma}{\hat{\sigma}}
\newcommand{\hphi}{\hat{\phi}}
\newcommand{\htau}{\hat{\tau}}
\newcommand{\heta}{\hat{\eta}}
\newcommand{\E}[1]{\mbox{E}\left[#1\right]}
\newcommand{\Var}[1]{\mbox{Var}\left[#1\right]}
\newcommand{\Indicator}[1]{\mathbbm{1}_{ \left( #1 \right) } }
\newcommand{\dNormal}[3]{ N\left( #1 \left| #2, #3 \right. \right) }
\newcommand{\Beta}[2]{\mbox{Beta}\left( #1, #2 \right)}
\newcommand{\alphaphi}{\alpha_{\hphi}}
\newcommand{\betaphi}{\beta_{\hphi}}
\newcommand{\expo}[1]{ \exp\left\{ #1 \right\}}
\newcommand{\tauSquareDelta}{\htau^2
  \left(\frac{1-\expo{-2\htheta\Delta}}{2\htheta} \right)}
\newcommand{\mumu}{\mu_{\hmu}}
\newcommand{\sigmamu}{\sigma^2_{\hmu}}
\newcommand{\sigmamuexpr}{\log\left( \frac{\VarX}{\EX^2} + 1 \right)}
\newcommand{\mumuexpr}{\log(\EX) -  \log\left( \frac{\VarX}{\EX^2} + 1 \right) /2 }

\newcommand{\EX}{\mbox{E}\left[ X \right] }
\newcommand{\VarX}{\mbox{Var}\left[ X \right] }
\newcommand{\mueta}{\mu_{\heta} }
\newcommand{\sigmaeta}{\sigma^2_{\heta}}
\newcommand{\sigmaetaexpr}{ \log\left( \frac{\VarX}{\EX^2} + 1 \right) }
\newcommand{\muetaexpr}{ \log(\EX) -  \sigmaetaexpr /2 }

\newcommand{\mualpha}{\mu_{\halpha} }
\newcommand{\sigmaalpha}{\sigma^2_{\halpha}}
\newcommand{\sigmaalphaexpr}{ \log\left( \frac{\VarX}{\EX^2} + 1 \right) }
\newcommand{\mualphaexpr}{ \log(\EX) -  \sigmaalphaexpr /2 }

\newcommand{\mutauexpr}{ \frac{2}{T} \EX }
\newcommand{\sigmatauexpr}{ \frac{4}{T^2} \Var{X}}

\newcommand{\alphatau}{\alpha_{\htau^2}}
\newcommand{\betatau}{\beta_{\htau^2}}

\newcommand{\Gam}[2]{\mbox{Gamma}\left( #1, #2 \right) }
\newcommand{\InvGam}[2]{\mbox{Inv-Gamma}\left( #1, #2 \right) }

%%% END Article customizations

%%% The "real" document content comes below...

% \newbox{\LegendeA}
% \savebox{\LegendeA}{
%    (\begin{pspicture}(0,0)(0.6,0)
%    \psline[linewidth=0.04,linecolor=red](0,0.1)(0.6,0.1)
%    \end{pspicture})}
% \newbox{\LegendeB}
%    \savebox{\LegendeB}{
%    (\begin{pspicture}(0,0)(0.6,0)
%    \psline[linestyle=dashed,dash=1pt 2pt,linewidth=0.04,linecolor=blue](0,0.1)(0.6,0.1)
%    \end{pspicture})}

\title{Solution to a Non-Seperable Diffusion Equation on a Regular Domain}
\author{Georgi Dinolov, Abel Rodriguez, Hongyun Wang}
\date{} % Activate to display a given date or no date (if empty),
         % otherwise the current date is printed 

\begin{document}
\def\spacingset#1{\renewcommand{\baselinestretch}%
{#1}\small\normalsize} \spacingset{1}

\bigskip

\vspace{1cm}
\noindent

\spacingset{1.00} % 
\section{Introduction}

We consider two-dimensional correlated Brownian motion with absorbing boundaries:
\begin{align}
  X(t) &= x_0 + \mu_x t + \sigma_x W_x(t) &a_x &< X(t) < b_x   \label{eq:X} \\
  Y(t) &= y_0 + \mu_y t + \sigma_y W_y(t) &a_y &< Y(t) < b_y   \label{eq:Y} 
\end{align}
where $W_i$ are standard Brownian motions with
$\mbox{Cov}(W_1(t), W_2(t)) = \rho t$ for $0 < t' \leq t$. In
particular, we find the joint transition density function for
$(X(t), Y(t))$ under the boundary conditions:
\begin{align}
  \Pr\left(X(t) \in dx, Y(t) \in dy,  \min_{t'}X(t') \geq a_x,
  \max_{t'}X(t')\leq b_x, \min_{t'} Y(t')\geq a_y, \max_{t'} Y(t')\leq b_y|  X(0)=x_0, Y(0)=y_0, \theta \right), \label{eq:CDF}
\end{align}
with $\theta := (\mu_x, \mu_y, \sigma_x, \sigma_y, \rho).$ This
function, which we shorten to $q(x,y,t)$ from now on, is the solution
to the Fokker-Planck equation \citep{oksendal2013stochastic}:
\begin{align}
  \frac{\partial}{\partial t} q(x,y,t') &= -\mu_x \frac{\partial}{\partial x}q(x,y,t')
                                         - \mu_y \frac{\partial}{\partial y}q(x,y,t')
                                         + \frac{1}{2}\sigma_x^2 \frac{\partial^2}{\partial x^2}q(x,y,t')
                                         + \rho\sigma_x\sigma_y \frac{\partial^2}{\partial x \partial y}q(x,y,t')
                                         + \frac{1}{2}\sigma_y^2 \frac{\partial^2}{\partial y^2}q(x,y,t'),
  \label{eq:1} \\
  q(a_x, y,t') &= q(b_x,y,t') = q(x,a_y,t') = q(x,b_y,t') = 0, \label{eq:2} \\
   0 &< t' \leq t. \nonumber
\end{align}
Differentiating $q(x,y,t)$ with respect to the boundaries produces the
transition density of a particle beginning and ending at the points
$(X_1(0), X_2(0))$ and $(X_1(t), X_2(t))$, respectively, while
attaining the minima $a_x/a_y$ and maxima $b_x/b_y$ in each coordinate
direction:
\begin{align*}
  \frac{\partial^4}{\partial a_x \partial b_x \partial a_y \partial
  b_y} q(x,y,t) = 
\end{align*}
\begin{align}
  \Pr\left(X(t) \in dx, Y(t) \in dy, \min_{t'}X(t') = a_x,
  \max_{t'}X(t')=b_x, \min_{t'} Y(t')=a_y, \max_{t'} Y(t')=b_y \left| 0 <
  t' \leq t, X(0)=x_0, Y(0)=y_0, \theta \right.\right). \label{eq:pdf}
\end{align}
The transition density (\ref{eq:CDF}) with less than all four
boundaries has been used in computing first passage times
\citep{kou2016first, sacerdote2016first}, with application to
structural models in credit risk and default correlations
\citep{haworth2008modelling, ching2014correlated}. \cite{he1998double}
use variants of (\ref{eq:pdf}) with respect to some of the boundaries
to price financial derivative instruments whose payoff depends on
\textbf{some} of the observed maxima/minima. \cite{rodriguez2012} use
a full likelihood-based (Bayesian) approach to estimate volatility in
\textit{univariate} financial timeseries where open, closing, highest,
and lowest prices are included. Their work
fits into a body of literature and collection of techniques by
practitioners where the observed range of prices is used to make
similar estimates. In this paper we will provide the efficient
numerical methods necessary to carry out similar inferential
procedures with correlated financial timeseries.

Closed-form solutions to (\ref{eq:1}) - (\ref{eq:2}) are available for
some parameter regimes. When $\rho = 0$, the transition density of the
process is the solution to a well-understood Sturm-Liouville problem
where the eigenfunctions of the differential operator are sine
functions. When $a_1 = -\infty$ and $b_1 = \infty$, the method of
images can be used to enforce the remaining boundaries. For either
$a_1, a_2 = -\infty$ or $b_1, b_2 = \infty$, eigenfunction of the
Fokker-Plank equation can be found in radial coordinates. Both of
these techniques are used and detailed by
\cite{he1998double}. However, to the best of our knowledge, there is
no closed-form solution to the general problem in (\ref{eq:1}) -
(\ref{eq:2}). This also limits the available ways to compute
(\ref{eq:pdf}), with the most straightforward approach being finite
difference with respect to the boundary conditions. This, however,
requires one to solve at least 16 eigenvalue problems to evaluate the
density function for a single observation (see Section
\ref{sec:likelihood-calc}), motivating the need for an efficient
numerical method to solve (\ref{eq:1}) - (\ref{eq:2}).

It is still possible to approach the general problem by proposing a
biorthogonal expansion in time and space
(\cite{risken1989fokker-planck}, sections 6.2), where the
eigenfunctions for the differential operator are approximated as a
sine series satisfying the boundary conditions. However, a drawback of
this out-of-the-box solution is that the system matrix for the
corresponding eigenvalue problem is dense. Additionally, for nonzero
$\rho$, it is necessary to have a large number of basis elements for
an accurate approximation. The denseness and size of the resultant
system matrices makes the eigenvalue expansion a slow, if not
unfeasible, solution. An alternative is to use a finite difference
scheme to directly solve the evolution problem after suitable
transformations. However, both of these methods need a high degree of
numerical resolution to produce practically useful approximations of
the transition density. We conjecture that these inefficiencies come
from either using a \textit{separable} representation for the
differential operator (trigonometric series) or introducing numerical
diffusion (finite difference).

In this paper, we propose a solution to the general problem
(\ref{eq:1}) - (\ref{eq:2}) which is obtained by combining a
small-time analytic solution with a finite-element method. Our method
directly takes into account the correlation parameter present in the
differential operator in order to efficiently represent the analytic
small-time solution and propagate it forward in time. We apply our
computational method to estimate equation parameters with a maximum
likelihood approach in settings where the model assumptions of
constant $(\mu_x, \mu_y, \sigma_x, \sigma_y, \rho)$ and Brownian
motion driving stochastic evolution are appropriate. Section
\ref{sec:approximate-sols} outlines some methods we considered,
including the one we used. Section [] includes our numerical
experiments.


\section{Approximate Numerical Solutions} \label{sec:approximate-sols}
Before considering solutions to the full Fokker-Planck equation
(\ref{eq:1}) - (\ref{eq:2}), we simplify the PDE by proposing a
scaling transformation and an exponential decomposition of the
solution, so that we can construct
\[
  p(x,y,t) = \exp(\alpha x + \beta y + \gamma t) q\left( x(b_x - a_x) + a_x, \quad y(b_y - a_y) + a_y, \quad t \right),
\]
where


\begin{align*}
  \alpha &= -\frac{\mu_x}{\sigma_x^2} - \frac{\rho}{\sigma_x\sigma_y(1-\rho^2)}\left( -\frac{\mu_y}{\sigma_y^2} + \frac{\mu_x \rho}{\sigma_x \sigma_y} \right), \\
  \beta &= \left( -\frac{\mu_y}{\sigma_y^2} + \frac{\mu_x \rho}{\sigma_x \sigma_y} \right), \\
  \gamma &= \frac{1}{2}\left( \frac{\sigma_x}{(b_x-a_x)} \right)^2 \alpha^2 + \frac{1}{2}\left(\frac{\sigma_y}{(b_y-a_y)}\right)^2 \beta^2 + \alpha\beta.
\end{align*}
This new formula satisfies the simpler diffusion equation:
\begin{equation}
  \frac{\partial}{\partial t} p(x,y,t) = \mathcal{L}p(x,y,t),\quad (x,y) \in (0,1) \times (0,1) := \Omega \label{eq:qq}
\end{equation}
subject to the constraints
\begin{align}
  p(x,y,t) &=0, & \mbox{ for } & (x,y) \in \{ x | x = 0\} \cup \{ y | y = 0\}, \nonumber \\
  p(x,y,0) &= \delta\left( x - \frac{x_0-a_x}{b_x - a_x} \right) \delta\left(y-\frac{y_0 - a_y}{b_y - a_y}\right), \nonumber
\end{align}
where the differential operator $\mathcal{L}$ takes the form
\[
  \mathcal{L} = \frac{1}{2} \tau_x^2 \frac{\partial^2}{\partial x^2}
  + \rho\tau_x\tau_y \frac{\partial^2}{\partial x \partial y} + \frac{1}{2}\tau_y^2 \frac{\partial^2}{\partial y^2},
  % \left(  \right)^2
  %                                                   \frac{\partial^2}{\partial x^2}p(x,y,t) + \rho\left( \frac{\sigma_x}{(b_x-a_x)} \right)\left( \frac{\sigma_y}{(b_y-a_y)} \right)
  %                                                   \frac{\partial^2}{\partial x \partial y}p(x,y,t) +
  %                                        \frac{1}{2}\left( \frac{\sigma_y}{(b_y-a_y)} \right)^2 \frac{\partial^2}{\partial y^2}p(x,y,t) , \label{eq:qq} \\
\]
where $\tau_x = \frac{\sigma_x}{(b_x-a_x)}$,
$\tau_y = \frac{\sigma_y}{(b_y-a_y)}$.  Note here that under this
transformation $\rho$ remains the same as in the original problem. We
will call equation (\ref{eq:qq}) the \textit{normalized} problem and
will consider its solution in terms of the diffusion parameters
$(\tau_x, \tau_y, \rho)$ without a loss of generality.

\subsection{Eigenfunction Expansion} \label{sec:eigenfunction}
Following Section 6.2 of \cite{risken1989fokker-planck}, we may use
the biorthogonal decomposition of the solution as a sum of
eigenfunctions and time-dependent coefficients determined by eigenvalues:
\begin{equation}
  p(x,y,t) = \phi_\nu(x,y) e^{-\lambda_\nu t}, \label{eq:biorthogonal}
\end{equation}
where the eigenfunctions $\phi_\nu(x, y)$ satisfy the boundary
conditions. Because the differential operator $\mathcal{L}$ is
self-adjoint, the family of eigenfunctions is complete in the Hilbert
space $L_2(\Omega)$. Moreover, the eigenvalues are bounded below by a
positive constant $c$, so that the solution behaves as expected (see
section 6.3 of \cite{risken1989fokker-planck}).

Since we require $\phi_\nu(x,y)$ to be zero on the boundaries, we
may represent the eigenfunction as a linear combination of sines
\[
  \phi_\nu(x,y) = \sum_{l=0}^L \sum_{m=0}^M c_{l,m, \nu}
  \sin\left(2\pi\, l\, x \right) \sin\left(2\pi\, m\, y \right) := \Psi(x,y)^T c_\nu,
\]
where we have truncated the infinite series for some suitably large
$L$ and $M$ and defined
\begin{align*}
  \psi_{l,m}(x,y) &= \sin\left(2\pi\, l\, x \right)
                         \sin\left(2\pi\, m\, y \right), \\
  \Psi(x,y) &= (\psi_{0,0}(x,y), \ldots, \psi_{L,M}(x,y))^T, \\
  c_\nu &= (c_{0,0,\nu}, \ldots, c_{L,M,\nu})^T.
\end{align*}

The biorthogonal representation (\ref{eq:biorthogonal}) leads to the
eigenvalue problem
\begin{equation}
  \mathcal{L} \phi_\nu = -\lambda_\nu \phi_\nu, \label{eq:eigenproblem}
\end{equation}
where $\mathcal{L}$ is the differential operator in the normalized
Fokker-Planck equation. Applying $\mathcal{L}$ to $\phi_\nu$ produces
the linear system
\[
  \mathcal{L}\phi_\nu = \mathcal{L}(\Psi(x,y)^T c_\nu) =
  \mathcal{L}(\Psi(x,y)^T) c_\nu = (A \Psi(x,y))^T c_\nu,
\] 
where $A$ is a constant matrix dependent on $\tilde{\theta}$. In the
case where $\rho = 0$, $A$ is diagonal because
$\left\{ \psi_{l,m}(x,y) \right\}_{l,m}$ are the eigenfunctions to
$\mathcal{L}$. When $\rho \neq 0$, $A$ is no longer diagonal and is in
fact dense. This caused by the mixing terms
\[
  \frac{\partial^2}{\partial x \partial y} \sin\left(2\pi\, l\,
    x\right) \sin\left(2\pi\, m\, y\right) = (2\pi\, l)(2\pi\, m)
  \cos\left(2\pi\, l\, x\right) \cos\left(2\pi\, m\, y\right)
\]
being the products of cosine functions, which have an inefficient sine
series representation [CITE]. Substituting the linear representation
of $\mathcal{L}\phi_\nu$ into the eigenvalue problem
(\ref{eq:eigenproblem}), we arrive to the system
\[
  \Psi(x,y)^T A^T c_\nu = -\lambda_\nu \Psi(x,y)^T c_\nu
  \quad \Leftrightarrow \quad A^T c_\nu = -\lambda_\nu c_\nu
\]
whose solution gives the family of orthonormal eigenfunctions. As
mentioned already, the efficiency of this approach is dependent on the
cost of solving the eigenvalue problem
$A^T c_\nu = -\lambda_\nu c_\nu$.

With all of the eigenpairs $(c_\nu, \lambda_\nu)$, the approximate solution is then
\[
  p(x,y,t) \approx \sum_{\nu}\Psi(x,y)^T c_\nu e^{-\lambda_\nu t}.
\]

\subsubsection{Likelihood Calculation} \label{sec:likelihood-calc}
Recalling equation (\ref{eq:pdf}), to compute the likelihood we are
interested in we must take derivatives of the solution $p(x,y,t)$ with
respect to the boundary parameters $(a_x, b_x, a_y, b_y)$. Because the
eigenvalues and eigenvectors for this problem are functions of the
parameters without explicit analytic form,
$\frac{\partial^4}{\partial a_x\partial b_x \partial a_y \partial
  b_y}p(x,y,t)$ must be computed numerically using a finite difference
approximation. We must therefore find the eigenvalues and
eigenvectors for each of the sixteen perturbed problems
\[
  p(x,y,t | a_x \pm \epsilon, b_x \pm \epsilon, a_y \pm \epsilon, b_y \pm \epsilon),
\]
such that
\[
  \frac{\partial^4}{\partial a_x\partial b_x \partial a_y \partial
    b_y}p(x,y,t) \approx \frac{\sum_{k} c_k p(x,y,t | a_x \pm \epsilon, b_x \pm \epsilon, a_y \pm \epsilon, b_y \pm \epsilon)}{(2\epsilon)^4}.
\]
Because a good, working approximation to $p(x,y,t)$ requires many
terms in the expansion of the eigenfunctions, and because each
resultant system matrix is dense, repeated computation of the
likelihood in this way is unfeasible.

\subsection{Finite Difference} \label{sec:finite-difference} A finite
difference method which approximates the spatial derivatives in
problem (\ref{eq:qq}) requires the solution a system of differential
equations
\begin{align}
  \dot{c}(t)= A c(t) &\Rightarrow& c(t) = \exp\left( At \right)c(0) \label{eq:eigenproblem-fd}, 
\end{align}
which reduces to the eigenvalue decomposition of a matrix $A$. Here,
$c(t)$ is a vector of coefficients representing the value of the
solution in (\ref{eq:qq}) on each point on a grid over $\Omega$ at
time $t$, and the product $Ac(t)$ approximates
$\mathcal{L}p(x,y,t)$. The system matrix $A$ is dependent on the
discretization scheme used to approximate $\mathcal{L}$. Using a
central in space scheme over a \textbf{regular} grid on $\Omega$ with
$\Delta x = \Delta y = h$,
\[ A = \frac{1}{2}\tau_x^2 \frac{1}{h^2}A_{x,x} +
  \rho\tau_x\tau_y \frac{1}{4h^2}A_{x,y} + \frac{1}{2}\tau_y^2
  \frac{1}{h^2}A_{y,y},
\]
where each of the matrices $A_{x,x}, A_{x,y}, A_{y,y}$ approximate
$\frac{\partial^2}{\partial x^2}, \frac{\partial^2}{\partial
  x \partial y}, \frac{\partial^2}{\partial y^2}$, respectively. For
example, using the labeling $c_{l,m}(t) \to c_k(t)$ on the vector
$c(t)$ to denote the approximation of the solution at point
$(x_l, y_m)$ on the grid,
\begin{align*}
  \frac{1}{2}\tau_x^2 \frac{\partial^2}{\partial x^2} p(x_l,y_m,t) \approx \frac{1}{2}\tau_x^2 \left( \frac{c_{l+1,m}(t) - 2c_{l,m}(t) + c_{l-1,m}(t)}{h^2} \right) = \frac{1}{2}\tau_x^2 \frac{1}{h^2}A_{x,x}[k,] c(t) 
\end{align*}
where $A_{x,x}[k,]$ is the $k^{th}$ row of $A_{x,x}$. Similarly,
\begin{align*}
  \rho\tau_x\tau_y \frac{\partial^2}{\partial x \partial y} p(x_l,y_m,t) \approx \rho\tau_x\tau_y \left( \frac{c_{l+1,m+1}(t) - c_{l+1,m-1}(t) - c_{l-1,m+1}(t) + c_{l-1,m-1}(t)}{4h^2} \right) = \rho\tau_x\tau_y \frac{1}{4h^2}A_{x,y}[k,] c(t) 
\end{align*}
It should be noted here that a regular grid approach with a constant
$h$ is appealing, because it allows us to construct once and store the
matrices $A_{x,x}, A_{x,y}, A_{y,y}$, which saves valuable
computational resources if we are to solve the finite difference
eigenproblem (\ref{eq:eigenproblem-fd}) repeatedly for different
parameter values $(\tau_x,\tau_y,\rho)$.

Unlike the system matrix for the trigonometric expansion, the system
matrix $A$ here is sparse: each row of $(A_{x,x}, A_{x,y}, A_{y,y})$
is composed of all zeros except for three or four entries. This
structure does not change as $h \to 0$. The eigenvalue problem is
therefore much cheaper to solve. The system matrix can be made even
sparser on a regular grid by performing a $45^{\circ}$ rotation which
removes the mixing term
$\rho\tau_x \tau_y\frac{\partial^2}{\partial x \partial y}$ from
the problem PDE and preserves the boundaries.

However, the fundamental limitation of using a finite difference
method is that differentiation with respect to boundaries is also done
using finite differences and it is of fourth order. Given this high
order differentiation, $h$ cannot be made too small because truncation
error becomes an issue relatively quickly. More explicitly, assuming
consistency and stability of the finite difference approximation to
$\mathcal{L}$, we can write down the finite difference solution as
\[
  p_{FD}(x_l,y_m,t | a_x) = p(x_l,y_m,t) + O(h^2; a_x).
\]
Note here the residual term $O(h^2; a_x)$ a function of the paramter
$a_x$. The finite difference approximation yields:
\begin{align*}
  \frac{p_{FD}(x_l,y_m,t | a_x) - p_{FD}(x_l,y_m,t | a_x - \epsilon)}{\epsilon} &= \frac{p(x_l,y_m,t | a_x) - p(x_l,y_m,t | a_x - \epsilon)}{\epsilon} + \frac{O(h^2 | a_x) - O(h^2 | a_x - \epsilon)}{\epsilon} \\
      &\approx \frac{\partial}{\partial a_x} p(x_l,y_m,t) + O(\epsilon) + \frac{O(h^2 | a_x) - O(h^2 | a_x - \epsilon)}{\epsilon}.
\end{align*}
If the residual term $O(h^2 | a_x)$ is differentiable with respect to
$a_x$,
\[
  \frac{O(h^2 | a_x) - O(h^2 | a_x - \epsilon)}{\epsilon} \to O(h^2) \mbox{ as } \epsilon \to 0.
\]
Otherwise
\[
  \frac{O(h^2 | a_x) - O(h^2 | a_x - \epsilon)}{\epsilon} \to O(h^2/\epsilon^r) \mbox{ as } \epsilon \to 0.
\]
for some $r > 0$. Under this condition, as $\epsilon \to 0$ with $h$
fixed, the term $O(h^2/\epsilon^r)$ begins to dominate the
approximation. Hence, for a fixed $h$, we have a bound on how
accurately we may represent the likelihood function provided the
finite difference residual term is not differentiable with respect to
the boundary parameters.

The finite value for $h$ introduces yet another practical concern. On
a regular grid, the delta function initial condition needs to be
either rounded to the nearest grid point or represented as a weighted
sum of delta functions on the four nearest grid points. Either
approach introduces a numerical diffusion into the problem which, for
a finite $h$, can bias the numerical solution. We will demonstrate
this phenomenon in Section [].

\subsection{Semidiscrete Galerkin Method}
We propose a semidiscrete Galerkin (continuous in time, discrete in
space) solution to the general diffusion problem (\ref{eq:qq}). The
success of this approach is based on an analytic approximation for the
solution $p(x,y,t)$ for small time $t$ and a basic convergence
estimate from approximation theory for semidiscrete Galerkin-type
solutions to parabolic problems. 

Our numerical method produces a functional representation for the
approximate solution which $1)$ imposes a computational burden
comparable to or better than that of the finite difference method and
$2)$ is infinitely differentiable with respect to the boundary
parameters, allowing us to perform the crucial numerical
differentiation with respect to these parameters.

As described in Section \ref{sec:eigenfunction}, the solution to the
model problem has the eigenfunction expansion 
\[
  p(x,y,t) = \sum_{\nu=0}^\infty\phi_\nu(x,y) e^{-\lambda_\nu t}.
\]
Proceeding with the standandard Galerkin approach, we propose a
solution $p_{k}(x,y,t)$ of similar form
\[
  p_{k}(x,y,t) = \sum_{i=0}^k c_i(t) \psi_i(x,y),
\]
where the basis functions $\psi_i(x,y)$ satisfy the boundary
conditions. We also require that all first- and second-order
derivatives of $\psi_i(x,y)$ are in $L_2(\Omega)$, i.e.
$\psi_k(x,y) \in W_{2}^{2}(\Omega)$. In other words, we look for a
solution in the span of $S_k = \{\psi_0, \ldots, \psi_k\}$. Since
$p_k$ is an approximation to the solution $p$, it does not follow the
differential equation exactly nor can it represent the initial
condition fully. We capture this by defining residuals
\begin{align*}
  \frac{\partial}{\partial t} p_k(x,y,t) - \mathcal{L}p_k(x,y,t) &:= R_e(k), \\
  p(x,y,0) - p_k(x,y,0) &:= R_0(k).
\end{align*}
There are various conditions that could be imposed on the residual
functions (see Section 2.10.3 of \cite{norrie1973finite} for a
summary). The \textit{orthogonality} condition coincides with the
Galerkin procedure:
\begin{align}
  \displaystyle \int_{\Omega} R_e(k) \psi_i(x,y) dx\,dy &= 0,& \displaystyle \int_{\Omega} R_0(k) \psi_i(x,y) dx\,dy &= 0,& i = 0,\ldots k, \label{eq:orthogonality-conditions}
\end{align}
which is equivalent to the weak formulation of the heat problem
\begin{align*}
  \left( \partial_t p_k(x,y,t), \psi_i \right) &= \left(\mathcal{L}p_k(x,y,t), \psi_i \right), \\
  \left( p_k(x,y,0), \psi_i \right) &= \left(p(x,y,0), \psi_i\right),
\end{align*}
where $(\cdot, \cdot)$ is the usual inner product in
$L_2(\Omega)$. The orthogonality conditions
(\ref{eq:orthogonality-conditions}) lead to the system of equations
\begin{align*}
  M \mathbf{\dot{c}}(t) &= S \mathbf{c}(t), \\
  M \mathbf{c}(0) &= \mathbf{p}(x,y),
\end{align*}
where $M$ is the mass matrix, $S$ is the stiffness matrix, and
$\mathbf{p}(x,y,0)$ is the vector projection of $p(x,y,0)$ onto the
span of $S_k$ with elements
\begin{align*}
  M_{ij} &= \displaystyle \int_\Omega \psi_i \psi_j dx\,dy, \\
  S_{ij} &= -\frac{1}{2}\tau_x^2 \displaystyle \int_\Omega \left( \frac{\partial}{\partial x} \psi_i(x,y) \right) \left( \frac{\partial}{\partial x} \psi_j(x,y) \right) dx\,dy -\rho\tau_x\tau_y \displaystyle \int_\Omega \left( \frac{\partial}{\partial x} \psi_i(x,y) \right) \left( \frac{\partial}{\partial y} \psi_j(x,y) \right) dx\,dy \\
         &\quad -\frac{1}{2}\tau_y^2 \displaystyle \int_\Omega \left( \frac{\partial}{\partial y} \psi_i(x,y) \right) \left( \frac{\partial}{\partial y} \psi_j(x,y) \right) dx\,dy \\
  \mathbf{p}_i(x,y) &= \displaystyle \int_\Omega p(x,y,0) \psi_i(x,y) dx\,dy.
\end{align*}
The entries of $S_{ij}$ are computed with integration by parts and
using the boundary conditions. The semidiscrete Galerkin approximation
becomes
\[
  p_k(x,y,t) = \boldsymbol{\psi}(x,y)^T \exp\left( M^{-1}S\, t \right) \mathbf{c}(0),
\]
with $\boldsymbol{\psi}(x,y) = (\psi_0(x,y), \ldots, \psi_k(x,y))^T.$

A bound on the closeness of the approximate solution $p_k(x,y,t)$ to
the strong solution $p(x,y,t)$ is developed in \cite{bramble1977some}
shows that the Galerkin approximation we use converges to the strong
solution, and it motivates the thrust of our numerical
solution. First, we define the \textit{error} term
\[
  e_k(t) = p(x,y,t) - p_k(x,y,t),
\]
as well as the norm
\[
  \| w \|_2 = \sum_{j=0}^\infty \lambda_j^2 (w, \phi_j)
\]
for the eigenpairs $(\lambda_j, \phi_j)$ of the operator
$\mathcal{L}$. As referred in \cite{bramble1977some}, functions
$w \in L_2(\Omega)$ with $\|w\|_2 < \infty$ are also in $W_2^2(\Omega)$. In a
sense, $\| \cdot \|_2$ measures the variation of functions. Functions
having higher-order eigenmodes have a bigger norm, and vice
versa. Finally, if we have the condition (corresponding to equation
2.1 in \cite{bramble1977some})
\begin{align}
  \| p(x,y,0) - p_k(x,y,0) \|_{L_2(\Omega)} &\leq C\, h(k)^2 \| p(x,y,0) \|_2. \label{eq:ic-bound}
\end{align}
as a decreasing function $h(k)$, Theorem 2.1 in \cite{bramble1977some}
applies and we have the error estimate
\begin{align}
  \| e(t) \|_{L_2(\Omega)} \leq C h(k)^2 \| p(x,y,0) \|_{2}. \label{eq:error-est}
\end{align}
We can ensure condition (\ref{eq:ic-bound}) is met if $S_k$ is complete
in $L_2(\Omega)$ as $k$ grows. The other two conditions for Theorem
2.1 to apply are demonstrated by \cite{bramble1977some} for the
Galerkin method. Equation (\ref{eq:error-est}) can be summarized in a
simple way: the error of the method is controlled by how much
variation the initial condition has; the rate of decrease of the error
is controlled by how well the span of $S_k$ represents the initial
condition compared to the variation of the initial condition.

However, the initial condition for (\ref{eq:qq}) requires all in
eigenmodes being included in the representation of the initial
conditions, so that
\[
  \| p(x,y,0) \|_{2} = \left\| \delta\left( x - \frac{x_0 - a_x}{b_x-a_x}
  \right) \delta\left( y - \frac{y_0 - a_y}{b_y-a_y} \right)\right\|_{2} = +\infty.
\]
It is obvious that we cannot apply the error estimates above. However,
evolving the solution forward in time, even for a short period,
diffuses the delta-function IC and attenuates out the highest
frequency modes. Further, because solutions to the diffusion equation
are smooth, $p(x,y,t_\epsilon)$ will give us a smooth initial function
which will make the error bounds admissible.

\subsubsection{Small-time Solution}
The small-time solution is derived by considering the fundamental
solution $G(x,y,t)$ for the unbounded problem in (\ref{eq:qq}), which
is the bivariate Gaussian density with mean and covariance determined
by the initial condition and the diffusion parameters
\citep{stakgold2011green}. We can then find a small enough
$t_\epsilon$ such that
$G\left(x,y, t_\epsilon \left| \frac{x_0-a_x}{b_x - a_x},
    \frac{y_0-a_y}{b_y - a_y} \right.\right)$ is numerically zero on
the boundaries of $\Omega$. This is done in the following way.
\begin{enumerate}[1.]
\item Scale and rotate the coordinate axes by
  \[
    \left( \begin{array}{c}
             \xi \\
             \eta
           \end{array} \right) = \mbox{Transformation goes here}
             \left( \begin{array}{cc}
             x \\
             y
           \end{array} \right) 
       \]
       so that the problem obeys the diffusion equation. The
       computational domain is transformed to the parallelogram
       $\tilde{\Omega}$. The transformed initial condition will be
       denoted as $(\xi_0, \eta_0)$.

     \item The fundamental solution $G(\xi,\eta,t | \xi_0, \eta_0)$ in
       this coordinate frame which does not take into account
       boundaries follows the bivariate Gaussian probability density
       function
  \[
    G(\xi,\eta,t | \xi_0, \eta_0) = \frac{1}{2\pi t} \exp\left(-\frac{(\xi-\xi_0)^2 +
        (\eta-\eta_0)^2}{2\,t} \right).
  \]
  We define \textbf{distance} between $G(\xi,\eta,t | \xi_0, \eta_0)$
  and any of the linear boundaries of $\tilde{\Omega}$ as the
  Euclidean norm of the segment of the vector extending from
  $(\xi_0,\eta_0)$ and normal to the boundary. There are four such
  distances $d_1, d_2, d_3, d_4$. Assume that they are listed in
  increasing magnitude.

  Setting $t_\epsilon = d_2/8$ ensures that the fundamental solution
  $G(\xi,\eta,t_\epsilon | \xi_0, \eta_0)$ is \textit{at most}
  $\approx 10^{-15}$ on the second-farthest boundary, as well as the
  other two farthest boundaries. In this way,
  $G(\xi,\eta,t_\epsilon | xi_0, \eta_0)$ satisfies the boundary
  condition on the three farthest boundaries numerically.

\item Reflect the point $(\xi_0, \eta_0) \to (\xi_0', \eta_0')$ about
  the closest boundary. The image function
  $G(\xi,\eta,t_\epsilon | \xi_0', \eta_0')$ satisfies the diffusion
  equation and is equal to $G(\xi,\eta,t_\epsilon | \xi_0, \eta_0)$ on
  the closest boundary. Further,
  $G(\xi,\eta,t_\epsilon | \xi_0', \eta_0')$ takes on values less than
  $10^{-15}$ on all other boundaries, because it is outside of
  $\bar{\Omega}$.

  Considering the difference of the two images, the small-time solution 
  \[
    p(\xi,\eta,t_\epsilon) := G(\xi,\eta,t_\epsilon | \xi_0, \eta_0) -
    G(\xi,\eta,t_\epsilon | \xi_0', \eta_0')
  \]
  satisfies all of the boundaries numerically and also satisfies the
  governing diffusion equation. Performing a change of variables
  produces the small-time solution $p(x,y,t_\epsilon)$.
\end{enumerate}
Using Theorem 5.E of \cite{zeidler1995applied}, we can solve for
$p(x,y,t)$ by considering the smooth $p(x,y,t_\epsilon)$ as an initial
condition and evolving it forward in time by $t-t_\epsilon$. The
bigger $t_\epsilon$, the less eigenmodes present in
$p(x,y,t_\epsilon)$, and the more accurate our weak solution according
to the error estimate
\[
  \| e(t) \|_{L_2(\Omega)} \leq C h(k)^2 \| p(x,y,t_\epsilon) \|_{2}.
\]

\subsubsection{Orthonormal Basis Family}
An upper bound of the rate at which the Galerkin approximation
converges to $p(x,y,t)$ is given by condition (\ref{eq:ic-bound}),
namely by how well the initial condition may be approximated via a
projection onto $S_k$. We motivate the construction of the orthonormal
basis functions by once again considering the fundamental solution for
the unbounded problem (\ref{eq:qq}). In the absence of boundaries,
(\ref{eq:qq}) is solved by the function
\[
  G(x,y,t | x_0', y_0') = \frac{1}{2\pi\,\,t\,\, \tau_x\tau_y\sqrt{1-\rho^2}} \exp\left\{ -\frac{1}{2(1-\rho^2)} \left( \frac{(x - x_0')^2}{\tau_x^2} - 2\rho \frac{(x-x_0')(y-y_0')}{\tau_x\tau_y} + \frac{(y - y_0')^2}{\tau_y^2}\right) \right\},
\]
with
$x_0' = (x_0 - a_x)/(b_x-a_x),\quad y_0' = (y_0 - a_y)/(b_y-a_y)$. We
choose the family of basis functions
$S_k = (\psi_0(x,y), \ldots, \psi_k(x,y))$
\begin{align}
  \psi_i(x,y) &= \frac{1}{2\pi \sigma^2\sqrt{1-\tilde{\rho}^2} } \exp\left\{ -\frac{1}{2(1-\tilde{\rho}^2)\sigma^2} \left( (x - x_i)^2 - 2\tilde{\rho} (x-x_i)(y-y_i) + (y - y_i)^2 \right) \right\} x\left(1-x\right)\, y(1-y)
\end{align}
for some parameters $(\tilde{\rho}, \sigma)$ and a collection of
points $\{ (x_i,y_i) \}_{i=0}^k$ which form a grid over
$\Omega$. Essentially, the collection
$\{ \psi_i(x,y| x_i, y_i, \tilde{\rho}, \sigma) \}_{i=0}^k$ is
composed of fundamental solutions to a heat diffusion problem tuned by
$\sigma$ and $\tilde{\rho}$, tampered such that their support is on
$\Omega$, zero on the boundaries, centered on some grid over $\Omega$,
and still smooth. In this manner, our basis function choice is in
keeping with the golden rule for the rate of convergence: \textbf{The
  smoother the solution to the original problem and the smoother the
  functions in the basis space, the faster the convergence of the
  Ritz-Galerkin method.} (Remark 1(c) in Chapter 2 of
\cite{zeidler1995applied}). Moreover, all of the basis functions are
infinitely differentiable with respect to the boundary and diffusion
parameters, so that
$\frac{\partial^4 p_k(x,y,t)}{\partial a_x \partial b_x \partial
  a_y \partial b_y}$ exists and can be computed numerically using
finite differences.

The other critical advantage of these elements is that they better
resolve the small-time solution $p(x,y,t_\epsilon)$ by taking into
account correlation in the covariance of each kernel.
[\textbf{\color{red} if not prove, demonstrate. I'm running out of
  steam on this.}].

\section{Estimation}
Consider the problem of estimating the parameters
$(\mu_x, \mu_y, \sigma_x, \sigma_y, \rho)$ from an i.i.d. sample
$(Z_1, \ldots, Z_n)$ from the process in (\ref{eq:X}) - (\ref{eq:Y}),
where each $Z_i$ is the vector of random variables
\[
  Z_i = (X_i(t), Y_i(t), A_{x,i}, B_{x,i}, A_{y,i}, B_{y,i}).
\]
We say that $Z_i$ is sampled from the distribution corresponding to
the probability density function (\ref{eq:pdf})
\[
  Z_i \sim F(\theta),
\]
where the distribution $F$ has the usual interpretation
\begin{align*}
  F(z = (x, y, a_x, b_x, a_y, b_y) | \theta) &= \Pr\left(X(t) \leq x,
    Y(t) \leq y, \min_{t' \leq t} X(t') \leq a_x, \max_{t' \leq t}
    X(t') \leq b_x, \min_{t' \leq t} Y(t') \leq a_y, \max_{t' \leq t}
                                               Y(t') \leq b_y\right).
\end{align*}

This estimation problem is of particular importance in quantitative
finance where the model equations (\ref{eq:X}) - (\ref{eq:Y}) (with
various bells and whistles attached) are widely used. However, to the
best of our knowledge, all current \textit{likelihood} methods in the
literature either ignore the observed maximum/minimum information or
use some of it. Likelihood-free approaches, like that of
\cite{rogers1991estimating}, on the other hand suffer from not being
able to be easily integrated into a bigger inferential framework.

Since we do not have a closed-form solution for the likelihood, we
will use an iterative derivative-free maximization algorithm (the
Nelder-Mead method; see \cite{lagarias1998convergence} for review and
convergence properties) which requires repeated evauluation of the
likelihood. For moderate to large samples sizes this is feasible,
because our numerical method is specifically designed for
computational efficiency for repeatedly evaluating the density
function (\ref{eq:pdf}). The maximum likelihood estimator (MLE) for
the true parameters based on $n$ samples, which we call
$\hat{\theta}_n := (\hat{\mu}_x, \hat{\mu}_y, \hat{\sigma}_x,
\hat{\sigma}_y, \hat{\rho})$, is especially useful in practical
settings when it exhibits \textit{consistency}, \textit{i.e.}: the MLE
gets closer to the true parameter vector $\theta$ as more data is
collected and included in the likelihood (assuming the model and its
parameters remain constant during the data collection). More
precise, the estimator $\hat{\theta}_n$ is consistent if
\[
  \Pr( | \hat{\theta}_n - \theta | ) \to 0 \qquad \mbox{as} \qquad n \to \infty.
\]

\subsection{Consistency}
In this section we prove that the MLE based on the Galerkin
approximation $p_k(x,y,t)$ to the governing Fokker-Planck equation
(\ref{eq:1}) is consistent. To do so, we first show that the
distribution on $Z$ based on the approximate $p_k(x,y,t)$ converges to
the true distribution $F$:
\[
  \lim_{k \to \infty} F_k(z | \theta) = F(z | \theta).
\]
We define $F_k(z | \theta)$ as
\begin{align}
  F_k(z | \theta) &= \displaystyle \int_{-\infty}^x \displaystyle \int_{-\infty}^y \displaystyle \int_{-\infty}^{a_x} \displaystyle \int_{-\infty}^{a_y} \displaystyle \int_{-\infty}^{b_x} \displaystyle \int_{-\infty}^{b_y} \frac{\partial^4}{\partial a'_x \partial b'_x \partial a'_y \partial b'_y} q_k(x', y', t)\,\,\, dx'\, dy'\, da'_x\, db'_x\, da'_y\, db'_y, \\
  &:= \Pr_{k} \left(X(t) \leq x,
    Y(t) \leq y, \min_{t' \leq t} X(t') \leq a_x, \max_{t' \leq t}
    X(t') \leq b_x, \min_{t' \leq t} Y(t') \leq a_y, \max_{t' \leq t}
                                               Y(t') \leq b_y\right)
\end{align}
where $q_k$ is the approximation to $q$ based on the $p_k$. Put
simply, we will show that the distribution based on the approximate
density approaches the true density as $k \to \infty$.

A key idea in the proof will be the interpretation of the measure $q(x,y,t)$ 

\begin{lemma} \label{lem:1}
  For $F_k$ and $F$ defined above, $\lim_{k\to \infty} F_k(z) = F(z).$
\end{lemma}
\begin{proof}
  \textbf{\color{red}do not read below; I'm working on it tonight}
  The weak solution $q_k(x,y,t)$ has the property that for any
  $C^{\infty}(\Omega)$ function $v \in L^2(\Omega)$,
  \[ \underset{k \to \infty}{\lim}(q_k,v) = (q,v), \] where
  $(v,u) := \displaystyle \int_{a_x}^{b_x} \int_{a_y}^{b_y}
  v(x,y,t)u(x,y,t) dx\,dy.$ and $q$ is the strong solution.

  Recalling equation (\ref{eq:CDF}), the strong solution has the probabilistic interpretation
  \begin{align*}
    q(x,y,t) &= \Pr\left(X(t) \in dx, Y(t) \in dy, a_x < X(t') < b_x, a_y < Y(t') < b_y |0 < t' \leq t, X(0)=x_0, Y(0)=y_0, \theta \right) \\
     &= \Pr\left(X(t) \in dx, Y(t) \in dy, \min_{t'}X(t') \geq a_x,
  \max_{t'}X(t') \leq b_x, \min_{t'} Y(t') \geq a_y, \max_{t'} Y(t') \leq b_y \left| 0 <
  t' \leq t, X(0)=x_0, Y(0)=y_0, \theta \right.\right)
  \end{align*}
\end{proof}


\begin{lemma}
  The maximum likelihood estimator is consistent as $n \to \infty$ and $k \to \infty$:
  \[ \hat{\theta}_{n,k} \to \theta \].
\end{lemma}
\begin{proof}
  \textbf{\color{red} It's a bit sloppy. Also, we do not need
    asymptotic efficiency to prove equation (\ref{eq:var-lim})} By
  Lemma \ref{lem:1}
    \[ X_k \xrightarrow[]{d} X \mbox { as } k \to \infty. \] Next,
    given Theorem 4.1 in \cite{singler2008differentiability}, we know
    that, for each $k$, $q_k$ is analytic in both the diffusion
    parameters and boundary parameters. Hence, the probability density
    function satisfies the criteria A1 - A6 in
    \cite{casella2002statistical} to guarantee that, for data
    $X_{k} \sim F_k(\theta)$,
    \[ \hat{\theta}_{n,k}(X_k) \xrightarrow[]{p} \theta \mbox{ as } n
      \to \infty. \]

    Now we need to show that the same holds for data sampled from $F$
    as $k \to \infty$. To do this, we will use Chebyshev's inequality:
  \[
    \Pr_{X}\left( \left| \hat{\theta}_{n,k}(X) - \theta \right| \geq
      \epsilon \right) \leq \frac{ \mbox{E}_{X}\left[
        (\hat{\theta}_{n,k}(X) - \theta)^2 \right] }{ \epsilon^2 }.
  \]
  By the Maximum theorem [REFERENCE], $\hat{\theta}_{n,k}(x)$ is a continuous
  function with respect to $x$, and further because we have bounded
  $\hat{\theta}$ from below and above,
  \[
    \mbox{E}_{X_k}\left[ (\hat{\theta}_{n,k}(X_k) - \theta)^2 \right]
    \to \mbox{E}_{X}\left[ (\hat{\theta}_{n,k}(X) - \theta)^2 \right]
    \mbox{ as } k \to \infty
  \]
  by the portmanteau lemma. Finally, we can show that
  \begin{equation}
    \mbox{E}_{X_k}\left[ (\hat{\theta}_{n,k}(X_k) - \theta)^2 \right]
    \to 0 \mbox{ as } n \to \infty, \label{eq:var-lim}
  \end{equation}
  since the expected value of the estimator tends to $\theta$ and its
  variance goes to 0 when $n \to \infty$. Therefore, given any
  $\epsilon > 0$ and $\delta > 0$, we can find a sufficiently large
  $n$ and $k$ such that
  \[
    \Pr_{X}\left( \left| \hat{\theta}_{n,k}(X) - \theta \right| \geq
      \epsilon \right) \leq \frac{ \mbox{E}_{X}\left[
        (\hat{\theta}_{n,k}(X) - \theta)^2 \right] }{ \epsilon^2 } < \delta    
  \]
\end{proof}


\bibliographystyle{plainnat}
\bibliography{master-bibliography}
\end{document}
