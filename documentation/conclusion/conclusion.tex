\label{ch:conclusion}

We have presented novel approaches for using two types of financial
data for statistical inference and prediction. We have demonstrated
the effectiveness of a Bayesian filtering approach in estimating
financial market volatility in noisy, high-frequency data. This method
outperforms the currently popular averaging approaches found in the
realized volatility literature in estimating integrated volatility on
a daily timescale. As part of our approach, we have elicited priors
invariant with respect to sampling frequency, and have formulated the
used discrete-time state-space models to be coherent over sampling
frequencies as well. Future work on this project includes extending
the problem to multiple assets. Given the nature of volatility shocks
in markets, namely their occurrence over a single or a few days, the
high-frequency resolution of the volatility paths and potentially
fast-changing correlations among assets are of interest to both
practitioners and academics in the field.

We have also developed a semi-analytic computational framework for
volatility estimation for bivariate assets with OCHL data, a
completely novel contribution. As part of our approach, we have
developed a non-separable variational-form basis expansion (Galerkin
solution) to the governing differential equation. Our numerical
Galerkin solver takes directly into account the correlation structure
inherent in the bivariate OCHL problem and is therefore more efficient
when compared to other proposed numerical methods. However, in the
context statistical inference, often the likelihood in question needs
to be evaluated in parameter regions where near-infinite computational
memory and time needs to be employed in order to produce a valid
numerical solution. To deal with this difficulty, we in turn developed
an analytic solution asymptotically valid in this (small-time, under
our proposed representation of the canonical problem) parameter
region. This was achieved via the method of images and a symmetry
argument. Next, we proposed a matching solution which interpolates
between the regions where the Galerkin and small-time solutions are
applicable. Finally, we performed a series of MLE exercises that
showed the validity of our method and also demonstrated the increase
in statistical power in using OCHL data in comparison to both the
classical Gaussian MLE estimator that disregards boundary conditions as
well as the other existing bivariate OCHL estimator.

A natural extension of our OCHL solution is its application in more
dynamical settings. In particular, when combined with a Gaussian
process emulator, it is possible to implement the OCHL likelihood
solution in the estimation of stochastic volatility dynamical models
with the use of particle filters.
