\documentclass[10pt]{article}

\usepackage{amsthm}
\usepackage[toc,page]{appendix}
\usepackage{amssymb}
\usepackage{bm}
\usepackage{pslatex,palatino,avant,graphicx,color}
\usepackage{colortbl}
\usepackage{fullpage}
\usepackage{mathtools}
\usepackage{natbib}
\usepackage{amsmath}
\usepackage{caption}
%\usepackage{subfigure}
\usepackage{subcaption}
\usepackage{bm}
\usepackage{wrapfig}
\usepackage{enumerate}
\usepackage{rotating}
\usepackage{multirow}
\usepackage{tabularx}
\usepackage{tikz}
\usepackage{pgfplots}
\usepackage{bbm}
\usepackage[titles,subfigure]{tocloft} % Alter the style of the Table of Contents

\newtheorem{lemma}{Lemma}
%\pdfminorversion=4
% NOTE: To produce blinded version, replace "0" with "1" below.
\newcommand{\blind}{0}

% DON'T change margins - should be 1 inch all around.
\addtolength{\oddsidemargin}{0in}%
\addtolength{\evensidemargin}{0in}%
\addtolength{\textwidth}{0in}%
\addtolength{\textheight}{0in}%
\addtolength{\topmargin}{0in}%


\renewcommand{\cftsecfont}{\rmfamily\mdseries\upshape}
\renewcommand{\cftsecpagefont}{\rmfamily\mdseries\upshape} % No bold!

\newcommand{\indicator}[1]{\mathbbm{1}\left( #1 \right) }
\newcommand{\hb}{\hat{b}}
\newcommand{\ha}{\hat{a}}
\newcommand{\htheta}{\hat{\theta}}
\newcommand{\halpha}{\hat{\alpha}}
\newcommand{\hmu}{\hat{\mu}}
\newcommand{\hsigma}{\hat{\sigma}}
\newcommand{\hphi}{\hat{\phi}}
\newcommand{\htau}{\hat{\tau}}
\newcommand{\heta}{\hat{\eta}}
\newcommand{\E}[1]{\mbox{E}\left[#1\right]}
\newcommand{\Var}[1]{\mbox{Var}\left[#1\right]}
\newcommand{\Indicator}[1]{\mathbbm{1}_{ \left( #1 \right) } }
\newcommand{\dNormal}[3]{ N\left( #1 \left| #2, #3 \right. \right) }
\newcommand{\Beta}[2]{\mbox{Beta}\left( #1, #2 \right)}
\newcommand{\alphaphi}{\alpha_{\hphi}}
\newcommand{\betaphi}{\beta_{\hphi}}
\newcommand{\expo}[1]{ \exp\left\{ #1 \right\}}
\newcommand{\tauSquareDelta}{\htau^2
  \left(\frac{1-\expo{-2\htheta\Delta}}{2\htheta} \right)}
\newcommand{\mumu}{\mu_{\hmu}}
\newcommand{\sigmamu}{\sigma^2_{\hmu}}
\newcommand{\sigmamuexpr}{\log\left( \frac{\VarX}{\EX^2} + 1 \right)}
\newcommand{\mumuexpr}{\log(\EX) -  \log\left( \frac{\VarX}{\EX^2} + 1 \right) /2 }

\newcommand{\EX}{\mbox{E}\left[ X \right] }
\newcommand{\VarX}{\mbox{Var}\left[ X \right] }
\newcommand{\mueta}{\mu_{\heta} }
\newcommand{\sigmaeta}{\sigma^2_{\heta}}
\newcommand{\sigmaetaexpr}{ \log\left( \frac{\VarX}{\EX^2} + 1 \right) }
\newcommand{\muetaexpr}{ \log(\EX) -  \sigmaetaexpr /2 }

\newcommand{\mualpha}{\mu_{\halpha} }
\newcommand{\sigmaalpha}{\sigma^2_{\halpha}}
\newcommand{\sigmaalphaexpr}{ \log\left( \frac{\VarX}{\EX^2} + 1 \right) }
\newcommand{\mualphaexpr}{ \log(\EX) -  \sigmaalphaexpr /2 }

\newcommand{\mutauexpr}{ \frac{2}{T} \EX }
\newcommand{\sigmatauexpr}{ \frac{4}{T^2} \Var{X}}

\newcommand{\alphatau}{\alpha_{\htau^2}}
\newcommand{\betatau}{\beta_{\htau^2}}

\newcommand{\Gam}[2]{\mbox{Gamma}\left( #1, #2 \right) }
\newcommand{\InvGam}[2]{\mbox{Inv-Gamma}\left( #1, #2 \right) }

%%% END Article customizations

%%% The "real" document content comes below...

% \newbox{\LegendeA}
% \savebox{\LegendeA}{
%    (\begin{pspicture}(0,0)(0.6,0)
%    \psline[linewidth=0.04,linecolor=red](0,0.1)(0.6,0.1)
%    \end{pspicture})}
% \newbox{\LegendeB}
%    \savebox{\LegendeB}{
%    (\begin{pspicture}(0,0)(0.6,0)
%    \psline[linestyle=dashed,dash=1pt 2pt,linewidth=0.04,linecolor=blue](0,0.1)(0.6,0.1)
%    \end{pspicture})}

\title{A Range-Based Bivariate Stochastic Volatility Model: Towards Closed-Form Solution}
\author{Georgi Dinolov, Abel Rodriguez, Hongyun Wang}
\date{} % Activate to display a given date or no date (if empty),
         % otherwise the current date is printed

\begin{document}
\def\spacingset#1{\renewcommand{\baselinestretch}%
{#1}\small\normalsize} \spacingset{1}

\bigskip

\vspace{1cm}



\subsection{A small-parameter analytic solution}
In the previous chapter we introduced a numerical solution that works
for relatively benign parameter ranges. However, the numerical
solution fails for small parameters of $(\tilde{\sigma},
\tilde{t})$. Here, we introduce an analytic solution that works for
certain small-$\tilde{\sigma}$.

\subsubsection{Illustration with $\rho=0$}
Consider the normalized diffusion problem with $\rho=0$ and an intial
condition $(x_0, y_0)$. As in Chapter 2, we scale, rotate, and scale
again, so that the fundamental solution is $N(x,y|t)$. However, since
$\rho=0$, only the initial scaling is sufficient to perform
reflections. Thus, the effective problem domain is a rectangle of
height $1/\tilde{\sigma}$. We can pick a $\tilde{\sigma}$ small enough
such that Now the fundamental solution is $N(x,y|t)$. Reflecting about
boundary 1 enforces the BC there. Then, reflecting the system about
boundary 2 enforces BC there, but the additional images (labeled by 3
and red) disturb the

The approach we use to motivate the calculations for the
small-parameter solution is considering the problem with
$\rho=0$. Here we consider the normalized problme such that
$\tilde{\sigma} \leq 1$ and the diffusion problem is
\[
  \frac{\partial q}{\partial t} = \frac{1}{2}\frac{\partial^2 q}{\partial x^2} + \frac{1}{2}\tilde{\sigma}^2 \frac{\partial^2 q}{\partial y^2}.
\]
Now consider a configuration of images generated by reflecting once
about each boundary. For a given $(\rho=0, \tilde{t})$, we can find a
sufficiently small $\tilde{\sigma}$ such that at least one of the 24
possible configurations satisfies the boundary conditions either
analytically or numerically. In this configuration, only one of the 16
images has a location parameter that is a function of all four
boundaries, and it is the only image that contributes to $\partial_{\Omega} q$. 


\end{document}