\documentclass[10pt]{article}

\usepackage{amsthm}
\usepackage[toc,page]{appendix}
\usepackage{amssymb}
\usepackage{bm}
\usepackage{pslatex,palatino,avant,graphicx,color}
\usepackage{colortbl}
\usepackage{fullpage}
\usepackage{mathtools}
\usepackage{natbib}
\usepackage{amsmath}
\usepackage{caption}
%\usepackage{subfigure}
\usepackage{subcaption}
\usepackage{bm}
\usepackage{wrapfig}
\usepackage{enumerate}
\usepackage{rotating}
\usepackage{multirow}
\usepackage{tabularx}
\usepackage{tikz}
\usepackage{pgfplots}
\usepackage{bbm}
\usepackage[titles,subfigure]{tocloft} % Alter the style of the Table of Contents

\newtheorem{lemma}{Lemma}
%\pdfminorversion=4
% NOTE: To produce blinded version, replace "0" with "1" below.
\newcommand{\blind}{0}

% DON'T change margins - should be 1 inch all around.
\addtolength{\oddsidemargin}{0in}%
\addtolength{\evensidemargin}{0in}%
\addtolength{\textwidth}{0in}%
\addtolength{\textheight}{0in}%
\addtolength{\topmargin}{0in}%


\renewcommand{\cftsecfont}{\rmfamily\mdseries\upshape}
\renewcommand{\cftsecpagefont}{\rmfamily\mdseries\upshape} % No bold!

\newcommand{\indicator}[1]{\mathbbm{1}\left( #1 \right) }
\newcommand{\hb}{\hat{b}}
\newcommand{\ha}{\hat{a}}
\newcommand{\htheta}{\hat{\theta}}
\newcommand{\halpha}{\hat{\alpha}}
\newcommand{\hmu}{\hat{\mu}}
\newcommand{\hsigma}{\hat{\sigma}}
\newcommand{\hphi}{\hat{\phi}}
\newcommand{\htau}{\hat{\tau}}
\newcommand{\heta}{\hat{\eta}}
\newcommand{\E}[1]{\mbox{E}\left[#1\right]}
\newcommand{\Var}[1]{\mbox{Var}\left[#1\right]}
\newcommand{\Indicator}[1]{\mathbbm{1}_{ \left( #1 \right) } }
\newcommand{\dNormal}[3]{ N\left( #1 \left| #2, #3 \right. \right) }
\newcommand{\Beta}[2]{\mbox{Beta}\left( #1, #2 \right)}
\newcommand{\alphaphi}{\alpha_{\hphi}}
\newcommand{\betaphi}{\beta_{\hphi}}
\newcommand{\expo}[1]{ \exp\left\{ #1 \right\}}
\newcommand{\tauSquareDelta}{\htau^2
  \left(\frac{1-\expo{-2\htheta\Delta}}{2\htheta} \right)}
\newcommand{\mumu}{\mu_{\hmu}}
\newcommand{\sigmamu}{\sigma^2_{\hmu}}
\newcommand{\sigmamuexpr}{\log\left( \frac{\VarX}{\EX^2} + 1 \right)}
\newcommand{\mumuexpr}{\log(\EX) -  \log\left( \frac{\VarX}{\EX^2} + 1 \right) /2 }

\newcommand{\EX}{\mbox{E}\left[ X \right] }
\newcommand{\VarX}{\mbox{Var}\left[ X \right] }
\newcommand{\mueta}{\mu_{\heta} }
\newcommand{\sigmaeta}{\sigma^2_{\heta}}
\newcommand{\sigmaetaexpr}{ \log\left( \frac{\VarX}{\EX^2} + 1 \right) }
\newcommand{\muetaexpr}{ \log(\EX) -  \sigmaetaexpr /2 }

\newcommand{\mualpha}{\mu_{\halpha} }
\newcommand{\sigmaalpha}{\sigma^2_{\halpha}}
\newcommand{\sigmaalphaexpr}{ \log\left( \frac{\VarX}{\EX^2} + 1 \right) }
\newcommand{\mualphaexpr}{ \log(\EX) -  \sigmaalphaexpr /2 }

\newcommand{\mutauexpr}{ \frac{2}{T} \EX }
\newcommand{\sigmatauexpr}{ \frac{4}{T^2} \Var{X}}

\newcommand{\alphatau}{\alpha_{\htau^2}}
\newcommand{\betatau}{\beta_{\htau^2}}

\newcommand{\Gam}[2]{\mbox{Gamma}\left( #1, #2 \right) }
\newcommand{\InvGam}[2]{\mbox{Inv-Gamma}\left( #1, #2 \right) }

%%% END Article customizations

%%% The "real" document content comes below...

% \newbox{\LegendeA}
% \savebox{\LegendeA}{
%    (\begin{pspicture}(0,0)(0.6,0)
%    \psline[linewidth=0.04,linecolor=red](0,0.1)(0.6,0.1)
%    \end{pspicture})}
% \newbox{\LegendeB}
%    \savebox{\LegendeB}{
%    (\begin{pspicture}(0,0)(0.6,0)
%    \psline[linestyle=dashed,dash=1pt 2pt,linewidth=0.04,linecolor=blue](0,0.1)(0.6,0.1)
%    \end{pspicture})}

\title{A Range-Based Bivariate Stochastic Volatility Model}
\author{Georgi Dinolov, Abel Rodriguez, Hongyun Wang}
\date{} % Activate to display a given date or no date (if empty),
         % otherwise the current date is printed 

\begin{document}
\def\spacingset#1{\renewcommand{\baselinestretch}%
{#1}\small\normalsize} \spacingset{1}

\bigskip

\vspace{1cm}
\noindent

\spacingset{1.00} % 
\section{Introduction}

The estimation and prediction of price volatility from market data is
an important problem in econometrics and finance
\citep{abramov2007estimation}, as well as practical risk management
\citep{brandt2006dynamic}. The literature on the subject of volatility
estimation is vast. Model-based approaches for a single observable
asset begin with the ARCH and GARCH models of \cite{engle1982} and
\cite{bollerslev1986}, moving on to stochastic volatility models (see
\cite{shephard2005selected-readings}, for example).

Multivariate equivalents for each of these model classes exist (see
\cite{bauwens2006multivariate} and \cite{asai2006multivariate} for
reviews of multivariate GARCH and for multivariate stochastic
volatility, respectively). However, the majority of work on the
subject uses opening and closing prices as data. This approach
invariably disregards information traditionally contained in financial
timeseries: the observed high and low price of an asset over the
quoted periods. To our current knowledge, only \cite{rodriguez2012}
use the observed maximum and minimum of prices in a likelihood to
estimate volatility. They do so, however, in a univariate
setting.

Explicit model-based approaches in the multivariate setting which take
into account extrema over observational periods are completely lacking
in the literature, because deriving an efficient approximation of the
corresponding likelihood function has hereto been an open problem
[cite?]. In this paper, we use a result addressing this problem and
introduce a novel, \textit{bivariate} stochastic volatility model
which takes into account the highest and lowest observed prices of
each asset as part of a likelihood-based (Bayesian) estimation
procedure.

\section{Model}
The model we will use is a bivaraite 1-factor stochastic volatility
model with leverage:

\begin{align}
  \left( \begin{array}{c}
           x_t \\
           y_t
         \end{array} \right) &= \left( \begin{array}{c}
                                         x_{t-\Delta} \\
                                         y_{t-\Delta}
                                       \end{array} \right) +
  \left( \begin{array}{c}
           \mu_x \\
           \mu_y \end{array} \right) +
  \left( \begin{array}{cc}
           \sqrt{1-\rho_t^2}\sigma_{x,t} & \rho_t \sigma_{x,t} \\
           0 & \sigma_{y,t}
         \end{array} \right)
               \left( \begin{array}{c}
                        \epsilon_{x,t} \\
                        \epsilon_{y,t}
                      \end{array} \right) \\
  \log(\sigma_{x,t+\Delta}) &= \alpha_x + \theta_x(\log(\sigma_{x,t}) - \alpha_x) + \tau_x \eta_{x,t} \\
  \log(\sigma_{y,t+\Delta}) &= \alpha_y + \theta_y(\log(\sigma_{y,t}) - \alpha_y) + \tau_y \eta_{y,t} \\
  \mbox{logit}((\rho_{t+\Delta} + 1)/2) &= \mbox{logit}((\rho_{t}+1)/2) + \tau_{\rho} \eta_{\rho,t}
\end{align}

\bibliographystyle{plainnat}
\bibliography{master-bibliography}

\end{document}