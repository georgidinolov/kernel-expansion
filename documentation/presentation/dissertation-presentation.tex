%%%%%%%%%%%%%%%%%%%%%%%%%%%%%%%%%%%%%%%%%
% Beamer Presentation
% LaTeX Template
% Version 1.0 (10/11/12)
%
% This template has been downloaded from:
% http://www.LaTeXTemplates.com
%
% License:
% CC BY-NC-SA 3.0 (http://creativecommons.org/licenses/by-nc-sa/3.0/)
%
%%%%%%%%%%%%%%%%%%%%%%%%%%%%%%%%%%%%%%%%%

%----------------------------------------------------------------------------------------
%	PACKAGES AND THEMES
%----------------------------------------------------------------------------------------

\documentclass{beamer}

\mode<presentation> {

% The Beamer class comes with a number of default slide themes
% which change the colors and layouts of slides. Below this is a list
% of all the themes, uncomment each in turn to see what they look like.

%\usetheme{default}
%\usetheme{AnnArbor}
%\usetheme{Antibes}
%\usetheme{Bergen}
%\usetheme{Berkeley}
%\usetheme{Berlin}
%\usetheme{Boadilla}
\usetheme{CambridgeUS}
%\usetheme{Copenhagen}
%\usetheme{Darmstadt}
%\usetheme{Dresden}
%\usetheme{Frankfurt}
%\usetheme{Goettingen}
%\usetheme{Hannover}
%\usetheme{Ilmenau}
%\usetheme{JuanLesPins}
%\usetheme{Luebeck}
%\usetheme{Madrid}
%\usetheme{Malmoe}
%\usetheme{Marburg}
%\usetheme{Montpellier}
%\usetheme{PaloAlto}
%\usetheme{Pittsburgh}
%\usetheme{Rochester}
%\usetheme{Singapore}
%\usetheme{Szeged}
%\usetheme{Warsaw}

% As well as themes, the Beamer class has a number of color themes
% for any slide theme. Uncomment each of these in turn to see how it
% changes the colors of your current slide theme.

%\usecolortheme{albatross}
%\usecolortheme{beaver}
%\usecolortheme{beetle}
%\usecolortheme{crane}
%\usecolortheme{dolphin}
%\usecolortheme{dove}
%\usecolortheme{fly}
%\usecolortheme{lily}
%\usecolortheme{orchid}
%\usecolortheme{rose}
%\usecolortheme{seagull}
%\usecolortheme{seahorse}
%\usecolortheme{whale}
%\usecolortheme{wolverine}

%\setbeamertemplate{footline} % To remove the footer line in all slides uncomment this line
%\setbeamertemplate{footline}[page number] % To replace the footer line in all slides with a simple slide count uncomment this line

%\setbeamertemplate{navigation symbols}{} % To remove the navigation symbols from the bottom of all slides uncomment this line
}

\usepackage{graphicx} % Allows including images
\usepackage{booktabs} % Allows the use of \toprule, \midrule and \bottomrule in tables

%----------------------------------------------------------------------------------------
%	TITLE PAGE
%----------------------------------------------------------------------------------------

\title[]{Volatility and Correlation Analysis for } % The short title appears at the bottom of every slide, the full title is only on the title page

\author{Georgi Dinolov} % Your name
\institute[UCSC] % Your institution as it will appear on the bottom of every slide, may be shorthand to save space
{
University of California, Santa Cruz \\ % Your institution for the title page
\medskip
\textit{gdinolov@soe.ucsc.edu} % Your email address
}
\date{\today} % Date, can be changed to a custom date

\begin{document}

\begin{frame}
\titlepage % Print the title page as the first slide
\end{frame}

\begin{frame}
\frametitle{Overview} % Table of contents slide, comment this block out to remove it
\tableofcontents % Throughout your presentation, if you choose to use \section{} and \subsection{} commands, these will automatically be printed on this slide as an overview of your presentation
\end{frame}

%----------------------------------------------------------------------------------------
%	PRESENTATION SLIDES
%----------------------------------------------------------------------------------------

%------------------------------------------------
\section{Introduction}
%------------------------------------------------

\begin{frame}
\frametitle{Two Data Types: High-Frequency and Open, Close High, Low Prices}
\begin{itemize}
  \item high-frequency prices
  \item open, close, high, low as summaries of volatility
\end{itemize}
\end{frame}

%------------------------------------------------
%------------------------------------------------
\section{High-Frequency Prices}
%------------------------------------------------
\begin{frame}
  \frametitle{High-frequency returns and difficulties}
  \begin{itemize}
  \item As sampling frequency approaches transaction-by-transaction
    frequency, \textbf{irregular spacing} between transactions, \textbf{discreteness} in
    transaction prices (such as the bid-ask spread), and very short term \textbf{temporal dependence} become
    dominant features of the data \citep{stoll2000presidential}.

  \item These confounding effects in estimating volatility are
    generally termed \textbf{microstructure noise}.

  \item Generally, microstructure noise effects become apparent with
    sampling periods below $5$ minutes.
  \end{itemize}
\end{frame}
% ------------------------------------------------
\begin{frame}
  \frametitle{Past model-based work and incoherence}
  \begin{itemize}
  \item Attempts have been made to fit the GARCH model to
    high-frequency data \citep{bollerslev1986,andersen1997intraday}.

  \item As formulated, such types of models are are not
    robust with respect to the specification of the sampling interval
    \citep{drost1993aggregation, andersen1997intraday,
      zumbach2000pitfalls}.

  \item Inference over shrinking sampling intervals leads to
    \textbf{incoherence}.
  \end{itemize}
\end{frame}
% ------------------------------------------------
\begin{frame}
  \frametitle{Realized volatility estimators}
  \begin{itemize}
  \item The \textbf{realized variance estimator} is a popular summary
    statistic that converges to the integrated variance of the process
    as sampling frequency increases.
  \item Corrections for the presence of microstructure noise have been
    successfully proposed in the form of:
    \begin{enumerate}
    \item sampling sparser grids and averaging,
    \item combining estimators based on subsampled data at different frequencies,
    \item kernel-based estimation.
    \end{enumerate}
  \item Disadvantages: averages out information; no inherent
    model-based projection forward in time.
  \end{itemize}
\end{frame}
% ------------------------------------------------
\begin{frame}
  \frametitle{A filtering-based approach} We employ a
  \textbf{filtering-based} approach which models microstructure noise
  as directly added to the true asset price and discretely observed:
  \begin{align*}
    P_j &= S_j + \nu_j,& \nu_j &\sim U[-D_p/2, D_p/2],
  \end{align*}
  where $D_p$ is the size of the bid-ask spread.
  A first-order Taylor expansion of the observed log price produces
  \begin{align*} log(P_j) &\approx log(S_j) + \zeta_j,& \zeta_j &= \frac{1}{S_j}\nu_j, \\
    log(P_j) &\approx log(S_j) + \zeta_j,& \zeta_j &\sim N(0, \frac{1}{Q^2}\nu^2_j). \\
  \end{align*}

\end{frame}
% ------------------------------------------------
\begin{frame}
  \frametitle{Full \textit{continuous-time} formulation}

  \begin{align}
  d\log(\hat{S}_t) &= \hat{\mu}\, dt + \sqrt{\hat{\sigma}_{t,1} \hat{\sigma}_{t,2}}\, \sqrt{dt} \hat{\epsilon}_{t,1} + dJ_t  ,   \label{eq:price_evo_2}\\
  d\log( \hat{ \sigma }_{t,1}) &= -\hat{\theta}_1 ( \log(\hat{\sigma}_{t,1} ) - \hat{\alpha} )\, dt + \hat{\tau}_1\, \sqrt{dt} \hat{\epsilon}_{t,1}  , \label{eq:vol_evo_slow} \\
  d\log( \hat{ \sigma }_{t,2}) &= -\hat{\theta}_2 ( \log(\hat{\sigma}_{t,2} ) - \hat{\alpha} )\, dt + \hat{\tau}_2\, \sqrt{dt} \hat{\epsilon}_{t,2}  . \label{eq:vol_evo_fast}
  \end{align}
  
\end{frame}
% ------------------------------------------------
\begin{frame}
  \frametitle{Full \textit{discrete-time} formulation}
  \begin{align}
    Y_j &= \log(S_j) + \zeta_j  ,   \label{eq:mod1}   \\
    \log(S_{j}) &= \log(S_{j-1}) + \mu(\Delta) + \sqrt{\sigma_{j,1}\sigma_{j,2}} \, \epsilon_{j} + J_j(\Delta)   ,  \label{eq:mod2}  \\
    \log(\sigma_{j+1,1}) &= \alpha(\Delta) + \theta_1(\Delta) \left\{ \log(\sigma_{j,1}) - \alpha(\Delta) \right\} + \tau_1(\Delta) \, \epsilon_{j,1}    ,  \label{eq:mod3}  \\
    \log(\sigma_{j+1,2}) &= \alpha(\Delta) + \theta_2(\Delta) \left\{ \log(\sigma_{j,2}) - \alpha(\Delta) \right\} + \tau_2(\Delta) \, \epsilon_{j,2}    , \label{eq:mod4}
  \end{align}

  \begin{align}
  \sigma_{j+1,i} &= \hat{\sigma}_{(j+1)\Delta,i}\sqrt{\Delta}, & S_j &= \hat{S}_{j\Delta}, & J_j(\Delta) &= J((j+1)\Delta) - J(j\Delta)
\end{align}
\begin{align}
  \label{eq:mu_sigma_tau}
  \begin{split}
    \alpha(\Delta) = \halpha + \frac{1}{2}\log(\Delta),\quad  & 
    \mu(\Delta) = \hat{\mu} \Delta,      \\
   \tau_i(\Delta) = \hat{\tau}_i \sqrt{ \frac{1 - \exp \left\{
          -2\hat{\theta}_i \Delta \right\}}{2\hat{\theta}_i } },\quad & \theta_i(\Delta) =
    \exp\left\{
      -\hat{\theta}_i \Delta \right\} ,
    \end{split}
\end{align}
  % ------------------------------------------------
\end{frame}
\begin{frame}
  \frametitle{\textit{Continuous-time} priors}
  \begin{enumerate}
  \item Define first two moments of each continuous-time parameter,
  \item Use the Delta Method to calculate the first two moments of each discrete-time parameter,
  \end{enumerate}
  Example:
  \begin{align}
    \tau^2(\Delta) &\sim \mbox{Inv-Gamma}\left(  a_{\tau^2}(\Delta), b_{\tau^2}(\Delta) \right) \\
    \mbox{E}(\tau^2(\Delta)) &\approx f(a_{\hat{\tau}^2}, b^2_{\hat{\tau}^2}, \Delta) \\
    \mbox{Var}(\tau^2(\Delta)) &\approx g(a_{\hat{\tau}^2}, b^2_{\hat{\tau}^2}, \Delta)
  \end{align}
\end{frame}
% ------------------------------------------------
\begin{frame}
  \frametitle{Computation}
  Linearize \eqref{eq:mod2} by considering 
\begin{multline*}
  \log(S_{j}) = \log(S_{j-1}) + \mu(\Delta) + \sqrt{\sigma_{j,1}\sigma_{j,2}} \, \epsilon_{j} + J_j(\Delta)  \\
  \leftrightarrow \quad \underbrace{ \log\left[ \left| \log(S_{j}/S_{j-1}) - \mu(\Delta) - J_j(\Delta) \right| \right] }_{y_j^*} = \frac{1}{2}\underbrace{  \log(\sigma_{j,1}) }_{h_{j,1}} + \frac{1}{2}\underbrace{  \log(\sigma_{j,2}) }_{h_{j,2}} \\
   + \underbrace{ \log(  \epsilon_{j,1}^2  )/2 }_{\epsilon_{j}^{*}}.
 \end{multline*}

 We approximate $\epsilon^*_{j}$ as a mixture of Normals
\[
	\epsilon^*_{j} = \log( \epsilon_{j}^2 )/2 \sim \sum_{l=1}^{10} p_l N \left( \frac{m_l}{2}, \frac{v_l^2}{4} \right).
\]
\end{frame}
% ------------------------------------------------
\begin{frame}
  \frametitle{Blocked Gibbs sampler with MH steps}
  \begin{enumerate}
  \item sample observational parameters
  \item sample latent prices
  \item sample volatility parameters
  \item sample latent mixture indicators
  \item sample valatility paths
  \item sample jump parameters
  \item sample jumps
  \end{enumerate}
\end{frame}
% ------------------------------------------------
\begin{frame}
  \frametitle{Coverage results for simulated data sets}
  \begin{table}[h]
\begin{center}
  \begin{tabular}{|l|ccccc|}
    \hline
    & \multicolumn{4}{c|}{Sampling period} \\
    &   	60 sec 	&   30 sec   &   15 sec & 5 sec  \\ \hline \hline
    Inference with $\xi^2 = 0$   &  72  &   28  &	 3 & 0 \\
    Inference with $\xi^2 = 2.5 \cdot 10^{-7}$ &  79 & 57 & 23 & 0 \\
    Inference with $\xi^2$ estimated & 91 & 92 & 96 & 97  \\ \hline
    Inference with kernel-based estimator &  51 & 48 & 59  & 76 \\
    \hline
\end{tabular}
\caption{Coverage table comparing the percentage of covered integrated variance levels by 95\% confidence/credible intervals of our estimator and a kernel-based estimator.}\label{ta:coverage}
\end{center}
\end{table}
\end{frame}
% ------------------------------------------------
\begin{frame}
  \frametitle{Parameter results for real data example}
\end{frame}

%------------------------------------------------
%------------------------------------------------
\section{Open, Close, High, Low Prices}
%------------------------------------------------
\begin{frame}
  \frametitle{Open, Close, High, Low Prices}
  Previous work, absence of bivariate solution
\end{frame}
% ------------------------------------------------
\begin{frame}
  \frametitle{Full model formulation: Diffusion Representation}
  We consider a two-dimensional correlated Brownian motion:
\begin{align}
  X(t) &= x_0 + \mu_x t + \sigma_x W_x(t)  \label{eq:X} \\
  Y(t) &= y_0 + \mu_y t + \sigma_y W_y(t)  \label{eq:Y}
\end{align}
where $W_i$ are standard Brownian motions with
$\mbox{Cov}(W_1(t), W_2(t)) = \rho t$. Our interest is in finding the
6-dimensional joint probability density function for the pair $(X(t), Y(t))$
and the random variables $M_X(t)=\max_{0\leq s\leq t}X(s),$
$m_X(t)=\min_{0\leq s\leq t}X(s),$ $M_Y(t)=\max_{0\leq s\leq t}Y(s),$
$m_Y(t)=\min_{0\leq s\leq t}Y(s)$.
\end{frame}
% ------------------------------------------------
\begin{frame}
  \frametitle{Full model formulation: Fokker-Planck Representation}
\begin{multline}
  \displaystyle \frac{\partial}{\partial t'} q(x,y,t') = -\mu_x \frac{\partial}{\partial x}q(x,y,t')
  - \mu_y \frac{\partial}{\partial y}q(x,y,t') + \\
  \frac{1}{2}\sigma_x^2 \frac{\partial^2}{\partial x^2}q(x,y,t') + \rho\sigma_x\sigma_y \frac{\partial^2}{\partial x \partial y}q(x,y,t')
  + \frac{1}{2}\sigma_y^2 \frac{\partial^2}{\partial y^2}q(x,y,t'). \label{eq:1}
\end{multline}
Given that we consider the subset of $(X(t), Y(t))$ that satisfies the constraints imposed by the boundary data, we impose the absorbing conditions on the Fokker-Planck equation
\begin{align}
  q(a_x, y,t') &= q(b_x,y,t') = q(x,a_y,t') = q(x,b_y,t') = 0, & 0 &< t' \leq t. \label{eq:2}
\end{align}

\end{frame}
% ------------------------------------------------
\begin{frame}
  \frametitle{Density definition, fourth-order derivative with respect to boundaries}

\begin{align}
  \frac{\partial^4}{\partial a_x \partial b_x \partial a_y \partial b_y} q(x,y,t) = f(x,y,a_x,b_x,a_y,b_y).
  \label{eq:pdf}
\end{align}
  
  Point out the analytic and numeric difficulties with the full problem.
\end{frame}
% ------------------------------------------------
\begin{frame}
  \frametitle{Approach 1: Finite Difference}
  \begin{enumerate}
  \item fast but ever-present truncation error
  \end{enumerate}
\end{frame}
% ------------------------------------------------
\begin{frame}
  \frametitle{Approach 2: Trigonometric expansion}
  \begin{enumerate}
  \item made impractical by the correlation term
  \end{enumerate}
\end{frame}
% ------------------------------------------------
\begin{frame}
  \frametitle{Introduce the Galerkin approach: weak solution to the PDE}
\end{frame}
% ------------------------------------------------
\begin{frame}
  \frametitle{Basis element choice}
\end{frame}
% ------------------------------------------------
\begin{frame}
  \frametitle{Upper bound on error based on initial condition and
    first version of small-time solution}
\end{frame}
% ------------------------------------------------
\begin{frame}
  \frametitle{Simulation Results}
\end{frame}
% ------------------------------------------------
\begin{frame}
  \frametitle{Need for Higher Resolution}
\end{frame}


%------------------------------------------------
%------------------------------------------------
\section{Open, Close, High, Low Prices: Small Time Solution}
%------------------------------------------------
\begin{frame}
  \frametitle{Counterexample: method of images and analytic differentiability}
  Lack of uniqueness
\end{frame}
% ------------------------------------------------
\begin{frame}
  \frametitle{Symmetry condition for solution}
\end{frame}
% ------------------------------------------------
\begin{frame}
  \frametitle{Solution}
\end{frame}
% ------------------------------------------------
\begin{frame}
  \frametitle{Analytic Derivative and higher order terms}
\end{frame}
% ------------------------------------------------
\begin{frame}
  \frametitle{Illustration of proof for existence}
\end{frame}
% ------------------------------------------------
\begin{frame}
  \frametitle{Basis Expansion and Small-Time leading order order solution}
\end{frame}
% ------------------------------------------------
\begin{frame}
  \frametitle{Matching Solution}
\end{frame}
% ------------------------------------------------
\begin{frame}
  \frametitle{Repeated Simulation Results}
\end{frame}

\end{document} 


