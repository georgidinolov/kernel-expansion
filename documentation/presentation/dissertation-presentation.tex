
%%%%%%%%%%%%%%%%%%%%%%%%%%%%%%%%%%%%%%%%% 
% Beamer Presentation
% LaTeX Template
% Version 1.0 (10/11/12)
%
% This template has been downloaded from:
% http://www.LaTeXTemplates.com
%
% License:
% CC BY-NC-SA 3.0 (http://creativecommons.org/licenses/by-nc-sa/3.0/)
%
%%%%%%%%%%%%%%%%%%%%%%%%%%%%%%%%%%%%%%%%%

%----------------------------------------------------------------------------------------
%	PACKAGES AND THEMES
%----------------------------------------------------------------------------------------

\documentclass{beamer}

\mode<presentation> {

% The Beamer class comes with a number of default slide themes
% which change the colors and layouts of slides. Below this is a list
% of all the themes, uncomment each in turn to see what they look like.

%\usetheme{default}
%\usetheme{AnnArbor}
%\usetheme{Antibes}
%\usetheme{Bergen}
%\usetheme{Berkeley}
%\usetheme{Berlin}
%\usetheme{Boadilla}
\usetheme{CambridgeUS}
% \usetheme{Copenhagen}
%\usetheme{Darmstadt}
%\usetheme{Dresden}
%\usetheme{Frankfurt}
%\usetheme{Goettingen}
%\usetheme{Hannover}
%\usetheme{Ilmenau}
%\usetheme{JuanLesPins}
%\usetheme{Luebeck}
%\usetheme{Madrid}
%\usetheme{Malmoe}
%\usetheme{Marburg}
%\usetheme{Montpellier}
%\usetheme{PaloAlto}
%\usetheme{Pittsburgh}
%\usetheme{Rochester}
%\usetheme{Singapore}
%\usetheme{Szeged}
%\usetheme{Warsaw}

% As well as themes, the Beamer class has a number of color themes
% for any slide theme. Uncomment each of these in turn to see how it
% changes the colors of your current slide theme.

%\usecolortheme{albatross}
%\usecolortheme{beaver}
%\usecolortheme{beetle}
%\usecolortheme{crane}
%\usecolortheme{dolphin}
%\usecolortheme{dove}
%\usecolortheme{fly}
%\usecolortheme{lily}
%\usecolortheme{orchid}
%\usecolortheme{rose}
\usecolortheme{seagull}
%\usecolortheme{seahorse}
%\usecolortheme{whale}
%\usecolortheme{wolverine}

%\setbeamertemplate{footline} % To remove the footer line in all slides uncomment this line
%\setbeamertemplate{footline}[page number] % To replace the footer line in all slides with a simple slide count uncomment this line

%\setbeamertemplate{navigation symbols}{} % To remove the navigation symbols from the bottom of all slides uncomment this line
}

\usepackage{graphicx} % Allows including images
\usepackage{booktabs} % Allows the use of \toprule, \midrule and \bottomrule in tables

\usepackage{bbm}
\usepackage{fixltx2e}
\usepackage{url}
\usepackage{amsmath, amsthm, amssymb}
\usepackage{mathtools}
\usepackage{multirow}
\usepackage{colortbl}
\usepackage{xspace}
\usepackage{verbatim}
\usepackage{graphicx}
\usepackage{psfrag}
\usepackage{natbib}
\usepackage{pbox}
\usepackage{epstopdf}
\usepackage{textcomp} % For copyright symbol
\DeclareGraphicsExtensions{.pdf,.eps,.png,.jpg}
\usepackage[font=singlespacing]{subcaption}
\captionsetup[table]{labelfont=bf,font=singlespacing}
\captionsetup[figure]{labelfont=bf,font=singlespacing}
\captionsetup[subfigure]{labelfont=bf,font=singlespacing}
\usepackage[titletoc,title]{appendix}
\usepackage{tabularx}
\usepackage{rotating}
\usepackage{hyperref} % Turns all internal references into links to pages within the document

\usepackage{lmodern} % Use the Latin Modern font
\usepackage[T1]{fontenc} % Use a modern font encoding

\newcommand{\Exp}[1]{\mbox{Exp}\left(#1\right)}
\newcommand{\Pois}[1]{\mbox{Pois}\left(#1\right)}

\newcommand{\hb}{\hat{b}}
\newcommand{\ha}{\hat{a}}
\newcommand{\htheta}{\hat{\theta}}
\newcommand{\halpha}{\hat{\alpha}}
\newcommand{\hmu}{\hat{\mu}}
\newcommand{\hsigma}{\hat{\sigma}}
\newcommand{\hphi}{\hat{\phi}}
\newcommand{\htau}{\hat{\tau}}
\newcommand{\heta}{\hat{\eta}}
\newcommand{\E}[1]{\mbox{E}\left[#1\right]}
\newcommand{\Var}[1]{\mbox{Var}\left[#1\right]}
\newcommand{\Indicator}[1]{\mathbbm{1}_{ \left( #1 \right) } }
\newcommand{\dNormal}[3]{ N\left( #1 \left| #2, #3 \right. \right) }
\newcommand{\Beta}[2]{\mbox{Beta}\left( #1, #2 \right)}
\newcommand{\alphaphi}{\alpha_{\hphi}}
\newcommand{\betaphi}{\beta_{\hphi}}
\newcommand{\expo}[1]{ \exp\left\{ #1 \right\}}
\newcommand{\tauSquareDelta}{\htau^2
  \left(\frac{1-\expo{-2\htheta\Delta}}{2\htheta} \right)}
\newcommand{\mumu}{\mu_{\hmu}}
\newcommand{\sigmamu}{\sigma^2_{\hmu}}
\newcommand{\sigmamuexpr}{\log\left( \frac{\VarX}{\EX^2} + 1 \right)}
\newcommand{\mumuexpr}{\log(\EX) -  \log\left( \frac{\VarX}{\EX^2} + 1 \right) /2 }

\newcommand{\EX}{\mbox{E}\left[ X \right] }
\newcommand{\VarX}{\mbox{Var}\left[ X \right] }
\newcommand{\mueta}{\mu_{\heta} }
\newcommand{\sigmaeta}{\sigma^2_{\heta}}
\newcommand{\sigmaetaexpr}{ \log\left( \frac{\VarX}{\EX^2} + 1 \right) }
\newcommand{\muetaexpr}{ \log(\EX) -  \sigmaetaexpr /2 }

\newcommand{\mualpha}{\mu_{\halpha} }
\newcommand{\sigmaalpha}{\sigma^2_{\halpha}}
\newcommand{\sigmaalphaexpr}{ \log\left( \frac{\VarX}{\EX^2} + 1 \right) }
\newcommand{\mualphaexpr}{ \log(\EX) -  \sigmaalphaexpr /2 }

\newcommand{\mutauexpr}{ \frac{2}{T} \EX }
\newcommand{\sigmatauexpr}{ \frac{4}{T^2} \Var{X}}

\newcommand{\alphatau}{\alpha_{\htau^2}}
\newcommand{\betatau}{\beta_{\htau^2}}

\newcommand{\Gam}[2]{\mbox{Gamma}\left( #1, #2 \right) }
\newcommand{\InvGam}[2]{\mbox{Inv-Gamma}\left( #1, #2 \right) }

%----------------------------------------------------------------------------------------
%	TITLE PAGE
%----------------------------------------------------------------------------------------

\title[]{Volatility and Correlation Analysis of Financial Market Data} % The short title appears at the bottom of every slide, the full title is only on the title page

\author{Georgi Dinolov} % Your name
\institute[UCSC] % Your institution as it will appear on the bottom of every slide, may be shorthand to save space
{
University of California, Santa Cruz \\ % Your institution for the title page
\medskip
\textit{gdinolov@soe.ucsc.edu} % Your email address
}
\date{\today} % Date, can be changed to a custom date

\begin{document}

\begin{frame}
\titlepage % Print the title page as the first slide
\end{frame}

\begin{frame}
\frametitle{Overview} % Table of contents slide, comment this block out to remove it
\begin{enumerate}
\item Introduction
\item High-Frequency Prices and Inference
\item Open, Close, High, Low Prices
  \begin{itemize}
  \item Galerkin soluiton
  \item Small-time Analytic Solution and Gap-Fill Approximation
  \end{itemize}
\end{enumerate}
\end{frame}

%----------------------------------------------------------------------------------------
%	PRESENTATION SLIDES
%----------------------------------------------------------------------------------------

%------------------------------------------------
\section{Introduction}
%------------------------------------------------

\begin{frame}
\frametitle{Two data types: High-Frequency}

\begin{itemize}
  \item Longitudonal view
\begin{table}
\begin{tabular}{|c|c|c|}
  Timestamp & Volume& Price \\
  \hline
  2011-08-01 03:02:52.112434 & 1000 & 179.6400 \\
  \hline
  2011-08-01 03:02:53.752456 & 1000 & 179.6200 \\
  \hline
  2011-08-01 03:02:53.900010 & 1000 & 179.6300 \\
  \hline
  2011-08-01 03:02:54.103493 & 1000 & 179.6050 \\
  \hline
  2011-08-01 03:02:54.343493 & 1000 & 179.6700 \\
\end{tabular}
\end{table}

\item Cross-sectional view: the bid-ask spread
  \begin{table}
  \begin{tabular}{c|c}
    Volume & Price \\
    \hline \hline
    101 & 179.6400 \\
    203 & 179.6200 \\
    \hline
    305 & 179.6100 \\
    500 & 179.6000 \\
  \end{tabular}
  \end{table}
\end{itemize}
\end{frame}


\begin{frame}
\frametitle{Two data types: Open, Close, High, Low}

\begin{figure}
  \includegraphics[width=1\textwidth]{/home/gdinolov/PDE-solvers/src/kernel-expansion/documentation/presentation/OCHL.png}
\end{figure}

\end{frame}

%------------------------------------------------
%------------------------------------------------
\section{High-Frequency Prices}
%------------------------------------------------
\begin{frame}
  \frametitle{High-frequency returns and difficulties}
  \begin{itemize}
  \item As sampling frequency approaches transaction-by-transaction
    frequency, \textbf{irregular spacing} between transactions and 
    \textbf{discreteness} in transaction prices (such as the bid-ask
    spread) become dominant features of the data
    \citep{stoll2000presidential}.

  \item These confounding effects in estimating volatility are
    generally termed \textbf{microstructure noise}, which is

    \begin{itemize}
    \item independent of sampling interval
    \item dominant over short time ($<5$ minutes)
    \item not cumulative (prices at later times are not affected)
    \end{itemize}

  \end{itemize}
\end{frame}
% ------------------------------------------------
\begin{frame}
  \frametitle{Past work and model-bashed incoherence}
  \begin{itemize}
  \item Attempts have been made to fit discrete time state-space
    models to high-frequency data and estimate price volatility
    \citep{bollerslev1986,andersen1997intraday}
    \begin{align*}
      y_{t+\Delta t} &= F_{t} \theta_t + C_t W_{t,1} \\
      \theta_{t+\Delta t} &= G_{t} \theta_t + V_t W_{t,1}. \\
    \end{align*}

  \item As formulated, such types of models are are not
    robust with respect to the specification of the sampling interval
    \citep{drost1993aggregation, andersen1997intraday,
      zumbach2000pitfalls}.

  \item Inference over shrinking sampling interval $\Delta t$ leads to
    \textbf{incoherence}.

  \item Most recent work has forcused on the \textbf{realized variance} \cite{comte1998long} estimator:
    \[
      RV_T := \sum_{i=1}^{N/\Delta t} r_{t_i}^2,
    \]
    where $r_{t_i}^2$ is the squared \textit{return} observed over $[t_{i-1}, t_i]$.
  \end{itemize}
\end{frame}
% ------------------------------------------------
\begin{frame}
  \frametitle{Realized variance estimator}
  In the absence of microstructure noise, 
  \[
    RV_t = \sum_{i=1}^{N/\Delta t} r_{t_i}^2 \,\,\, \to \,\,\, \displaystyle \int_{0}^T \sigma^2\left(X_t, t\right)\, dt
  \]
  
  \begin{itemize}
  \item Corrections for the presence of microstructure noise have been
    proposed in the form of:
    \begin{itemize}
    \item sampling sparser grids and averaging \cite{zhang2005tale},
    \item combining estimators based on subsampled data at different frequencies \cite{ait2011ultra},
    \item kernel-based estimation \cite{hansen2006realized, barndorff2008designing}.
    \end{itemize}
  \item Disadvantages: averages out information; no inherent
    model-based projection forward in time.
  \end{itemize}
\end{frame}
% ------------------------------------------------
\begin{frame}
  \frametitle{A filtering-based approach}
  \framesubtitle{modeling microstructure noise}

  Given the \textbf{true} price $S_j$ at sample index $j$, we model
  the \textbf{measured} price $P_j$ as contaminated by noise due to
  the bid-ask spread $D$:
  \begin{align*}
    P_j &= S_j + \nu_j,& \nu_j &\sim U[-D/2, D/2],
  \end{align*}

  A first-order Taylor expansion of the \textbf{observed log price} $\log(P_j)$ produces
  \begin{align*}
    log(P_j) &\approx log(S_j) + \frac{1}{S_j}\nu_j.
  \end{align*}

  We further model the noise term with a Gaussian distribution
  \begin{align*}
    \zeta_j := \frac{1}{S_j}\nu_j \sim N\left(0, \frac{D}{4Q^2} \right)
  \end{align*}
  where $Q$ is an order-magnitude approximation of the true price
  $S_j$.
\end{frame}
% ------------------------------------------------
\begin{frame}
  \frametitle{Stochastic evolution of price and volatility}
  the idealized joint log-price $\log(\hat{S}_t)$ and log-volatility
  $\log(\hat{\sigma}_{t,1}), \log(\hat{\sigma}_{t,2})$ diffusion
  process as the system of SDEs:
  \begin{align*}
  d\log(\hat{S}_t) &= \hat{\mu}\, dt + \sqrt{\hat{\sigma}_{t,1} \hat{\sigma}_{t,2}}\, \sqrt{dt} \hat{\epsilon}_{t} + dJ_t  ,   \\
  d\log( \hat{ \sigma }_{t,1}) &= -\hat{\theta}_1 ( \log(\hat{\sigma}_{t,1} ) - \hat{\alpha} )\, dt + \hat{\tau}_1\, \sqrt{dt} \hat{\epsilon}_{t,1}  , \\
  d\log( \hat{ \sigma }_{t,2}) &= -\hat{\theta}_2 ( \log(\hat{\sigma}_{t,2} ) - \hat{\alpha} )\, dt + \hat{\tau}_2\, \sqrt{dt} \hat{\epsilon}_{t,2}  . 
  \end{align*}

  \begin{itemize}
  \item $\log(\hat{\sigma}_{t,1})$ evolves stochastically with long time scale $1/\hat{\theta}_1,$
  \item $\log(\hat{\sigma}_{t,2})$ evolves stochastically with short time scale $1/\hat{\theta}_2,$
  \item $dJ_t$ is a compound Poisson process modeling jumps,
  \item $\hat{\epsilon}_t, \hat{\epsilon}_{t,1}, \hat{\epsilon}_{t,2}$ are Normally distributed with 
    \begin{align*}
      \E{\hat{\epsilon}_t\hat{\epsilon}_{t,2}} &= \rho, & \E{\hat{\epsilon}_t\hat{\epsilon}_{t,2}} &= 0, & \E{\hat{\epsilon}_{t,1}\hat{\epsilon}_{t,2}} &= 0.
    \end{align*}
  \end{itemize}
    
\end{frame}
% ------------------------------------------------
\begin{frame}
  \frametitle{Discrete interpretation of the formulation}
  Defining the log of the observed price $P_j$ as $Y_j := \log(P_j)$
  \begin{align*}
    Y_j &= \log(S_j) + \zeta_j  ,\\
    \log(S_{j}) &= \log(S_{j-1}) + \mu(\Delta) + \sqrt{\sigma_{j,1}\sigma_{j,2}} \, \epsilon_{j} + J_j(\Delta)   ,  \\
    \log(\sigma_{j+1,1}) &= \alpha(\Delta) + \theta_1(\Delta) \left\{ \log(\sigma_{j,1}) - \alpha(\Delta) \right\} + \tau_1(\Delta) \, \epsilon_{j,1}    ,  \\
    \log(\sigma_{j+1,2}) &= \alpha(\Delta) + \theta_2(\Delta) \left\{ \log(\sigma_{j,2}) - \alpha(\Delta) \right\} + \tau_2(\Delta) \, \epsilon_{j,2}.
  \end{align*}

  \begin{align*}
  \sigma_{j+1,i} &= \hat{\sigma}_{(j+1)\Delta,i}\sqrt{\Delta}, & S_j &= \hat{S}_{j\Delta}, & J_j(\Delta) &= J((j+1)\Delta) - J(j\Delta)
  \end{align*}
  
\begin{align*}
  \begin{split}
    \alpha(\Delta) = \halpha + \frac{1}{2}\log(\Delta),\quad  & 
    \mu(\Delta) = \hat{\mu} \Delta,      \\
    \tau_i(\Delta) = \hat{\tau}_i \sqrt{ \frac{1 - \exp \left\{
          -2\hat{\theta}_i \Delta \right\}}{2\hat{\theta}_i } },\quad & \theta_i(\Delta) =
    \exp\left\{
      -\hat{\theta}_i \Delta \right\}.
    \end{split}
\end{align*}
\end{frame}
% ------------------------------------------------
\begin{frame}
  \frametitle{Prior formulation} To ensure that the model estimation
  is coherent across choices of sampling periods $\Delta$, we
  
  \begin{enumerate}
  \item define the first two moments of each continuous-time parameter,
  % \item define the prior for each continuous-time parameter,
  \item use the Delta Method to calculate the first two moments of each discrete-time parameter,
  \item define the prior for each discrete-time parameter.
  \end{enumerate}
  % Example:
  % \begin{align}
  %   \tau^2(\Delta) &\sim \mbox{Inv-Gamma}\left(  a_{\tau^2}(\Delta), b_{\tau^2}(\Delta) \right) \\
  %   \mbox{E}(\tau^2(\Delta)) &\approx f(a_{\hat{\tau}^2}, b^2_{\hat{\tau}^2}, \Delta) \\
  %   \mbox{Var}(\tau^2(\Delta)) &\approx g(a_{\hat{\tau}^2}, b^2_{\hat{\tau}^2}, \Delta)
  % \end{align}
\end{frame}
% ------------------------------------------------
\begin{frame}
  \frametitle{Computation}
  \begin{itemize}
  \item We separate the standard innovation term from volatility
    factors $\sigma_{j,1}$ and $\sigma_{j,2}$ by considering the
    transformation
  \begin{multline*}
    \log(S_{j}) = \log(S_{j-1}) + \mu(\Delta) + \sqrt{\sigma_{j,1}\sigma_{j,2}} \, \epsilon_{j} + J_j(\Delta) \\
  \leftrightarrow \quad \underbrace{ \log\left[ \left| \log(S_{j}/S_{j-1}) - \mu(\Delta) - J_j(\Delta) \right| \right] }_{y_j^*} = \frac{1}{2}\underbrace{  \log(\sigma_{j,1}) }_{h_{j,1}} + \frac{1}{2}\underbrace{  \log(\sigma_{j,2}) }_{h_{j,2}} \\
   + \underbrace{ \log(  \epsilon_{j}^2  )/2 }_{\epsilon_{j}^{*}}.
 \end{multline*}
 \pause
\item We approximate $\epsilon^*_{j}$ as a mixture of Normals
 \[
   \epsilon^*_{j} = \log( \epsilon_{j}^2 )/2 \sim \sum_{l=1}^{10} p_l N \left( \frac{m_l}{2}, \frac{v_l^2}{4} \right).
 \]
\end{itemize}
\end{frame}
% ------------------------------------------------
\begin{frame}
  \frametitle{Computation, continued}
  The joint distribution $(\epsilon^*_j, \epsilon_{j,2})$ conditional on the latent mixture element $\gamma_j$ becomes
  \begin{align*}
    p(\epsilon^*_j, \epsilon_{j,2} | \gamma_j) &= p(\epsilon_{j,2}|\epsilon^*_{j}, \gamma_j)p(\epsilon^*_{j}| \gamma_j) \\
                                               &= p(\epsilon_{j,2}|\underbrace{d_j exp(\epsilon^*_{j})}_{\epsilon_j}, \gamma_j)p(\epsilon^*_{j}| \gamma_j) \\
    &= N\left( \epsilon_{j,2} \left| d_j exp(\epsilon^*_{j}), (1-\rho^2) \right.\right) N\left( \epsilon_j^* \left| \frac{m_{\gamma_j}}{2}, \frac{v^2_{\gamma_j}}{4} \right. \right),
  \end{align*}
  where $d_j$ is the sign of $\epsilon_j$. \pause The joint
  distribution becomes a linear combination of independent Normal
  random variables when we replace the nonlinear $exp(\epsilon^*_{j})$
  term with the first-order approximation
  \[
    exp(\epsilon^*_{j}) | \gamma_j \approx \exp(m_{\gamma_j}/2)(a_{\gamma_j} + b_{\gamma_j}(2\epsilon_j^* - m_{\gamma_j}))
  \]
 
\end{frame}
% ------------------------------------------------
\begin{frame}
  \frametitle{Blocked Gibbs sampler with predominantly Metropolis-Hastings and Forward Filter Backward Sampling steps}
  \begin{enumerate}
  \item sample observational parameters (MH)
  \item sample latent prices (FFBS)
  \item sample volatility parameters (MH)
  \item sample latent mixture indicators (Discrete)
  \item sample valatility paths (FFBS)
  \item sample jump parameters (MH)
  \item sample jumps (Conjugate prior)
  \end{enumerate}
\end{frame}
% ------------------------------------------------
\begin{frame}
  \frametitle{Estimating integrated volatility}
  \begin{itemize}
    \item We consider 300 simulated data sets over 1 trading day
  
    \item Coverage : percent of $95\%$ probability/confidence intervals
      covering the true data-generating integrated volatility
      $\int \hat{\sigma}^2_t dt$:
  \end{itemize}
  \begin{table}[h]
    \begin{center}
      \begin{tabular}{l|ccccc}
        \hline
        & \multicolumn{4}{c}{Sampling period} \\
        &   	60 sec 	&   30 sec   &   15 sec & 5 sec  \\ \hline \hline
        Inference with $\xi^2 = 0$   &  72  &   28  &	 3 & 0 \\
        Inference with $\xi^2$ fixed at prior mean &  79 & 57 & 23 & 0 \\
        Inference with $\xi^2$ estimated & \textcolor{red}{91} & \textcolor{red}{92} & \textcolor{red}{96} & \textcolor{red}{97}  \\ \hline
        Inference with kernel-based estimator &  51 & 48 & 59  & 76 \\
        \hline
      \end{tabular}
    \end{center}
  \end{table}
\end{frame}
% ------------------------------------------------
\begin{frame}
  \frametitle{Results for real data example}

  \centering
  \begin{tabular}{m{0.25cm}ccc}
    & Inference with & Inference with & Inference with \\
    & $\xi^2 = 0$ & $\xi^2\,\, \mbox{fixed at prior mean} $ & $\xi^2 \mbox{ estimated }$ \\
    % 
    \begin{sideways} $\Delta = 5$ min \end{sideways}
    & \begin{minipage}{0.25\textwidth}
      \centering
      \includegraphics[width=1\linewidth]{/home/gdinolov/PDE-solvers/src/SV-with-leverage/paper-revisions-7-11-2016/results-real-data-plots-VOL-PATHS-microstructure-VOL-PATHS-XI-0-dt-3e05-SDs-0.png}
    \end{minipage}
                     & \begin{minipage}{0.25\textwidth}
                       \centering
                       \includegraphics[width=1\linewidth]{{/home/gdinolov/PDE-solvers/src/SV-with-leverage/paper-revisions-7-11-2016/results-real-data-plots-VOL-PATHS-microstructure-VOL-PATHS-XI-2.5e-07-dt-3e05-SDs-0}.png}
                     \end{minipage}
                                      & \begin{minipage}{0.25\textwidth}
                                        \centering
                                        \includegraphics[width=1\linewidth]{/home/gdinolov/PDE-solvers/src/SV-with-leverage/paper-revisions-7-11-2016/results-real-data-plots-VOL-PATHS-microstructure-VOL-PATHS-XI-Inf-dt-3e05-SDs-0.png}
                                      \end{minipage}  \\
                                      % 
    % \begin{sideways} $\Delta = 15$ sec \end{sideways}
    % & \begin{minipage}{0.25\textwidth}
    %   \centering
    %   \includegraphics[width=1\linewidth]{/home/gdinolov/PDE-solvers/src/SV-with-leverage/paper-revisions-7-11-2016/results-real-data-plots-VOL-PATHS-microstructure-VOL-PATHS-XI-0-dt-15000-SDs-0.png}
    % \end{minipage}
    %                  & \begin{minipage}{0.25\textwidth}
    %                    \centering
    %                    \includegraphics[width=1\linewidth]{{/home/gdinolov/PDE-solvers/src/SV-with-leverage/paper-revisions-7-11-2016/results-real-data-plots-VOL-PATHS-microstructure-VOL-PATHS-XI-2.5e-07-dt-15000-SDs-0}.png}
    %                  \end{minipage}
    %                                   & \begin{minipage}{0.25\textwidth}
    %                                     \centering
    %                                     \includegraphics[width=1\linewidth]{/home/gdinolov/PDE-solvers/src/SV-with-leverage/paper-revisions-7-11-2016/results-real-data-plots-VOL-PATHS-microstructure-VOL-PATHS-XI-Inf-dt-15000-SDs-0.png}
    %                                   \end{minipage}  \\
                                      % 
    \begin{sideways} $\Delta = 5$ sec \end{sideways}
    & \begin{minipage}{0.25\textwidth}
      \centering
      \includegraphics[width=1\linewidth]{/home/gdinolov/PDE-solvers/src/SV-with-leverage/paper-revisions-7-11-2016/results-real-data-plots-VOL-PATHS-microstructure-VOL-PATHS-XI-0-dt-5000-SDs-0.png}
    \end{minipage}
                     & \begin{minipage}{0.25\textwidth}
                       \centering
                       \includegraphics[width=1\linewidth]{{/home/gdinolov/PDE-solvers/src/SV-with-leverage/paper-revisions-7-11-2016/results-real-data-plots-VOL-PATHS-microstructure-VOL-PATHS-XI-2.5e-07-dt-5000-SDs-0}.png}
                     \end{minipage}
                                      & \begin{minipage}{0.25\textwidth}
                                        \centering
                                        \includegraphics[width=1\linewidth]{/home/gdinolov/PDE-solvers/src/SV-with-leverage/paper-revisions-7-11-2016/results-real-data-plots-VOL-PATHS-microstructure-VOL-PATHS-XI-Inf-dt-5000-SDs-0.png}
                                      \end{minipage}  \\
                                      %% 
                                      % \begin{sideways} $\Delta = 1$ sec \end{sideways}
                                      % & \begin{minipage}{0.25\textwidth}
                                      %   \centering
                                      %   \includegraphics[width=1\linewidth]{results/real-data/plots/VOL-PATHS/microstructure-VOL-PATHS-XI-0-dt-1000-SDs-0.png}
                                      % \end{minipage}
                                      % & \begin{minipage}{0.25\textwidth}
                                      %   \centering
                                      %   \includegraphics[width=1\linewidth]{results/real-data/plots/VOL-PATHS/microstructure-VOL-PATHS-XI-1e-06-dt-1000-SDs-0.png}
                                      % \end{minipage}
                                      % & \begin{minipage}{0.25\textwidth}
                                      %   \centering
                                      %   \includegraphics[width=1\linewidth]{results/real-data/plots/VOL-PATHS/microstructure-VOL-PATHS-XI-Inf-dt-1000-SDs-0.png}
                                      % \end{minipage}
  \end{tabular}
  Log-volatility paths for the AAPL 03/06/2014 data.
\end{frame}
% % ------------------------------------------------
% \begin{frame}
%   \frametitle{Parameter results for real data example}

%   \centering
%   \begin{tabular}{m{0.25cm}ccc}
%     & Inference with & Inference with & Inference with \\
%     & $\xi^2 = 0$ & $\xi^2\,\, \mbox{fixed at prior mean} $ & $\xi^2 \mbox{ estimated }$ \\
%     % 
%     \begin{sideways} $\Delta = 5$ min \end{sideways}
%     & \begin{minipage}{0.25\textwidth}
%       \centering
%       \includegraphics[width=1\linewidth]{/home/gdinolov/PDE-solvers/src/SV-with-leverage/paper-revisions-7-11-2016/results-real-data-plots-VOL-PATHS-microstructure-VOL-PATHS-XI-0-dt-3e05-SDs-0.png}
%     \end{minipage}
%                      & \begin{minipage}{0.25\textwidth}
%                        \centering
%                        \includegraphics[width=1\linewidth]{{/home/gdinolov/PDE-solvers/src/SV-with-leverage/paper-revisions-7-11-2016/results-real-data-plots-VOL-PATHS-microstructure-VOL-PATHS-XI-2.5e-07-dt-3e05-SDs-0}.png}
%                      \end{minipage}
%                                       & \begin{minipage}{0.25\textwidth}
%                                         \centering
%                                         \includegraphics[width=1\linewidth]{/home/gdinolov/PDE-solvers/src/SV-with-leverage/paper-revisions-7-11-2016/results-real-data-plots-VOL-PATHS-microstructure-VOL-PATHS-XI-Inf-dt-3e05-SDs-0.png}
%                                       \end{minipage}  \\
%                                       % 
%     % \begin{sideways} $\Delta = 15$ sec \end{sideways}
%     % & \begin{minipage}{0.25\textwidth}
%     %   \centering
%     %   \includegraphics[width=1\linewidth]{/home/gdinolov/PDE-solvers/src/SV-with-leverage/paper-revisions-7-11-2016/results-real-data-plots-VOL-PATHS-microstructure-VOL-PATHS-XI-0-dt-15000-SDs-0.png}
%     % \end{minipage}
%     %                  & \begin{minipage}{0.25\textwidth}
%     %                    \centering
%     %                    \includegraphics[width=1\linewidth]{{/home/gdinolov/PDE-solvers/src/SV-with-leverage/paper-revisions-7-11-2016/results-real-data-plots-VOL-PATHS-microstructure-VOL-PATHS-XI-2.5e-07-dt-15000-SDs-0}.png}
%     %                  \end{minipage}
%     %                                   & \begin{minipage}{0.25\textwidth}
%     %                                     \centering
%     %                                     \includegraphics[width=1\linewidth]{/home/gdinolov/PDE-solvers/src/SV-with-leverage/paper-revisions-7-11-2016/results-real-data-plots-VOL-PATHS-microstructure-VOL-PATHS-XI-Inf-dt-15000-SDs-0.png}
%     %                                   \end{minipage}  \\
%                                       % 
%     \begin{sideways} $\Delta = 5$ sec \end{sideways}
%     & \begin{minipage}{0.25\textwidth}
%       \centering
%       \includegraphics[width=1\linewidth]{/home/gdinolov/PDE-solvers/src/SV-with-leverage/paper-revisions-7-11-2016/results-real-data-plots-VOL-PATHS-microstructure-VOL-PATHS-XI-0-dt-5000-SDs-0.png}
%     \end{minipage}
%                      & \begin{minipage}{0.25\textwidth}
%                        \centering
%                        \includegraphics[width=1\linewidth]{{/home/gdinolov/PDE-solvers/src/SV-with-leverage/paper-revisions-7-11-2016/results-real-data-plots-VOL-PATHS-microstructure-VOL-PATHS-XI-2.5e-07-dt-5000-SDs-0}.png}
%                      \end{minipage}
%                                       & \begin{minipage}{0.25\textwidth}
%                                         \centering
%                                         \includegraphics[width=1\linewidth]{/home/gdinolov/PDE-solvers/src/SV-with-leverage/paper-revisions-7-11-2016/results-real-data-plots-VOL-PATHS-microstructure-VOL-PATHS-XI-Inf-dt-5000-SDs-0.png}
%                                       \end{minipage}  \\
%                                       %% 
%                                       % \begin{sideways} $\Delta = 1$ sec \end{sideways}
%                                       % & \begin{minipage}{0.25\textwidth}
%                                       %   \centering
%                                       %   \includegraphics[width=1\linewidth]{results/real-data/plots/VOL-PATHS/microstructure-VOL-PATHS-XI-0-dt-1000-SDs-0.png}
%                                       % \end{minipage}
%                                       % & \begin{minipage}{0.25\textwidth}
%                                       %   \centering
%                                       %   \includegraphics[width=1\linewidth]{results/real-data/plots/VOL-PATHS/microstructure-VOL-PATHS-XI-1e-06-dt-1000-SDs-0.png}
%                                       % \end{minipage}
%                                       % & \begin{minipage}{0.25\textwidth}
%                                       %   \centering
%                                       %   \includegraphics[width=1\linewidth]{results/real-data/plots/VOL-PATHS/microstructure-VOL-PATHS-XI-Inf-dt-1000-SDs-0.png}
%                                       % \end{minipage}
%   \end{tabular}
%   Log-volatility paths for the AAPL 03/06/2014 data.
% \end{frame}
% %----------------------------------------------
\section{Bivariate Open, Close, High, Low Prices}
%------------------------------------------------
\begin{frame}
  \frametitle{Open, Close, High, Low Prices}
  \begin{itemize}
  \item Some work has been done in the univariate case
  \item No equivalent results in the bivariate setting
  \item There exists only a single estimator in the literature which
    used bivarate OCHL to estimate asset correlation.
  \end{itemize}
\end{frame}
% ------------------------------------------------
\begin{frame}
  \frametitle{Model formulation}
  We consider a two-dimensional correlated Brownian motion:
\begin{align*}
  X(t) &= x_0 + \mu_x t + \sigma_x W_x(t)  \\
  Y(t) &= y_0 + \mu_y t + \sigma_y W_y(t) 
\end{align*}
where $W_x(t)$ and $W_y(t)$ are correlated standard Brownian motions
with $\mbox{Cov}(W_x(t), W_x(t)) = \rho t$.

\begin{itemize}
\item We seek the 6-dimensional joint probability density
function for the pair $(X(t), Y(t))$ and the random variables
$M_X(t)=\max_{0\leq s\leq t}X(s),$ $m_X(t)=\min_{0\leq s\leq t}X(s),$
$M_Y(t)=\max_{0\leq s\leq t}Y(s),$ $m_Y(t)=\min_{0\leq s\leq t}Y(s)$.
\end{itemize}
\end{frame}
% ------------------------------------------------
\begin{frame}
  \frametitle{Transition density} To calculate the joint density
  \begin{multline*}
    p(X(t) = x, Y(t) = y, \\
    m_X(t) = a_x, M_X(t) = b_X, m_Y(t) = a_Y, M_Y(t) = b_Y),
  \end{multline*}
  we first study cumulative-like distribution
  \begin{multline*}
    p(X(t) = x, Y(t) = y, \\
    m_X(t) \geq a_x, M_X(t) \leq b_X, m_Y(t) \geq
    a_Y, M_Y(t) \leq b_Y),
  \end{multline*}
  which is governed by a Fokker-Planck equation with absorbing boundaries:
\end{frame}
% ------------------------------------------------
\begin{frame}
  \frametitle{Governing Fokker-Planck equation}
  Abbreviating 
  \begin{multline*}
    q(x,y,t) := p(X(t) = x, Y(t) = y, \\
    m_X(t) \geq a_x, M_X(t) \leq b_X, m_Y(t) \geq
    a_Y, M_Y(t) \leq b_Y),
  \end{multline*}
  the FP equation is
  \begin{multline*}
    \displaystyle \frac{\partial}{\partial t'} q(x,y,t') = -\mu_x \frac{\partial}{\partial x}q(x,y,t')
    - \mu_y \frac{\partial}{\partial y}q(x,y,t') + \\
    \frac{1}{2}\sigma_x^2 \frac{\partial^2}{\partial x^2}q(x,y,t') + \rho\sigma_x\sigma_y \frac{\partial^2}{\partial x \partial y}q(x,y,t')
    + \frac{1}{2}\sigma_y^2 \frac{\partial^2}{\partial y^2}q(x,y,t'),
  \end{multline*}
  \begin{align*}
    q(a_x, y,t') &= q(b_x,y,t') = q(x,a_y,t') = q(x,b_y,t') = 0, & 0 &< t' \leq t, \\
    q(x,y,0) &= \delta(x_0 - x)\delta(y_0 - y)
  \end{align*}

\end{frame}
% ------------------------------------------------
\begin{frame}
  \frametitle{Normalized problem}
  We can introduce a series of transformations to solve the equivalent initial-boundary problem
  \begin{align*}
    \frac{\partial}{\partial \tilde{t}} q(\tilde{x},\tilde{y},\tilde{t}) &= \tilde{\mathcal{L}} q(\tilde{x},\tilde{y},\tilde{y}), & (\tilde{x}, \tilde{y}) \in \tilde{\Omega} \\
    \tilde{\mathcal{L}} &= \frac{1}{2} \frac{\partial^2}{\partial \tilde{x}^2} + \rho \sigma_{\tilde{y}} \frac{\partial^2}{\partial \tilde{x} \partial \tilde{y}} + \frac{1}{2} \sigma^2_{\tilde{y}} \frac{\partial^2}{\partial \tilde{y}^2},& \tilde{\Omega} := (0,1) \times (0,1)
  \end{align*}
  where $0 < \sigma_{\tilde{y}}^2 \leq 1$.

  \begin{itemize}
  \item This new parameterization makes the computational domain invariant
  to data/parameter combinations.
  \end{itemize}
\end{frame}
% ------------------------------------------------
\begin{frame}
  \frametitle{Joint density is given by the fourth derivative with respect to boundaries}
  Denoting the joint density of the diffusion process and the attained extrema over the interval $t$ as
  $f(x,y,a_x,b_x,a_y,b_y)$,
  \begin{align*}
  \frac{\partial^4}{\partial a_x \partial b_x \partial a_y \partial b_y} q(x,y,t) = f(x,y,a_x,b_x,a_y,b_y).
  \end{align*}

  \begin{itemize}
  \item The finite difference approximation of
    $\partial^4/\partial a_x \partial b_x \partial a_y \partial b_y$
    introduces a fundamental limitation due to finite precision
    round-off errors when the analytic solution $q(x,y,t)$ is not available
    \[
      \mathcal{O}\left(\frac{\varepsilon_{mach}}{\varepsilon^4} \right) \to \infty \,\,\, \mbox{ as } \,\,\, \varepsilon \to 0,
    \]
    where $\varepsilon$ is the step size of the numerical differentiation with respect to the boundaries.
  \end{itemize}
\end{frame}
% ------------------------------------------------

\begin{frame}
  \frametitle{Approach 1: Finite Difference approximation to FP equation}
  Requires the solution to a system of ODEs:
  \begin{align*}
    & \dot{c}(\tilde{t}) = Bc(\tilde{t}) &\Rightarrow& c(\tilde{t}) = \exp\left(B\tilde{t}\right)c(0), 
  \end{align*}
  where the sparse system matrix $B$ can be computed once and stored for a regular grid over $\tilde{\Omega}$ with finite step size $h$. 
  \[ B = \frac{1}{2} \frac{1}{h^2}B_{\tilde{x},\tilde{x}} +
    \rho\sigma_{\tilde{y}} \frac{1}{4h^2}B_{\tilde{x},\tilde{y}} +
    \frac{1}{2}\sigma_{\tilde{y}}^2
    \frac{1}{h^2}B_{\tilde{y},\tilde{y}}.
\]
\end{frame}
% ------------------------------------------------
\begin{frame}
  \frametitle{Approach 1: Finite Difference approximation to FP equation}
  Letting $k := 1/h$ and $b$ denote the boundary parameters, 
\begin{itemize}
  \item FD approximation is fast but also limited by irregular truncation errors
  \begin{multline*}
    q^{(k)}(\tilde{x},\tilde{y},\tilde{t} | b) - q(\tilde{x},\tilde{y},\tilde{t} | b) = \underbrace{\left( \frac{1}{k}
    \right)^{\alpha} \mbox{F}_{reg}(b)}_{\mbox{smooth in } b, \,\,\alpha > 0} \\
    + \underbrace{\left( \frac{1}{k}\right)^{\beta}\mbox{F}_{irreg}(b)}_{\mbox{continuous but not differentiable w.r.t }b, \,\, \beta > 0} + \underbrace{\varepsilon_{mach}F_{round}(b)}_{\mbox{behaves as a R.V}}
  \end{multline*}
\item With linear interpolation which arises when function arguments
  are not on grid points with respect to boundary differentiation
  \begin{align*}
    \mathcal{O}(F_{irreg}(b)) &= k^\beta/\varepsilon &\Rightarrow \left( \frac{1}{k}\right)^{\beta}\mbox{F}_{irreg}(b) &\to \infty \,\,\, \mbox{as} \,\,\, \varepsilon \to 0.
  \end{align*}
\end{itemize}
\end{frame}
% ------------------------------------------------
\begin{frame}
  \frametitle{Approach 2: Trigonometric expansion of the differential operator}

  \begin{align*}
    q(\tilde{x},\tilde{y},\tilde{t}) &\approx  \sum_\nu h_\nu \phi_\nu (\tilde{x}, \tilde{y}) e^{-\lambda_\nu \tilde{t}}.
  \end{align*}
  
  \begin{itemize}
  \item When $\rho = 0$,
    \begin{align*}
      q(\tilde{x},\tilde{y},\tilde{t}) &\approx \sum_{l=1}^L \sum_{m=1}^M c_{l,m}
                                         \sin\left(2\pi\, l\, \tilde{x} \right) \sin\left(2\pi\, m\, \tilde{y} \right)
    \end{align*}

    \begin{itemize}
    \item the system matrix is diagonal
    \item no new error is introduced in the time evolution
    \item truncated solution converges fast
    \end{itemize}
  \end{itemize}
\end{frame}

\begin{frame}
  \frametitle{Approach 2: Trigonometric expansion of the differential operator}
  \begin{itemize}
    
  \item When $\rho \neq 0$, mixing term in the FP equation produces
    $\cos(2\pi l \tilde{x})\cos(2\pi k \tilde{y})$ terms in the above
    expansion such that
    
    \begin{align*}
      \phi_\nu(\tilde{x},\tilde{y}) &= \sum_{l=1}^L \sum_{m=1}^M c_{l,m, \nu}
                                      \sin\left(2\pi\, l\, \tilde{x} \right) \sin\left(2\pi\, m\, \tilde{y} \right) := \Psi(\tilde{x},\tilde{y})^T c_\nu,
    \end{align*}
    
    \begin{itemize}
    \item system matrix is dense and convergence is slow,
    \item new error is introduced in the time evolution
    \end{itemize}
    
  \end{itemize}
\end{frame}
% ------------------------------------------------

\begin{frame}
  \frametitle{The Galerkin approach: weak solution to the PDE}
  We propose a solution $q^{(k)}(\tilde{x},\tilde{y},\tilde{t})$ of similar
  form
  \[
    q^{(k)}(\tilde{x},\tilde{y},\tilde{t}) = \sum_{i=0}^k c_i(\tilde{t})
    \psi_i(\tilde{x},\tilde{y}),
  \]
  where the basis functions $\psi_i(\tilde{x},\tilde{y})$ satisfy the
  boundary conditions on $\tilde{\Omega}$ and they need not be
  eigenfunctions but capture some essential features of the
  solution. We also require that all first- and second-order
  derivatives of $\psi_i(\tilde{x},\tilde{y})$ are in
  $L_2(\tilde{\Omega})$.
  \begin{align*}
  \frac{\partial}{\partial \tilde{t}} q^{(k)}(\tilde{x},\tilde{y},\tilde{t}) - \tilde{\mathcal{L}}q^{(k)}(\tilde{x},\tilde{y},\tilde{t}) &:= R_e(k), \\
  q(\tilde{x},\tilde{y},0) - q^{(k)}(\tilde{x},\tilde{y},0) &:= R_0(k).
  \end{align*}
\end{frame}
% ------------------------------------------------
\begin{frame}
  \frametitle{The Galerkin approach: weak solution to the PDE} The
  \textit{orthogonality} condition of the Galerkin procedure:
  \begin{align*}
    \displaystyle \int_{\Omega} R_e(k) \psi_i(\tilde{x},\tilde{y}) d\tilde{x}\,d\tilde{y} &= 0,& \displaystyle \int_{\Omega} R_0(k) \psi_i(\tilde{x},\tilde{y}) d\tilde{x}\,d\tilde{y} &= 0,& i = 0,\ldots k, 
  \end{align*}
  which is equivalent to the weak formulation of the heat problem
  \begin{align*}
    \left< \partial_t q^{(k)}(\tilde{x},\tilde{y},\tilde{t}), \psi_i \right> &= \left<\tilde{\mathcal{L}}q^{(k)}(\tilde{x},\tilde{y},\tilde{t}), \psi_i \right>, \\
    \left< q^{(k)}(\tilde{x},\tilde{y},0), \psi_i \right> &= \left<q(\tilde{x},\tilde{y},0), \psi_i\right>,
  \end{align*}
  where $<\cdot, \cdot>$ is the usual inner product in
  $L_2(\tilde{\Omega})$.
\end{frame}
% ------------------------------------------------
\begin{frame}
  \frametitle{Basis element choice}
  \begin{align*}
    \psi_i(\tilde{x},\tilde{y}) &= \frac{1}{2\pi \tilde{\sigma}^2\sqrt{1-\tilde{\rho}^2} } \\
                                &\quad \times \exp\left\{ -\frac{\left( (\tilde{x} - \tilde{x}_i)^2 - 2\tilde{\rho} (\tilde{x}-\tilde{x}_i)(\tilde{y}-\tilde{y}_i) + (\tilde{y} - \tilde{y}_i)^2 \right)}{2(1-\tilde{\rho}^2)\tilde{\sigma}^2}  \right\} \nonumber \\
                                &\quad \times \tilde{x}\left(1-\tilde{x}\right)\, \tilde{y}(1-\tilde{y}) \nonumber
  \end{align*}
  for some parameters $(\tilde{\rho}, \tilde{\sigma})$ and a collection of nodes
  $\{ (\tilde{x}_i,\tilde{y}_i) \}_{i=0}^k$ which form a grid over
  $\tilde{\Omega}$.
  \begin{itemize}
  \item Align basis elements along the principal axis of the differential operator,
  \item Spacing between nodes is $l$ times the bandwidth $\tilde{\sigma}$ in each principal direction
  \item This scheme is aimed at better resolving the correlatio in the solution
  \end{itemize}
\end{frame}
% ------------------------------------------------
\begin{frame}
\begin{figure}
  \centering
  %%
  %%
  \begin{tabular}{cc}
    \begin{minipage}{0.4\textwidth}
      \centering
      \includegraphics[width=1\linewidth]{../nodes-1.pdf}
    \end{minipage}
    %%
    & \begin{minipage}{0.4\textwidth}
      \centering
      \includegraphics[width=1\linewidth]{../nodes-2.pdf}
    \end{minipage}
  \end{tabular}
  %%
  %%
  % \caption{A sample grid design for $l=1$, $\sigma=0.3$ and $\rho=0.6$. The
  %   left panel corresponds to the initial grid
  %   $\{ (x'_j,y'_j) \}_{j=0}^{k'}$ over $\Omega$ (solid black
  %   square). The right panel depicts the rotated initial grid. The set
  %   of final node points $\{ (x_i,y_i) \}_{i=0}^{k}$ is contained
  %   within $\Omega$ and is denoted by the red solid points.}
\end{figure}
\begin{itemize}
  \item A sample grid design for $l=1$, $\tilde{\sigma}=0.3$ and
    $\tilde{\rho}=0.6$.
  \item \textbf{Left:} Grid along coordinates of observed prices
    within the computational domain (solid black square).
  \item \textbf{Right:} Grid along the major axes of the basis
    element kernels. The set of final node points
    $\{ (x_i,y_i) \}_{i=0}^{k}$ is contained within the computational
    domain and is denoted by the red solid points.
\end{itemize}
\end{frame}
% ------------------------------------------------
\begin{frame}
  \frametitle{Small-time solution to the Fokker-Planck equation}
  \begin{itemize}
  \item A semi-analytic soluiont for a small time $\tilde{t}_\epsilon$ allows us to project a smooth function onto our basis family instead of a $\delta$-function
  \item This reduces the numerical error of the Galerkin solver
    \begin{figure}
      \centering
      %% 
      %% 
      \begin{tabular}{cc}
        \begin{minipage}{0.3\textwidth}
          \centering
          \includegraphics[width=1\linewidth]{../small-time-solution.png}
        \end{minipage}
        %% 
        & \begin{minipage}{0.3\textwidth}
          \centering
          \includegraphics[width=1\linewidth]{../small-time-solution-contour.png}
        \end{minipage}
      \end{tabular}
      %% 
      %% 
    \end{figure}
  \item \textbf{Left:} Computational domain under the transformation
    where diffusion parameters are both unity and correlation is zero.
  \item \textbf{Right:} Contour of small-time solution in original coordinate system.
  \end{itemize}
\end{frame}
% ------------------------------------------------
\begin{frame}
  \centering
  \includegraphics[scale=0.40]{../small-time-solution.png}
\end{frame}
% ------------------------------------------------
\begin{frame}
  \centering
  \includegraphics[scale=0.55]{../small-time-solution-contour.png}
\end{frame}
% ------------------------------------------------
\begin{frame}
  \frametitle{Consistency Results}
  Denoting the approximate density due to the Galerkin solution as
  \[
    f^{(k)}(x,y,a_x,b_x,a_y,b_y) = \frac{\partial^4
      q^{(k)}(x,y,t)}{\partial a_x \partial b_x \partial a_y \partial
      b_y},
  \]
  we show that the two integrals below converge in $L_2(\Omega)$:
  \begin{lemma}[1]\label{lem:1}
  \begin{multline*}
    \lim_{k\to \infty} \displaystyle \int_{a_x}^{b_x} \displaystyle
    \int_{a_y}^{b_y} \left( f^{(k)}(x,y,a_x,b_x,a_y,b_y) - f(x,y,a_x,b_x,a_y,b_y) \right)^2\, dx\,dy \\
    = 0
  \end{multline*}
\end{lemma}
\end{frame}
% ------------------------------------------------
\begin{frame}
Denoting the random variable
\[Z := (X(t), Y(t), m_X(t), M_X(t), m_Y(t), M_Y(t)),\] we consider the
distribution function $F(Z \leq z)$ generated by the true solution to the
Fokker-Planck equation as well as the distribution $F^{(k)}(Z \leq z)$
generated by the Galerkin solution.
\begin{lemma}[Convergence in distribution] \label{lem:conv-dist}
  For any $z \in Z$,
  $ \lim_{k \to \infty} F^{(k)}(Z \leq z ) = F(Z \leq z).$
\end{lemma}

Assuming that the maximum likelihood estimate (MLE) (under $F^{(k)}$
for sufficiently large $k$) is continuous with respect to the data, we
can use the convergence result above and Chebyshev's inequality to
show that the MLE under $F^{(k)}$ is consistent as the number of basis
elements $k$ and number of data points $n$ go to infinity.
\end{frame}
% ------------------------------------------------
\begin{frame}
  \frametitle{Results and solution behavior for small $\tilde{t}$}
  \framesubtitle{Here we consider the parameters defining the basis family
  $(\tilde{\rho}, \tilde{\sigma})$}
  \begin{figure}
  \centering
  \includegraphics[scale=0.45]{../chapter-2/figures/{limitations-rho-0.95-data-point-4}.pdf}
\end{figure}
The behavior of Galerkin solution is valid only up to a some small time $\tilde{t}$.
\end{frame}
% ------------------------------------------------
\begin{frame}
  \frametitle{Simulation Study} We consider $50$ simulated data sets with
  \begin{align*}
    \sigma_x &= 1, & \sigma_y &= 1, & \mu_x &= 0, & \mu_y &= 0
  \end{align*}
  with increasing sample size over several values of $\rho$. For each
  case, we consider the ratio of mean-square error of the Galerkin
  solver compared to that generated by Gaussian likelihoods ignoring the boundaries:
  \begin{table}  
  \centering
  \begin{tabular}{cccc|ccc}
    &  \multicolumn{3}{c}{$\rho=0.95$} & \multicolumn{3}{c}{$\rho=0.60$} \\
    & $m=4$ & $m=8$ & $m=16$ & $m=4$ & $m=8$ & $m=16$ \\
    \hline
    $\hat{\sigma}_x$ & 0.475 & 1.238 & 1.304 & 0.203 & 0.127 & 0.232   \\
    \hline
    $\hat{\sigma}_y$ & 0.593 & 1.040 & 1.088 &  0.111 & 0.120 & 0.260 \\
    \hline
    $\hat{\rho}$ & 0.287 & 0.910 & 0.445 & 0.315 & 0.283 & 0.463 
  \end{tabular}
\end{table}

  \begin{table}  
  \centering
  \begin{tabular}{cccc}
    & \multicolumn{3}{c}{$\rho=0.0$}\\
    & $m=4$ & $m=8$ & $m=16$ \\
    \hline
    $\hat{\sigma}_x$ & 0.137 & 0.243 & 0.167 \\
    \hline
    $\hat{\sigma}_y$ &  0.189 & 0.171 & 0.107 \\
    \hline
    $\hat{\rho}$ &  0.517 & 0.365 & 0.194
  \end{tabular}
  \end{table}

\end{frame}
% ------------------------------------------------
\begin{frame}
  \frametitle{Simulation study}
%   \begin{figure}
%   \centering
%   %%
%   %%
%   \begin{tabular}{ccc}
    
%     \begin{minipage}{0.3\textwidth}
%       \centering
%       \includegraphics[width=1\linewidth]{../chapter-2/results/mle-results-rho-0.95-n-16/estimates-rho.pdf}
%     \end{minipage}
%     & \begin{minipage}{0.3\textwidth}
%       \centering
%       \includegraphics[width=1\linewidth]{../chapter-2/results/mle-results-rho-0.95-n-16/estimates-sigma-x.pdf}
%     \end{minipage}
%     & \begin{minipage}{0.3\textwidth}
%       \centering
%       \includegraphics[width=1\linewidth]{../chapter-2/results/mle-results-rho-0.95-n-16/estimates-sigma-y.pdf}
%     \end{minipage}
%   \end{tabular}
% \end{figure}
  
\begin{itemize}
\item For data generated with $\rho=0.95$, kernel-density approximations
  of the repeated-sampling densities of the MLEs are shown.  Samples
  are obtained from the Galerkin likelihood (green) and the classical
  Gaussian likelihood (red) The data-generating parameters are denoted
  with the vertical solid line.

\item The Rogers estimator (blue) is the only existing correlation estimator for
  bivariate OCHL data.
  \end{itemize}
\end{frame}
%------------------------------------------------
\begin{frame}
  \centering
  \includegraphics[scale=0.5]{../chapter-2/results/mle-results-rho-0.95-n-16/estimates-rho.pdf}
\end{frame}
\begin{frame}
  \centering
  \includegraphics[scale=0.5]{../chapter-2/results/mle-results-rho-0.95-n-16/estimates-sigma-x.pdf}
\end{frame}
\begin{frame}
  \centering
  \includegraphics[scale=0.5]{../chapter-2/results/mle-results-rho-0.95-n-16/estimates-sigma-y.pdf}
\end{frame}
%------------------------------------------------
%------------------------------------------------
\section{Open, Close, High, Low Prices: Small Time Solution}
% ------------------------------------------------
\begin{frame}
Open, Close, High, Low Prices: Small Time Solution and Gap-Fill Solution
\end{frame}
\begin{frame}
  \frametitle{Method of images and analytic differentiability}
  \centering
  \includegraphics[scale=0.45]{../chapter-3/figures/chapter-3-figure-illustration-1.pdf}
\end{frame}
% ------------------------------------------------
\begin{frame}
  \frametitle{Symmetry constraint in constructing approximate solutions}

  A condition weaker than uniqueness which nonetheless restricts the
solution space is the symmetry obeyed by the problem. We consider the transformation
\begin{align*}
  x^{new} = (a_x + b_x) - x^{old}, \\
  y^{new} = (a_y + b_y) - y^{old}.
\end{align*}

Performing the set of reflections
\begin{align*}
  \left\{ 2,4,1,3 \right\} \cup \left\{ 2,4,3,1 \right\} \cup \left\{ 4,2,1,3 \right\} \cup \left\{ 4,2,3,1 \right\}. 
\end{align*}
produces a sum of images where only four elements are differentiable
with respecto to all four boundaries:

\[
  \frac{\partial^4 p_\epsilon(\tilde{x}, \tilde{y}, \tilde{t})}{\partial a_x \partial b_x \partial a_y \partial b_y}  = \sum_{j'=1}^{4}
                                                                                                                        \frac{\partial^4G(\tilde{x},\tilde{y},\tilde{t}|\tilde{x}_{(j')},\tilde{y}_{(j')})}{\partial a_x \partial b_x \partial a_y \partial b_y}.
\]

\end{frame}
% ------------------------------------------------
\begin{frame}
  \frametitle{Calculation of the joint density}
  We can express the derivatives:
\begin{align*}
  % \frac{\partial}{\partial a_x} G(x,y|t^{*}, \tilde{\sigma}, \rho, x_0^{(j^*)}, y_0^{(j^*)}) &= G \cdot \mathcal{C}\,\,\cdot \left( \frac{\partial \mathcal{P}}{\partial a_x} \right)\\
  % \frac{\partial^2}{\partial a_x \partial b_x} G(x,y|t^{*}, \tilde{\sigma}, \rho, x_0^{(j^*)}, y_0^{(j^*)}) &= G \cdot \mathcal{C}^2 \left( \frac{\partial \mathcal{P}}{\partial b_x} \right) \left( \frac{\partial \mathcal{P}}{\partial a_x} \right) + G \cdot \mathcal{C} \cdot \left( \frac{\partial^2 \mathcal{P}}{\partial a_x \partial b_x} \right) \\
  % %% %% %%
  % \frac{\partial^3}{\partial a_x \partial b_x \partial a_y} G(x,y|t^{*}, \tilde{\sigma}, \rho, x_0^{(j^*)}, y_0^{(j^*)}) &= G \cdot \mathcal{C}^3 \left( \frac{\partial \mathcal{P}}{\partial b_x} \right) \left( \frac{\partial \mathcal{P}}{\partial a_x} \right) \left( \frac{\partial \mathcal{P}}{\partial a_y} \right) \nonumber \\
  %                                                                                            &\,\, + G \cdot \mathcal{C}^2 \left[\left( \frac{\partial^2 \mathcal{P}}{\partial a_x \partial a_y} \right) \left( \frac{\partial \mathcal{P}}{\partial b_x} \right) + \left( \frac{\partial^2 \mathcal{P}}{\partial b_x \partial a_y} \right) \left( \frac{\partial \mathcal{P}}{\partial a_x} \right) + \left( \frac{\partial^2 \mathcal{P}}{\partial a_x \partial b_x} \right) \left( \frac{\partial \mathcal{P}}{\partial a_y} \right)\right] \nonumber \\
  %                                                                                            &\,\, + G \cdot \mathcal{C} \left( \frac{\partial^3 \mathcal{P}}{\partial a_x \partial b_x \partial a_y} \right) \nonumber \\
  %                                                                                            %% %% %%
  \frac{\partial^4}{\partial a_x \partial b_x \partial a_y \partial b_y} G(\tilde{x},\tilde{y},\tilde{t} |  \tilde{x}_{(j)}, \tilde{y}_{(j)}) &= \\
  \MoveEqLeft[15] G \cdot \mathcal{C}^4 \cdot \left(\partial_{a_x}\partial_{b_x} \partial_{a_y}\partial_{b_y} \right)\mathcal{P}&  \nonumber \\
  \MoveEqLeft[15] \,\, + G \cdot \mathcal{C}^3 \cdot \left( \partial^2_{a_x\, b_x} \partial_{a_y} \partial_{b_y} + \partial^2_{a_x\, a_y} \partial_{b_x} \partial_{b_y} + \partial^2_{a_x\, b_y} \partial_{b_x} \partial_{a_y} + \right. &\nonumber \\
  \MoveEqLeft[15] \left. \qquad\qquad\qquad +\partial^2_{b_x\, a_y} \partial_{a_x} \partial_{b_y} + \partial^2_{b_x\, b_y} \partial_{a_x} \partial_{a_y} \partial^2_{a_y\, b_y} \partial_{a_x} \partial_{b_x} \right) \mathcal{P} &  \nonumber \\
  \MoveEqLeft[15] \,\, + G \cdot \mathcal{C}^2 \cdot \left( \partial^3_{a_x\,b_x\,a_y} \partial_{b_y} + \partial^2_{a_x\,b_x}\partial^2_{a_y\,b_y} + \partial^3_{a_x\,b_x\,b_y}\partial_{a_y}  + \right.& \nonumber \\
  \MoveEqLeft[15] \left. \qquad\qquad\qquad + \partial^3_{a_x\,a_y\,b_y} \partial_{b_x} + \partial^2_{a_x\,a_y} \partial^2_{b_y\,b_x} + \partial^3_{b_x\,a_y\,b_y}\partial_{a_x} + \partial^2_{b_x\,a_y}\partial_{a_x}\partial_{b_y} \right) \mathcal{P} & \nonumber \\
  \MoveEqLeft[15] \,\, + G \cdot \mathcal{C} \cdot \partial^4_{a_x\,b_x\,a_y\,b_y} \mathcal{P}& \label{eq:fourth-deriv}
\end{align*}
\end{frame}
% ------------------------------------------------
\begin{frame}
  \frametitle{Analytic Derivative and higher order terms}

  Thinking of $\tilde{t}$ as variable allows us to further simplify
  the expression. Since $\mathcal{C} = \mathcal{O}(1/\tilde{t})$, all
  three terms $G\cdot \mathcal{C}^3, G\cdot \mathcal{C}^2,$ and
  $G\cdot \mathcal{C}$ are
  $o\left( G\cdot \mathcal{C}^4 \cdot
    \left(\partial_{a_x}\partial_{b_x} \partial_{a_y}\partial_{b_y}
    \right)\mathcal{P} \right)$, so that the $G\cdot \mathcal{C}^4$
  order term in the derivative dominates the others for sufficiently
  small $\,\,\tilde{t}$.

  \begin{align*}
  \frac{\partial^4 p_\epsilon(\tilde{x}, \tilde{y}, \tilde{t})}{\partial a_x
  \partial b_x \partial a_y \partial b_y} &\approx \sum_{j'=1}^{4} G(\tilde{x},\tilde{y},\tilde{t}|\tilde{x}_{(j')},\tilde{y}_{(j')}) \cdot \mathcal{C}^4 \cdot \left(\partial_{a_x}\partial_{b_x} \partial_{a_y}\partial_{b_y} \right)\mathcal{P}_{j'}. 
\end{align*}
\end{frame}
% ------------------------------------------------

% ------------------------------------------------
\begin{frame}
  \frametitle{Basis Expansion of Galerkin likelihood: RHS}
  The likelihood computed with the Galerkin solution as a
function of $\tilde{t}$ is of the form
\begin{align*}
  \frac{\partial^4 p_{G}(\tilde{x}, \tilde{y}, \tilde{t})}{\partial a_x
  \partial b_x \partial a_y \partial b_y} = \sum_{k=1}^{K} e^{-\lambda_k\tilde{t}} p_k^{(4)}(\tilde{t}),
\end{align*}
where $p_i^{(4)}(\tilde{t})$ is a fourth-order polynomial. This
proceeds from the Galerkin solution being dependent on $\tilde{t}$
only through the exponential term: the eigenfunctions of the solution
are by design solely functions of $(a_x, b_x, a_y, b_y)$ and
$(\tilde{x}, \tilde{y})$


\begin{itemize}
\item This suggests a leading-order approximation that is fitted with the Galerking solver via least squares:
  \begin{align*}
     f_{LS}(\tilde{t}) &= \tilde{t}^4 \left( \omega_1 e^{-\lambda_1\tilde{t}} + \omega_2 e^{-\lambda_2\tilde{t}}\right), \\
    \Rightarrow \log f_{LS}(\tilde{t}) &= \log(\omega_1) + 4 \log(\tilde{t}) -\lambda_1\tilde{t} + \log\left(1 + \omega_2/\omega_1 e^{-(\lambda_2-\lambda_1)\tilde{t}} \right)
  \end{align*}
\end{itemize}
\end{frame}
% ------------------------------------------------
\begin{frame}
  \frametitle{Leading order for small-time solution: LHS}

  The summand in the small-time likelihood with the greatest
  $\beta_{j'}$ contributes the most to the truncated small-time
  solution in the $\tilde{t} \leq 1$ region where the matched solution
  will be applied. Indexing $j'$ such that
  $\beta_1 \geq \beta_2 \geq \beta_3 \geq \beta_4$, the small-time
  log-likelihood is
\begin{align*}
  \log\left( \frac{\partial^4 p_\epsilon(\tilde{x}, \tilde{y}, \tilde{t})}{\partial a_x
  \partial b_x \partial a_y \partial b_y} \right) &\approx \log(K) - 4.5\log(\tilde{t}) + \log(c_1) - \frac{\beta_1}{\tilde{t}} \nonumber \\
  &\quad + \log\left(1 + \sum_{j \neq 1} \frac{c_j}{c_1}\exp\left( -\frac{(\beta_j-\beta_1)}{\tilde{t}} \right) \right) \nonumber \\
  &\approx \log(K) - 4.5\log(\tilde{t}) + \log(c_1) - \frac{\beta_1}{\tilde{t}} + \log\left(1 + \epsilon(\tilde{t}) \right), 
\end{align*}

  \begin{itemize}
\item The proposed matched solution is of the form
  \[
    \log f_{\mbox{gap}}(\tilde{t}) = \log(\omega(\tilde{t})) -
    \gamma(\tilde{t})\log(\tilde{t}) -
    \frac{\beta(\tilde{t})}{\tilde{t}}
  \]
  \end{itemize}
\end{frame}
% ------------------------------------------------
\begin{frame}
  \begin{align*}
    \log f_{LS}(\tilde{t}) &= \log(\omega_1) + 4 \log(\tilde{t}) -\lambda_1\tilde{t} + \log\left(1 + \omega_2/\omega_1 e^{-(\lambda_2-\lambda_1)\tilde{t}} \right) \\
    \log f_{\mbox{gap}}(\tilde{t}) &= \log(\omega(\tilde{t})) -
                                         \gamma(\tilde{t})\log(\tilde{t}) -
                                         \frac{\beta(\tilde{t})}{\tilde{t}}
  \end{align*}

  \begin{itemize}
  \item At $\tilde{t}^*$, the
    left-hand side of the matching condition, the values for these
    parameters are defined such that they match the small-time solution
    \begin{align*}
      \omega(\tilde{t}^*) &= K, & \gamma(\tilde{t}^*) &= 4.5, & \beta(\tilde{t}^*) &= \beta_1.
    \end{align*}
  \item At $\tilde{t}_m$, the right-hand side of the matching condition and
    the maximum of the LS solution,
    $\omega(\tilde{t}), \gamma(\tilde{t}),$ and $\beta(\tilde{t})$ are
    chosen to match the value, first, and second derivatives of the
    logarithmic form of the LS solution. The form of the parameters
    between $\tilde{t}^*$ and $\tilde{t}_m$ is chosen to be a sigmoid
    function which rapidly transitions away from $\tilde{t}^*$ and is the
    same for all three parameters.
  \end{itemize}
  \end{frame}

  \begin{frame}
    \begin{align*}
      \omega(\tilde{t}) &= \omega(\tilde{t}^*)e^{-k(\tilde{t}-\tilde{t}^*)} + \omega(\tilde{t}_m)\left(1-e^{-k(\tilde{t}-\tilde{t}^*)}\right), \\
      \gamma(\tilde{t}) &= \gamma(\tilde{t}^*)e^{-k(\tilde{t}-\tilde{t}^*)} + \gamma(\tilde{t}_m)\left(1-e^{-k(\tilde{t}-\tilde{t}^*)}\right), \\
      \beta(\tilde{t}) &= \beta(\tilde{t}^*)e^{-k(\tilde{t}-\tilde{t}^*)} + \beta(\tilde{t}_m)\left(1-e^{-k(\tilde{t}-\tilde{t}^*)}\right).
    \end{align*}
\end{frame}
% ------------------------------------------------
\begin{frame}
  \centering
  \includegraphics[scale=0.8]{../chapter-3/figures/{matched-rho-0.95-data-point-4}.pdf}
\end{frame}
% ------------------------------------------------
\begin{frame}
  \frametitle{Repeated Simulation Results}
  \begin{table}
  \centering
  \begin{tabular}{cccc}
    \multicolumn{4}{c}{$\rho=0.95$} \\
    & $m=4$ & $m=8$ & $m=16$ \\
    \hline
    $\hat{\sigma}_x$ & 0.124 & 0.429  & 0.318 \\
    \hline
    $\hat{\sigma}_y$ & 0.310 & 0.147  & 0.365 \\
    \hline
    $\hat{\rho}$ & 0.250 & 0.753 & 0.699
  \end{tabular}
\end{table}
\end{frame}

\begin{frame}
\centering
\includegraphics[scale=0.5]{../chapter-3/results/mle-results-rho-0.95-n-16/estimates-sigma-x.pdf}
\end{frame}


\section{Future Work}
\begin{frame}
  \begin{itemize}
  \item Multivariate High-Frequency models
  \item Gaussian Process Emulator for interpolating the OCHL likelihood
  \item Particle Filter
  \end{itemize}
\end{frame}

\begin{frame}
  \frametitle{Illustration of proof for existence}
    \begin{tabular}{cc}
    \begin{minipage}{0.5\textwidth}
      \centering
      \includegraphics[width=1\linewidth]{../chapter-3/chapter-3-figure-proof-1.pdf}
    \end{minipage}
    %%
    & \begin{minipage}{0.5\textwidth}
      \centering
      \includegraphics[width=1\linewidth]{../chapter-3/chapter-3-figure-proof-2.pdf}
    \end{minipage}
  \end{tabular}
\end{frame}


\begin{frame}[allowframebreaks]
  \frametitle{References}
  \bibliographystyle{plain}
  \bibliography{../master-bibliography}
  \end{frame}

\end{document}

