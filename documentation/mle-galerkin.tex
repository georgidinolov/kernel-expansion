\documentclass[10pt]{article}

\usepackage{amsthm}
\usepackage[toc,page]{appendix}
\usepackage{amssymb}
\usepackage{bm}
\usepackage{pslatex,palatino,avant,graphicx,color}
\usepackage{colortbl}
\usepackage{fullpage}
\usepackage{mathtools}
\usepackage{natbib}
\usepackage{amsmath}
\usepackage{caption}
%\usepackage{subfigure}
\usepackage{subcaption}
\usepackage{bm}
\usepackage{wrapfig}
\usepackage{enumerate}
\usepackage{rotating}
\usepackage{multirow}
\usepackage{tabularx}
\usepackage{tikz}
\usepackage{pgfplots}
\usepackage{bbm}
\usepackage[titles,subfigure]{tocloft} % Alter the style of the Table of Contents

\newtheorem{lemma}{Lemma}
%\pdfminorversion=4
% NOTE: To produce blinded version, replace "0" with "1" below.
\newcommand{\blind}{0}

% DON'T change margins - should be 1 inch all around.
\addtolength{\oddsidemargin}{0in}%
\addtolength{\evensidemargin}{0in}%
\addtolength{\textwidth}{0in}%
\addtolength{\textheight}{0in}%
\addtolength{\topmargin}{0in}%


\renewcommand{\cftsecfont}{\rmfamily\mdseries\upshape}
\renewcommand{\cftsecpagefont}{\rmfamily\mdseries\upshape} % No bold!

\newcommand{\indicator}[1]{\mathbbm{1}\left( #1 \right) }
\newcommand{\hb}{\hat{b}}
\newcommand{\ha}{\hat{a}}
\newcommand{\htheta}{\hat{\theta}}
\newcommand{\halpha}{\hat{\alpha}}
\newcommand{\hmu}{\hat{\mu}}
\newcommand{\hsigma}{\hat{\sigma}}
\newcommand{\hphi}{\hat{\phi}}
\newcommand{\htau}{\hat{\tau}}
\newcommand{\heta}{\hat{\eta}}
\newcommand{\E}[1]{\mbox{E}\left[#1\right]}
\newcommand{\Var}[1]{\mbox{Var}\left[#1\right]}
\newcommand{\Indicator}[1]{\mathbbm{1}_{ \left( #1 \right) } }
\newcommand{\dNormal}[3]{ N\left( #1 \left| #2, #3 \right. \right) }
\newcommand{\Beta}[2]{\mbox{Beta}\left( #1, #2 \right)}
\newcommand{\alphaphi}{\alpha_{\hphi}}
\newcommand{\betaphi}{\beta_{\hphi}}
\newcommand{\expo}[1]{ \exp\left\{ #1 \right\}}
\newcommand{\tauSquareDelta}{\htau^2
  \left(\frac{1-\expo{-2\htheta\Delta}}{2\htheta} \right)}
\newcommand{\mumu}{\mu_{\hmu}}
\newcommand{\sigmamu}{\sigma^2_{\hmu}}
\newcommand{\sigmamuexpr}{\log\left( \frac{\VarX}{\EX^2} + 1 \right)}
\newcommand{\mumuexpr}{\log(\EX) -  \log\left( \frac{\VarX}{\EX^2} + 1 \right) /2 }

\newcommand{\EX}{\mbox{E}\left[ X \right] }
\newcommand{\VarX}{\mbox{Var}\left[ X \right] }
\newcommand{\mueta}{\mu_{\heta} }
\newcommand{\sigmaeta}{\sigma^2_{\heta}}
\newcommand{\sigmaetaexpr}{ \log\left( \frac{\VarX}{\EX^2} + 1 \right) }
\newcommand{\muetaexpr}{ \log(\EX) -  \sigmaetaexpr /2 }

\newcommand{\mualpha}{\mu_{\halpha} }
\newcommand{\sigmaalpha}{\sigma^2_{\halpha}}
\newcommand{\sigmaalphaexpr}{ \log\left( \frac{\VarX}{\EX^2} + 1 \right) }
\newcommand{\mualphaexpr}{ \log(\EX) -  \sigmaalphaexpr /2 }

\newcommand{\mutauexpr}{ \frac{2}{T} \EX }
\newcommand{\sigmatauexpr}{ \frac{4}{T^2} \Var{X}}

\newcommand{\alphatau}{\alpha_{\htau^2}}
\newcommand{\betatau}{\beta_{\htau^2}}

\newcommand{\Gam}[2]{\mbox{Gamma}\left( #1, #2 \right) }
\newcommand{\InvGam}[2]{\mbox{Inv-Gamma}\left( #1, #2 \right) }

%%% END Article customizations

%%% The "real" document content comes below...

% \newbox{\LegendeA}
% \savebox{\LegendeA}{
%    (\begin{pspicture}(0,0)(0.6,0)
%    \psline[linewidth=0.04,linecolor=red](0,0.1)(0.6,0.1)
%    \end{pspicture})}
% \newbox{\LegendeB}
%    \savebox{\LegendeB}{
%    (\begin{pspicture}(0,0)(0.6,0)
%    \psline[linestyle=dashed,dash=1pt 2pt,linewidth=0.04,linecolor=blue](0,0.1)(0.6,0.1)
%    \end{pspicture})}

\title{Solution to a Non-Seperable Diffusion Equation on a Regular Domain}
\author{Georgi Dinolov, Abel Rodriguez, Hongyun Wang}
\date{} % Activate to display a given date or no date (if empty),
         % otherwise the current date is printed 

\begin{document}
\def\spacingset#1{\renewcommand{\baselinestretch}%
{#1}\small\normalsize} \spacingset{1}

\bigskip

\vspace{1cm}
\noindent

\spacingset{1.00} % 
\section{Introduction}

We consider two-dimensional correlated Brownian motion with absorbing boundaries:
\begin{align}
  X(t) &= x_0 + \mu_x t + \sigma_x W_x(t) &a_x &\leq X(t) \leq b_x   \label{eq:X} \\
  Y(t) &= y_0 + \mu_y t + \sigma_y W_y(t) &a_y &\leq Y(t) \leq b_y   \label{eq:Y} 
\end{align}
where $W_i$ are standard Brownian motions with
$\mbox{Cov}(W_1(t), W_2(t)) = \rho t$. In
particular, we find the joint transition density function for
$(X(t), Y(t))$ under the boundary conditions:
\begin{align}
  \frac{1}{dx dy} \mbox{Pr}\left( \left. \begin{array}{ll}
X(t) \in [x, x+dx), Y(t) \in [y, y+dy),  \\
\displaystyle \min_{t'\in[0, t]} X(t') \ge a_x, \max_{t'\in[0, t]} X(t') \le b_x, \\
\displaystyle \min_{t'\in[0, t]} Y(t') \ge a_y, \max_{t'\in[0, t]} Y(t') \le b_y
\end{array} \right | X(0)=x_0, Y(0)=y_0, \theta
\right), \label{eq:CDF}
  % \Pr\left(X(t) \in dx, Y(t) \in dy,  \min_{t'}X(t') \geq a_x,
  % \max_{t'}X(t')\leq b_x, \min_{t'} Y(t')\geq a_y, \max_{t'} Y(t')\leq b_y|  X(0)=x_0, Y(0)=y_0, \theta \right), \label{eq:CDF}
\end{align}
with $\theta := (\mu_x, \mu_y, \sigma_x, \sigma_y, \rho).$ This
function, which we shorten to $q(x,y,t)$ from now on, is the solution
to the Fokker-Planck equation \citep{oksendal2013stochastic}:
\begin{align}
  \displaystyle \frac{\partial}{\partial t'} q(x,y,t') &= -\mu_x \frac{\partial}{\partial x}q(x,y,t')
                                         - \mu_y \frac{\partial}{\partial y}q(x,y,t')
                                         + \frac{1}{2}\sigma_x^2 \frac{\partial^2}{\partial x^2}q(x,y,t')
                                         + \rho\sigma_x\sigma_y \frac{\partial^2}{\partial x \partial y}q(x,y,t')
                                         + \frac{1}{2}\sigma_y^2 \frac{\partial^2}{\partial y^2}q(x,y,t'),
  \label{eq:1} \\
  q(a_x, y,t') &= q(b_x,y,t') = q(x,a_y,t') = q(x,b_y,t') = 0, \label{eq:2} \\
   0 &< t' \leq t. \nonumber
\end{align}
Differentiating $q(x,y,t)$ with respect to the boundaries produces the
transition density of a particle beginning and ending at the points
$(X_1(0), X_2(0))$ and $(X_1(t), X_2(t))$, respectively, while
attaining the minima $a_x/a_y$ and maxima $b_x/b_y$ in each coordinate
direction:
\begin{align}
  \frac{\partial^4}{\partial a_x \partial b_x \partial a_y \partial b_y} q(x,y,t) = 
\mbox{density} \left( \left. \begin{array}{ll}
X(t) =x, Y(t) =y,  \\
\displaystyle \min_{t'\in[0, t]} X(t') = a_x, \max_{t'\in[0, t]} X(t') = b_x, \\
\displaystyle \min_{t'\in[0, t]} Y(t') = a_y, \max_{t'\in[0, t]} Y(t') = b_y
\end{array} \right | X(0)=x_0, Y(0)=y_0, \theta
\right) \label{eq:pdf}
\end{align}
The transition density (\ref{eq:CDF}) with less than all four
boundaries has been used in computing first passage times
\citep{kou2016first, sacerdote2016first}, with application to
structural models in credit risk and default correlations
\citep{haworth2008modelling, ching2014correlated}. \cite{he1998double}
use variants of (\ref{eq:pdf}) with respect to some of the boundaries
to price financial derivative instruments whose payoff depends on
\textbf{some} of the observed maxima/minima. \cite{rodriguez2012} use
a full likelihood-based (Bayesian) approach to estimate volatility in
\textit{univariate} financial timeseries where open, closing, highest,
and lowest prices are included. Their work fits into a body of
literature and collection of techniques by practitioners where the
observed range of prices is used to make similar estimates. In this
paper we will provide the efficient numerical method necessary for
carrying out inferential procedures with correlated financial
timeseries.

Closed-form solutions to (\ref{eq:1}) - (\ref{eq:2}) are available for
some parameter regimes. When $\rho = 0$, the transition density of the
process is the solution to a well-understood Sturm-Liouville problem
where the eigenfunctions of the differential operator are sine
functions. When $a_x = -\infty$ and $b_x = \infty$, the method of
images can be used to enforce the remaining boundaries. For either
$a_x, a_y = -\infty$ or $b_x, b_y = \infty$, eigenfunction of the
Fokker-Plank equation can be found in radial coordinates. Both of
these techniques are used and detailed by
\cite{he1998double}. However, to the best of our knowledge, there is
no closed-form solution to the general problem in (\ref{eq:1}) -
(\ref{eq:2}). This also limits the available ways to compute
(\ref{eq:pdf}), with the most straightforward approach being finite
difference with respect to the boundary conditions. This, however,
requires one to solve Fokker-Planck equation (\ref{eq:1}) - (\ref{eq:2})
accurately for at least 16 slightly different sets of boundaries and
combine the results with numerical differentiation to evaluate the
density function for a just single observation (see Section
\ref{sec:likelihood-calc}), motivating the need for an efficient
numerical method to solve (\ref{eq:1}) - (\ref{eq:2}).

For non-zero correlation, it is still possible to approach the general
problem by proposing a biorthogonal expansion in time and space
(\cite{risken1989fokker-planck}, sections 6.2), where the
eigenfunctions for the differential operator are approximated
 by the
eigenvectors of a linear system obtained using a truncated expansion
based on a set of separable basis function, each of which is a product
of two sine functions (one in each dimension) satisfying the boundary
conditions. However, a drawback of this out-of-the-box solution is
that the system matrix for the corresponding eigenvalue problem is
dense. Additionally, for nonzero $\rho$, it is necessary to have a
large number of basis elements for an accurate approximation. The
denseness and size of the resultant system matrices makes the
expansion a slow, if not unfeasible, solution. An alternative is to
use a finite difference scheme to directly solve the evolution problem
after suitable transformations. However, both of these methods need a
high degree of numerical resolution to produce practically useful
approximations of the transition density. We conjecture that these
inefficiencies come from either using a separable representation for
the differential operator that is intrinsically correlated in the two
dimensions (trigonometric series) or introducing numerical
diffusion (finite difference).

In this paper, we propose a robust and efficient solution to the
general problem (\ref{eq:1}) - (\ref{eq:2}). The solution is obtained
by combining a small-time analytic solution with a Galerkin
discretization based on basis functions that are correlated. Our
method directly takes into account the correlation parameter present
in the differential operator in order to efficiently represent the
analytic small-time solution and propagate it forward in time. We
apply our computational method to estimate equation parameters with a
maximum likelihood approach in settings where the model assumptions of
constant $(\mu_x, \mu_y, \sigma_x, \sigma_y, \rho)$ and Brownian
motion driving stochastic evolution are appropriate. Section
\ref{sec:approximate-sols} outlines some methods we considered,
including the one we used. Section [] includes our numerical
experiments.


\section{Approximate Numerical Solutions} \label{sec:approximate-sols}
Before considering solutions to the full Fokker-Planck equation
(\ref{eq:1}) - (\ref{eq:2}), we simplify the PDE by proposing a
scaling transformation and an exponential decomposition of the
solution, so that we can construct
\begin{align}
  p(x,y,t) &= \exp(\alpha x + \beta y + \gamma t) q\left( x(b_x - a_x) + a_x, \,\, y(b_y - a_y) + a_y, \,\, t \right), \label{eq:q-to-p}
\end{align}
where


\begin{align*}
  \alpha &= -\frac{\mu_x}{\sigma_x^2} - \frac{\rho}{\sigma_x\sigma_y(1-\rho^2)}\left( -\frac{\mu_y}{\sigma_y^2} + \frac{\mu_x \rho}{\sigma_x \sigma_y} \right), \\
  \beta &= \left( -\frac{\mu_y}{\sigma_y^2} + \frac{\mu_x \rho}{\sigma_x \sigma_y} \right), \\
  \gamma &= \frac{1}{2}\left( \frac{\sigma_x}{(b_x-a_x)} \right)^2 \alpha^2 + \frac{1}{2}\left(\frac{\sigma_y}{(b_y-a_y)}\right)^2 \beta^2 + \alpha\beta.
\end{align*}
This new formula satisfies the simpler diffusion equation:
\begin{equation}
  \frac{\partial}{\partial t} p(x,y,t) = \mathcal{L}p(x,y,t),\quad (x,y) \in (0,1) \times (0,1) := \Omega \label{eq:qq}
\end{equation}
subject to the constraints
\begin{align}
  p(x,y,t) &=0, & \mbox{ for } & (x,y) \in \{ x | x = 0\} \cup \{ y | y = 0\}, \nonumber \\
  p(x,y,0) &= \delta\left( x - \frac{x_0-a_x}{b_x - a_x} \right) \delta\left(y-\frac{y_0 - a_y}{b_y - a_y}\right), \nonumber
\end{align}
where the differential operator $\mathcal{L}$ takes the form
\[
  \mathcal{L} = \frac{1}{2} \tau_x^2 \frac{\partial^2}{\partial x^2}
  + \rho\tau_x\tau_y \frac{\partial^2}{\partial x \partial y} + \frac{1}{2}\tau_y^2 \frac{\partial^2}{\partial y^2},
  % \left(  \right)^2
  %                                                   \frac{\partial^2}{\partial x^2}p(x,y,t) + \rho\left( \frac{\sigma_x}{(b_x-a_x)} \right)\left( \frac{\sigma_y}{(b_y-a_y)} \right)
  %                                                   \frac{\partial^2}{\partial x \partial y}p(x,y,t) +
  %                                        \frac{1}{2}\left( \frac{\sigma_y}{(b_y-a_y)} \right)^2 \frac{\partial^2}{\partial y^2}p(x,y,t) , \label{eq:qq} \\
\]
where $\tau_x = \frac{\sigma_x}{(b_x-a_x)}$,
$\tau_y = \frac{\sigma_y}{(b_y-a_y)}$.  Note here that under this
transformation $\rho$ remains the same as in the original problem. We
will call equation (\ref{eq:qq}) the \textit{normalized} problem and
will consider its solution in terms of the diffusion parameters
$(\tau_x, \tau_y, \rho)$ without a loss of generality.

\subsection{Eigenfunction Expansion} \label{sec:eigenfunction}
Following Section 6.2 of \cite{risken1989fokker-planck}, we may use
the biorthogonal decomposition of the solution as a sum of
eigenfunctions and time-dependent coefficients determined by eigenvalues:
\begin{equation}
  p(x,y,t) =  \sum_\nu h_\nu \phi_\nu (x, y) e^{-\lambda_\nu t}, \label{eq:biorthogonal}
\end{equation}
where $h_\nu$ is the coefficient of $\phi_\nu(x, y)$ is the
eigenfunction expansion of $p(x,y,0)$. Because the differential
operator $\mathcal{L}$ is self-adjoint, the family of eigenfunctions
is complete in the Hilbert space $L_2(\Omega)$. Moreover, the
eigenvalues are bounded below by a positive constant $c$, so that the
solution behaves as expected (see section 6.3 of
\cite{risken1989fokker-planck}).

Since we require $\phi_\nu(x,y)$ to be zero on the boundaries, we
approximate the eigenfunction using a finite set of orthogonal basis
functions satisfying boundary conditions, i.e., a finite sequence of
sines,
\[
  \phi_\nu(x,y) = \sum_{l=1}^L \sum_{m=1}^M c_{l,m, \nu}
  \sin\left(2\pi\, l\, x \right) \sin\left(2\pi\, m\, y \right) := \Psi(x,y)^T c_\nu,
\]
where we have truncated the infinite series for some suitably large
$L$ and $M$ and defined
\begin{align*}
  \psi_{l,m}(x,y) &= \sin\left(2\pi\, l\, x \right)
                         \sin\left(2\pi\, m\, y \right), \\
  \Psi(x,y) &= (\psi_{1,1}(x,y), \ldots, \psi_{L,M}(x,y))^T, \\
  c_\nu &= (c_{1,1,\nu}, \ldots, c_{L,M,\nu})^T.
\end{align*}

The biorthogonal representation (\ref{eq:biorthogonal}) leads to the
eigenvalue problem
\begin{equation}
  \mathcal{L} \phi_\nu = -\lambda_\nu \phi_\nu, \label{eq:eigenproblem}
\end{equation}
where $\mathcal{L}$ is the differential operator in the normalized
Fokker-Planck equation. Applying $L$ to the basis function expansion
of $\phi_v $ and again approximating the result using the finite set
of basis functions yields
\[
  \mathcal{L}\phi_\nu = \mathcal{L}(\Psi(x,y)^T c_\nu) =
  \mathcal{L}(\Psi(x,y)^T) c_\nu = (A \Psi(x,y))^T c_\nu,
\] 
where $A$ is a constant matrix dependent on
$\theta = (\mu_x, \mu_y, \sigma_x, \sigma_y, \rho)$. In the last part
of the equation above, we have truncated the infinite sine series
expansion of $L \psi_{l, m}(x, y)$. In the case where $\rho = 0$, $A$
is diagonal because $\left\{ \psi_{l,m}(x,y) \right\}$ are the
eigenfunctions to $\mathcal{L}$. When $\rho \neq 0$, $A$ is no longer
diagonal and is in fact dense. The dense structure of matrix $A$ is
caused by the mixing terms
\[
  \frac{\partial^2}{\partial x \partial y} \sin\left(2\pi\, l\,
    x\right) \sin\left(2\pi\, m\, y\right) = (2\pi\, l)(2\pi\, m)
  \cos\left(2\pi\, l\, x\right) \cos\left(2\pi\, m\, y\right)
\]
being the products of cosine functions, which have an inefficient sine
series representation [CITE]. Substituting the linear representation
of $\mathcal{L}\phi_\nu$ into the eigenvalue problem
(\ref{eq:eigenproblem}), we arrive at the matrix eigenvalue problem
\[
  \Psi(x,y)^T A^T c_\nu = -\lambda_\nu \Psi(x,y)^T c_\nu
  \quad \Leftrightarrow \quad A^T c_\nu = -\lambda_\nu c_\nu
\]
whose solution gives the family of orthonormal eigenfunctions. As
mentioned already, the efficiency of this approach is dependent on the
cost of solving the eigenvalue problem
$A^T c_\nu = -\lambda_\nu c_\nu$.  With all of the eigenpairs
$(c_\nu, \lambda_\nu)$, the approximate solution is then
\[
  p(x,y,t) \approx \Psi(x,y)^T \sum_{\nu} h_\nu c_\nu e^{-\lambda_\nu t}.
\]

\subsubsection{Likelihood Calculation} \label{sec:likelihood-calc}
Recalling equation (\ref{eq:pdf}), to compute the likelihood we are
interested in we must take derivatives of the solution $p(x,y,t)$ with
respect to the boundary parameters $(a_x, b_x, a_y, b_y)$. Because the
eigenvalues and eigenvectors for this problem are functions of the
parameters without explicit analytic form,
$\frac{\partial^4}{\partial a_x\partial b_x \partial a_y \partial
  b_y}p(x,y,t)$ must be computed numerically using a finite difference
approximation. We must therefore find the eigenvalues and
eigenvectors for each of the sixteen perturbed problems
\[
  p(x,y,t | a_x \pm \epsilon, b_x \pm \epsilon, a_y \pm \epsilon, b_y \pm \epsilon).
\]
The $4^{th}$ derivative of $p$ with respect to the 4 boundaries is
approximated as
\begin{eqnarray*} 
& & \frac{\partial^4}{\partial a_x \partial b_x \partial a_y \partial b_y} p(x, y, t) 
\nonumber \\
& & \hspace*{0.5cm} \approx \frac{\displaystyle \sum_{k_1, k_2, k_3, k_4 = \pm 1}
c_{\{ k_1, k_2, k_3, k_4 \} } p(\left. x, y, t \right| a_x+k_1 \varepsilon, b_x+k_2 \varepsilon, 
a_y+k_3 \varepsilon, b_y+k_4 \varepsilon) }{(2 \varepsilon )^4} 
\nonumber
\end{eqnarray*}
Because a good, working approximation to $p(x,y,t)$ requires many
terms in the basis function expansions of the eigenfunctions, and
because each resultant system matrix is dense, repeated computation of
the likelihood in this way is unfeasible.

\subsection{Finite Difference} \label{sec:finite-difference} A finite
difference method which approximates the spatial derivatives in
problem (\ref{eq:qq}) requires the solution a system of differential
equations
\begin{align}
  \dot{c}(t)= A c(t) &\Rightarrow& c(t) = \exp\left( At \right)c(0) \label{eq:eigenproblem-fd}, 
\end{align}
which reduces to the eigenvalue decomposition of a matrix $A$. Here,
$c(t)$ is a vector consisting of values of the solution in
(\ref{eq:qq}) on a set of grid points over $\Omega$ at time $t$, and
the product $Ac(t)$ approximates $\mathcal{L}p(x,y,t)$. The system
matrix $A$ is dependent on the discretization scheme used to
approximate $\mathcal{L}$. Using a central in space scheme over a
\textbf{regular} grid on $\Omega$ with $\Delta x = \Delta y = h$,
\[ A = \frac{1}{2}\tau_x^2 \frac{1}{h^2}A_{x,x} +
  \rho\tau_x\tau_y \frac{1}{4h^2}A_{x,y} + \frac{1}{2}\tau_y^2
  \frac{1}{h^2}A_{y,y},
\]
where each of the matrices $A_{x,x}, A_{x,y}, A_{y,y}$ approximate
$\frac{\partial^2}{\partial x^2}, \frac{\partial^2}{\partial
  x \partial y}, \frac{\partial^2}{\partial y^2}$, respectively. For
example, using $c_{l, m}(t) $ to denote the approximation of the
solution at grid point $(x_l, y_m)$, and assembling vector $c(t)$
using the index scheme $c_{l, m}(t) \rightarrow c_k(t)$ with
$k= (l-1) M + m$, we can approximate the operator
$\frac{\partial^2 }{\partial x^2} $ as
\begin{align*}
  \frac{1}{2}\tau_x^2 \frac{\partial^2}{\partial x^2} p(x_l,y_m,t) \approx \frac{1}{2}\tau_x^2 \left( \frac{c_{l+1,m}(t) - 2c_{l,m}(t) + c_{l-1,m}(t)}{h^2} \right) = \frac{1}{2}\tau_x^2 \frac{1}{h^2}A_{x,x}[k,] c(t) 
\end{align*}
where $A_{x,x}[k,]$ is the $k^{th}$ row of $A_{x,x}$. Similarly,
\begin{align*}
  \rho\tau_x\tau_y \frac{\partial^2}{\partial x \partial y} p(x_l,y_m,t) \approx \rho\tau_x\tau_y \left( \frac{c_{l+1,m+1}(t) - c_{l+1,m-1}(t) - c_{l-1,m+1}(t) + c_{l-1,m-1}(t)}{4h^2} \right) = \rho\tau_x\tau_y \frac{1}{4h^2}A_{x,y}[k,] c(t) 
\end{align*}
It should be noted here that a regular grid approach with a constant
$h$ independent of parameters is appealing, because it allows us to
construct once and store the matrices $A_{x,x}, A_{x,y}, A_{y,y}$,
which saves valuable computational resources if we are to solve the
finite difference eigenproblem (\ref{eq:eigenproblem-fd}) repeatedly
for different parameter values $(\tau_x,\tau_y,\rho)$.

Unlike the system matrix for the trigonometric expansion, the system
matrix $A$ here is sparse: each row of $(A_{x,x}, A_{x,y}, A_{y,y})$
is composed of all zeros except for three or four entries. This
structure does not change as $h \to 0$. The eigenvalue problem is
therefore much cheaper to solve.

However, the fundamental limitation of using a finite difference
method is that differentiation with respect to boundaries is also done
using finite differences and it is of fourth order. Given this high
order differentiation, $h$ cannot be made too small because round-off
error and the irregular part of the truncation error become an issue
relatively quickly. More explicitly, assuming consistency and
stability of the finite difference approximation to $\mathcal{L}$, we
can write down the finite difference solution as
\[
  p_{FD}(\left. x, y, t \right| b) = p(\left. x, y, t \right| b) 
+ h^2 \mbox{F}_{reg}(b) + h^2 \mbox{F}_{irreg}(b) + 
\varepsilon_{mach} \mbox{F}_{round}(b),
\]
where we have included parameter $b $ explicitly as a simplified notation for 
$[a_x, b_x]$ and $[a_y, b_y]$ after shifting $a_x$ and$a_y$ to 0.

Note that when expressed using the chain rule, both 
$\displaystyle \frac{\partial}{\partial a_x}$ and  $\displaystyle \frac{\partial}{\partial b_x}$
contain $\displaystyle \frac{\partial}{\partial b}$. As a result, 
$\displaystyle \frac{\partial^2 }{\partial a_x \partial b_x} $ leads to 
$\displaystyle \frac{\partial^2 }{\partial b^2} $. Although in the discussion below, 
for simplicity, we only illustrate the numerical differentiation on the first derivative, 
keep in mind that it is the second derivative that is more relevant in the calculation of 
 $\displaystyle \frac{\partial^4}{\partial a_x \partial b_x \partial a_y \partial b_y}
p(x, y, t) $. 

In the expression of finite difference solution above, 
$h^2 \mbox{F}_{reg}(b) $ is the regular part of 
the truncation error from discretizing the differential operator on the grid; 
$h^2 \mbox{F}_{irreg}(b) $ is the irregular part of the truncation error 
from the interpolations invoked by $\{x_0, y_0, x, y \}$ not being on grid points; 
and $\varepsilon_{mach} \mbox{F}_{round}(b) $ is the effect of round-off errors
with $\varepsilon_{mach} \sim 10^{-16}$ denoting the machine epsilon for 
IEEE double precision system. 
The coefficient, $\mbox{F}_{reg}(b) $, of the regular part of truncation error 
is a smooth function of $b$ with derivative $= O(1)$. 
The coefficient, $\mbox{F}_{round}(b) $, in the effect of round-off errors, 
behaves virtually like a random variable, discontinuous in $b$. 
For the irregular part of truncation error, the coefficient $\mbox{F}_{irreg}(b) $ 
in general is continuous in $b$ but not smooth in $b$ where the derivative has 
discontinuities of magnitude $O\left( \frac{1}{h} \right)$. 
For example, the error in 
a piece-wise linear interpolation of a smooth function at position $b$ using step $h$ 
has the general form of 
$$\mbox{Interpolation error } = O(h^2) (1-\mbox{rem}(b/h,1)) \mbox{rem}(b/h,1)$$ 
The coefficient part $\mbox{F}_{irreg}(b) = (1-\mbox{rem}(b/h,1)) \mbox{rem}(b/h,1) $ 
is continuous in $b$ but not differentiable. Its first derivative 
has the behavior of 
$$ \frac{\partial }{\partial b} \mbox{F}_{irreg}(b) =  \frac{1}{h} \left(1-2\mbox{rem}(b/h,1) \right)$$

Based on the expression we wrote out above for the finite difference solution, 
applying the numerical differentiation on t with step $\varepsilon$ yields: 
\begin{eqnarray}
& & \frac{p_{FD}(\left. x, y, t \right| b+\varepsilon)-p_{FD}(\left. x, y, t \right| b -\varepsilon)}{2 \varepsilon} \nonumber \\
& & \hspace*{1cm} =  \frac{\partial }{\partial b} p(\left. x, y, t \right| b) + O(\varepsilon^2)
 + h^2 \frac{\mbox{F}_{reg}(b+\varepsilon)-\mbox{F}_{reg}(b-\varepsilon)}{2 \varepsilon}  \nonumber \\
& & \hspace*{1cm} + h^2 \frac{\mbox{F}_{irreg}(b+\varepsilon)-\mbox{F}_{irreg}(b-\varepsilon)}{2 \varepsilon}
 + \varepsilon_{mach} \frac{\mbox{F}_{round}(b+\varepsilon)-\mbox{F}_{round}(b-\varepsilon)}{2 \varepsilon}  \nonumber
\end{eqnarray}
In the equation above, as the step in the numerical differentiation is refined, the first line 
of the RHS is well behaved, converging to the true value 
$\frac{\partial }{\partial b} p(\left. x, y, t \right| b)$ as $\varepsilon \rightarrow 0$. 
The second line of RHS, however, is problematic. As $\varepsilon \rightarrow 0$, 
the contribution from round-off error blows up to infinity
$$ \varepsilon_{mach} \frac{\mbox{F}_{round}(b+\varepsilon)-\mbox{F}_{round}(b-\varepsilon)}{2 \varepsilon}  = O\left(\frac{\varepsilon_{mach}}{\varepsilon} \right) 
\longrightarrow \infty \;\;\;\; \mbox{as } \varepsilon \rightarrow 0 $$
The contribution from the irregular part of truncation error is 
$$ h^2 \frac{\mbox{F}_{irreg}(b+\varepsilon)-\mbox{F}_{irreg}(b-\varepsilon)}{2 \varepsilon}  = 
O\left(\frac{h^2}{\max(\varepsilon, h)} \right) $$ 
In the second order numerical differentiation, however, the contribution from 
the irregular part of truncation error behaves like 
$$ h^2 \frac{\mbox{F}_{irreg}(b+\varepsilon)-2\mbox{F}_{irreg}(b)
+\mbox{F}_{irreg}(b-\varepsilon)}{\varepsilon^2 }  = 
O\left(\frac{h^2}{\max(\varepsilon^2, h^2)} \right) $$

\section{A Novel Semidiscrete Galerkin
  Method} \label{sec:semidiscrete-galerkin} We propose a semidiscrete
Galerkin (continuous in time, discrete in space) solution to the
general diffusion problem (\ref{eq:qq}). The method relies on an
analytic approximation for the solution $p(x,y,t)$ for small time $t$
and a basic convergence estimate from approximation theory for
semidiscrete Galerkin-type solutions to parabolic problems.

Our numerical method produces a functional representation for the
approximate solution which $1)$ imposes a computational burden
comparable to or better than that of the finite difference method and
$2)$ is infinitely differentiable with respect to the boundary
parameters, allowing us to perform the crucial numerical
differentiation with respect to these parameters.

As described in Section \ref{sec:eigenfunction}, the solution to the
model problem has the eigenfunction expansion 
\[
  p(x,y,t) = \sum_{\nu=0}^\infty\phi_\nu(x,y) e^{-\lambda_\nu t}.
\]
Proceeding with the standandard Galerkin approach, we propose a
solution $p^{(k)}(x,y,t)$ of similar form
\[
  p^{(k)}(x,y,t) = \sum_{i=0}^k c_i(t) \psi_i(x,y),
\]
where the basis functions $\psi_i(x,y)$ satisfy the boundary
conditions. We also require that all first- and second-order
derivatives of $\psi_i(x,y)$ are in $L_2(\Omega)$, i.e.
$\psi_k(x,y) \in W_{2}^{2}(\Omega)$. Since $p^{(k)}$ is an
approximation to the solution $p$, it does not follow the differential
equation exactly nor can it represent the initial condition fully. We
capture this by defining residuals
\begin{align*}
  \frac{\partial}{\partial t} p^{(k)}(x,y,t) - \mathcal{L}p^{(k)}(x,y,t) &:= R_e(k), \\
  p(x,y,0) - p^{(k)}(x,y,0) &:= R_0(k).
\end{align*}
There are various conditions that could be imposed on the residual
functions (see Section 2.10.3 of \cite{norrie1973finite} for a
summary). The \textit{orthogonality} condition coincides with the
Galerkin procedure:
\begin{align}
  \displaystyle \int_{\Omega} R_e(k) \psi_i(x,y) dx\,dy &= 0,& \displaystyle \int_{\Omega} R_0(k) \psi_i(x,y) dx\,dy &= 0,& i = 0,\ldots k, \label{eq:orthogonality-conditions}
\end{align}
which is equivalent to the weak formulation of the heat problem
\begin{align*}
  \left< \partial_t p^{(k)}(x,y,t), \psi_i \right> &= \left<\mathcal{L}p^{(k)}(x,y,t), \psi_i \right>, \\
  \left< p^{(k)}(x,y,0), \psi_i \right> &= \left<p(x,y,0), \psi_i\right>,
\end{align*}
where $<\cdot, \cdot>$ is the usual inner product in
$L_2(\Omega)$. The orthogonality conditions
(\ref{eq:orthogonality-conditions}) lead to the system of equations
\begin{align}
  M \mathbf{\dot{c}}(t) &= S \mathbf{c}(t), \label{eq:orthogonality-conditions-mat-1}\\
  M \mathbf{c}(0) &= \mathbf{p}(0), \label{eq:orthogonality-conditions-mat-2}
\end{align}
where $M$ is the mass matrix, $S$ is the stiffness matrix, and
$\mathbf{p}(0)$ is the vector projection of $p(x,y,0)$ onto the
span of $S_k$ with elements
\begin{align*}
  [M]_{ij} &= \displaystyle \int_\Omega \psi_i \psi_j dx\,dy, \\
  [S]_{ij} &= -\frac{1}{2}\tau_x^2 \displaystyle \int_\Omega \left( \frac{\partial}{\partial x} \psi_i(x,y) \right) \left( \frac{\partial}{\partial x} \psi_j(x,y) \right) dx\,dy -\rho\tau_x\tau_y \displaystyle \int_\Omega \left( \frac{\partial}{\partial x} \psi_i(x,y) \right) \left( \frac{\partial}{\partial y} \psi_j(x,y) \right) dx\,dy \\
         &\quad -\frac{1}{2}\tau_y^2 \displaystyle \int_\Omega \left( \frac{\partial}{\partial y} \psi_i(x,y) \right) \left( \frac{\partial}{\partial y} \psi_j(x,y) \right) dx\,dy \\
  [\mathbf{p}(0)]_i &= \displaystyle \int_\Omega p(x,y,0) \psi_i(x,y) dx\,dy.
\end{align*}
The entries of $S_{ij}$ are computed with integration by parts and
using the boundary conditions. The semidiscrete Galerkin approximation
becomes
\[
  p^{(k)}(x,y,t) = \boldsymbol{\psi}(x,y)^T \exp\left( M^{-1}S\, t \right) \mathbf{c}(0),
\]
with $\boldsymbol{\psi}(x,y) = (\psi_0(x,y), \ldots, \psi_k(x,y))^T.$

\subsection{Small-time Solution}
Before solving the Galerkin equations
(\ref{eq:orthogonality-conditions-mat-1}) -
(\ref{eq:orthogonality-conditions-mat-2}), we develop a small-time
analytic solution to the problem. The small-time solution is derived
by considering the fundamental solution $G(x,y,t)$ for the unbounded
problem in (\ref{eq:qq}), which is the bivariate Gaussian density with
mean and covariance determined by the initial condition and the
diffusion parameters \citep{stakgold2011green}. We can then find a
small enough $t_\epsilon$ such that
$G\left(x,y, t_\epsilon \left| \frac{x_0-a_x}{b_x - a_x},
    \frac{y_0-a_y}{b_y - a_y} \right.\right)$ is numerically zero on
the boundaries of $\Omega$. This is done in the following way.
\begin{enumerate}[1.]
\item Scale and rotate the coordinate axes by
  \[
    \left( \begin{array}{c}
             \xi \\
             \eta
           \end{array} \right) = \mbox{Transformation goes here}
             \left( \begin{array}{cc}
             x \\
             y
           \end{array} \right) 
       \]
       so that the problem obeys the diffusion equation. The
       computational domain is transformed to the parallelogram
       $\tilde{\Omega}$. The transformed initial condition will be
       denoted as $(\xi_0, \eta_0)$.

     \item The fundamental solution $G(\xi,\eta,t | \xi_0, \eta_0)$ in
       this coordinate frame which does not take into account
       boundaries follows the bivariate Gaussian probability density
       function
  \[
    G(\xi,\eta,t | \xi_0, \eta_0) = \frac{1}{2\pi t} \exp\left(-\frac{(\xi-\xi_0)^2 +
        (\eta-\eta_0)^2}{2\,t} \right).
  \]
  We define \textbf{distance} between $G(\xi,\eta,t | \xi_0, \eta_0)$
  and any of the linear boundaries of $\tilde{\Omega}$ as the
  Euclidean norm of the segment of the vector extending from
  $(\xi_0,\eta_0)$ and normal to the boundary. There are four such
  distances $d_1, d_2, d_3, d_4$. Assume that they are listed in
  increasing magnitude.

  Setting $t_\epsilon = d_2/8$ ensures that the fundamental solution
  $G(\xi,\eta,t_\epsilon | \xi_0, \eta_0)$ is \textit{at most}
  $\approx 10^{-15}$ on the second-farthest boundary, as well as the
  other two farthest boundaries. In this way,
  $G(\xi,\eta,t_\epsilon | xi_0, \eta_0)$ satisfies the boundary
  condition on the three farthest boundaries numerically.

\item Reflect the point $(\xi_0, \eta_0) \to (\xi_0', \eta_0')$ about
  the closest boundary. The image function
  $G(\xi,\eta,t_\epsilon | \xi_0', \eta_0')$ satisfies the diffusion
  equation and is equal to $G(\xi,\eta,t_\epsilon | \xi_0, \eta_0)$ on
  the closest boundary. Further,
  $G(\xi,\eta,t_\epsilon | \xi_0', \eta_0')$ takes on values less than
  $10^{-15}$ on all other boundaries, because it is outside of
  $\bar{\Omega}$.

  Considering the difference of the two images, the small-time solution 
  \[
    p(\xi,\eta,t_\epsilon) := G(\xi,\eta,t_\epsilon | \xi_0, \eta_0) -
    G(\xi,\eta,t_\epsilon | \xi_0', \eta_0')
  \]
  satisfies all of the boundaries numerically and also satisfies the
  governing diffusion equation. Performing a change of variables
  produces the small-time solution $p(x,y,t_\epsilon)$.
\end{enumerate}
Using Theorem 5.E of \cite{zeidler1995applied}, we can solve for
$p(x,y,t)$ by considering the smooth $p(x,y,t_\epsilon)$ as an initial
condition and evolving it forward in time by $t-t_\epsilon$. This
replaces initial condition orthogonality condition in
(\ref{eq:orthogonality-conditions-mat-2}) with
\begin{align}
  M \mathbf{c}(t_\epsilon) &= \mathbf{p}(t_\epsilon), \\
   [\mathbf{p}(t_\epsilon)]_i &= \displaystyle \int_\Omega p(x,y,t_\epsilon) \psi_i(x,y) dx\,dy. \nonumber
\end{align}
Intuitively, the bigger $t_\epsilon$, the smaller both $R_e(k)$ and
$R_0(k)$ will be, and thef better our method will do.
% less eigenmodes present in $p(x,y,t_\epsilon)$, and the more accurate
% our weak solution according to the error estimate
% \[
%   \| e(t) \|_{L_2(\Omega)} \leq C h(k)^2 \| p(x,y,t_\epsilon) \|_{2}.
% \]


\subsection{Orthonormal Basis Family}
% An upper bound of the rate at which the Galerkin approximation
% converges to $p(x,y,t)$ is given by condition (\ref{eq:ic-bound}),
% namely by how well the initial condition may be approximated via a
% projection onto $S_k$.
We motivate the construction of the orthonormal basis functions by
once again considering the fundamental solution for the unbounded
problem (\ref{eq:qq}). In the absence of boundaries, (\ref{eq:qq}) is
solved by the function
\[
  G(x,y,t | x_0', y_0') = \frac{1}{2\pi\,\,t\,\, \tau_x\tau_y\sqrt{1-\rho^2}} \exp\left\{ -\frac{1}{2\,t(1-\rho^2)} \left( \frac{(x - x_0')^2}{\tau_x^2} - 2\rho \frac{(x-x_0')(y-y_0')}{\tau_x\tau_y} + \frac{(y - y_0')^2}{\tau_y^2}\right) \right\},
\]
with
$x_0' = (x_0 - a_x)/(b_x-a_x),\quad y_0' = (y_0 - a_y)/(b_y-a_y)$. We
choose the family of basis functions
$S_k = (\psi_0(x,y), \ldots, \psi_k(x,y))$
\begin{align}
  \psi_i(x,y) &= \frac{1}{2\pi \sigma^2\sqrt{1-\tilde{\rho}^2} } \exp\left\{ -\frac{1}{2(1-\tilde{\rho}^2)\sigma^2} \left( (x - x_i)^2 - 2\tilde{\rho} (x-x_i)(y-y_i) + (y - y_i)^2 \right) \right\} x\left(1-x\right)\, y(1-y)
\end{align}
for some parameters $(\tilde{\rho}, \sigma)$ and a collection of
points $\{ (x_i,y_i) \}_{i=0}^k$ which form a grid over $\Omega$.

The critical advantage of these elements, however, is that they better
resolve the small-time solution $p(x,y,t_\epsilon)$ by taking into
account some degree of correlation $\tilde{\rho}$ in the covariance of
each kernel.  Essentially, the collection
$\{ \psi_i(x,y| x_i, y_i, \tilde{\rho}, \sigma) \}_{i=0}^k$ is
composed of fundamental solutions to a heat diffusion problem tuned by
$\sigma$ and $\tilde{\rho}$, tampered such that their support is on
$\Omega$, zero on the boundaries, still smooth, and inheriting the
correlation structure of the fundamental solution to the
problem. % In this manner,
% our basis function choice is in keeping with the golden rule for the
% rate of convergence: \textbf{The smoother the solution to the original
%   problem and the smoother the functions in the basis space, the
%   faster the convergence of the Ritz-Galerkin method.} (Remark 1(c) in
% Chapter 2 of \cite{zeidler1995applied}).
Moreover, all of the basis functions are infinitely differentiable
with respect to the boundary and diffusion parameters, so that
$\frac{\partial^4 p^{(k)}(x,y,t)}{\partial a_x \partial b_x \partial
  a_y \partial b_y}$ exists and can be computed numerically using
finite differences.

\subsection{Error Bound} \label{sec:error-bound}
A bound on the closeness of the approximate solution $p^{(k)}(x,y,t)$
to the strong solution $p(x,y,t)$ is developed in
\cite{bramble1977some}.  Their result shows that the Galerkin
approximation we use converges to the strong solution in
$L_2(\Omega)$, and it motivates the thrust of our numerical
solution. First, we define the \textit{error} term
\[
  e^{(k)}(t) = p(x,y,t) - p^{(k)}(x,y,t),
\]
as well as the norm
\[
  \| w \|_2 = \sum_{j=0}^\infty \lambda_j^2 \left<w, \phi_j\right>
\]
for the eigenpairs $(\lambda_j, \phi_j)$ of the operator
$\mathcal{L}$. As referred to in \cite{bramble1977some}, functions
$w \in L_2(\Omega)$ with $\|w\|_2 < \infty$ are also in
$W_2^2(\Omega)$. Finally, if we have the condition (corresponding to
equation 2.1 in \cite{bramble1977some})
\begin{align}
  \| p(x,y,t_\epsilon) - p^{(k)}(x,y,t_\epsilon) \|_{L_2(\Omega)} &\leq C\, h(k)^2 \| p(x,y,t_\epsilon) \|_2 \label{eq:ic-bound}
\end{align}
as a decreasing function $h(k) \geq 0$, Theorem 2.1 in
\cite{bramble1977some} applies and we have the error estimate
\begin{align}
  \| e^{(k)}(t) \|_{L_2(\Omega)} \leq C h(k)^2 \| p(x,y,t_\epsilon) \|_{2}. \label{eq:error-est}
\end{align}
We can ensure condition (\ref{eq:ic-bound}) is met if $S_k$ is
complete in $L_2(\Omega)$ as $k$ grows. The other two conditions
necessary for the error bound to apply are demonstrated by
\cite{bramble1977some} for the Galerkin method. Equation
(\ref{eq:error-est}) can be summarized in a simple way: the
\textit{error} of the method is controlled by how much variation the
small-time solution has; the \textit{rate} of decrease of the error is
controlled by how well the span of $S_k$ represents the small-time
solution compared to the variation of the initial condition as $k$
increases..

In the context of (\ref{eq:error-est}), our method, with its small-time
analytic solution and choice of basis functions, is specifically
tailored to minimize the error between the strong solution and its
Galerkin approximation under $L_2(\Omega)$.

% However, the initial condition for (\ref{eq:qq}) requires all in
% eigenmodes being included in the representation of the initial
% conditions, so that
% \[
%   \| p(x,y,0) \|_{2} = \left\| \delta\left( x - \frac{x_0 - a_x}{b_x-a_x}
%   \right) \delta\left( y - \frac{y_0 - a_y}{b_y-a_y} \right)\right\|_{2} = +\infty.
% \]
% It is obvious that we cannot apply the error estimates above. However,
% evolving the solution forward in time, even for a short period,
% diffuses the delta-function IC and attenuates out the highest
% frequency modes. Further, because solutions to the diffusion equation
% are smooth, $p(x,y,t_\epsilon)$ will give us a smooth initial function
% which will make the error bounds admissible.


\section{Estimation}
Consider the problem of estimating the parameters
$(\mu_x, \mu_y, \sigma_x, \sigma_y, \rho)$ from an i.i.d. sample
$(Z_1, \ldots, Z_n)$ from the process in (\ref{eq:X}) - (\ref{eq:Y}),
where each $Z_i$ is the vector of random variables
\[
  Z_i = (X_i(t), Y_i(t), m_{x,i}, M_{x,i}, m_{y,i}, M_{y,i}).
\]
We say that $Z_i$ is sampled from the distribution corresponding to
the probability density function (\ref{eq:pdf})
\[
  Z_i \sim F(\theta),
\]
where the distribution $F$ has the usual interpretation
\begin{align*}
  F(z = (x, y, a_x, b_x, a_y, b_y) | \theta) &= \Pr\left(X(t) \leq x,
    Y(t) \leq y, m_x \leq a_x, M_x \leq b_x, m_y \leq a_y, M_y \leq b_y\right).
\end{align*}

This estimation problem is of particular importance in quantitative
finance where the model equations (\ref{eq:X}) - (\ref{eq:Y}) (with
various bells and whistles attached) are widely used. However, to the
best of our knowledge, all current \textit{likelihood} methods in the
literature either ignore the observed maximum/minimum information or
use some of it. Likelihood-free approaches, like that of
\cite{rogers1991estimating}, on the other hand suffer from not being
able to be easily integrated into a bigger inferential framework.

Since we do not have a closed-form solution for the likelihood, we
will use an iterative derivative-free maximization algorithm (the
Nelder-Mead method; see \cite{lagarias1998convergence} for review and
convergence properties) which requires repeated evauluation of the
likelihood. For moderate to large samples sizes this is feasible,
because our numerical method is specifically designed for
computational efficiency for repeatedly evaluating the density
function (\ref{eq:pdf}). The maximum likelihood estimator (MLE) for
the true parameters based on $n$ samples, which we call
$\hat{\theta}_n := (\hat{\mu}_x, \hat{\mu}_y, \hat{\sigma}_x,
\hat{\sigma}_y, \hat{\rho})$, is especially useful in practical
settings when it exhibits \textit{consistency}, \textit{i.e.}: the MLE
gets closer to the true parameter vector $\theta$ as more data is
collected and included in the likelihood (assuming the model and its
parameters remain constant during the data collection). More
precise, the estimator $\hat{\theta}_n$ is consistent if
\[
  \Pr( | \hat{\theta}_n - \theta | ) \to 0 \qquad \mbox{as} \qquad n \to \infty.
\]

\subsection{Consistency}
In this section we prove that the MLE based on the Galerkin
approximation $p^{(k)}(x,y,t)$ to the governing Fokker-Planck equation
(\ref{eq:1}) is consistent. To do so, we first show that the
distribution on $Z$ based on the approximate $p^{(k)}(x,y,t)$ converges to
the true distribution $F(\cdot | \theta)$. Define the probability densities
\begin{align}
  f(z) &= f(x,y,a_x,b_x,a_y,b_y) := \Pr\left(X(t) \in dx, Y(t) \in dy, m_x \in da_x,
         M_x \in db_x, m_y \in da_y, M_y \in db_y \left| \theta \right.\right) \label{eq:stong-density} \\
    f^{(k)}(z) &= f^{(k)}(x,y,a_x,b_x,a_y,b_y) := \frac{\partial^4}{\partial a_x \partial b_x \partial a_y \partial
         b_y} q^{(k)}(x,y,t | a_x, b_x, a_y, b_y). \label{eq:galerkin-density}
\end{align}
The function $f(z)$ is the probability density function corresponding
to $q(x,y,t)$, the strong solution to the governing Fokker-Planck
equation; $f^{(k)}(z)$ corresponds to the Galerkin approximation
$q^{(k)}$. Before proceeding, we should note that both $f(z)$ and
$f^{(k)}(z)$ exist. So see the case for $f(z)$, consider the relation
\begin{align*}
  &f(x,y,a_x,b_x,a_y,b_y) = \\
  &\Pr\left(X(t) \in dx, Y(t) \in dy, m_x \in da_x, M_x \in db_x, m_y
  \in a_y, M_y \in db_y \left| \theta \right.\right) = \\
  &\mathbb{P}_{W}\left( \underbrace{\left\{ \omega \in W \left| X_\omega(t) = x,
  Y_\omega(t)=y, \inf_{t'\in [0,1]} X_\omega(t') = a_x,
  \sup_{t'\in [0,1]} X_\omega(t') = b_x, \inf_{t'\in [0,1]}
        Y_\omega(t') = a_y, \sup_{t'\in [0,1]} Y_\omega(t') = b_y
      \right. \right\}}_{A(x,y,a_x,b_x,a_y,b_y) = A(z)}\right)
\end{align*}
where $\mathbb{P}_{W}$ is the Wiener measure on the sample space $W$
of all realizations (paths) $(X_\omega(t), Y_\omega(t))$ from the
stochastic process (\ref{eq:X}) - (\ref{eq:Y}) defined in the usal way
using Kolmogorov's extension of measure over cylinder sets on
$t \to \mathbb{R}^2$ (see \cite{freidlin1985functional}, Section
1.2). Sets of the form $A(x,y,a_x,b_x,a_y,b_y)$ can be defined as a
countable intersection/union of cyliner sets on $t \to \mathbb{R}^2$,
hence they are measurable under $\mathbb{P}_{W}$; therefore $f(z)$ exists and
is bounded above by 1. Moreover, since $q(x,y,t|a_x,b_x,a_y,b_y)$ represents the
integral of $f(z)$ with respect to the boundary variables, we have
\[
  f(x,y,a_x,b_x,a_y,b_y) = \frac{\partial^4}{\partial a_x \partial b_x \partial a_y \partial
         b_y} q(x,y,t | a_x, b_x, a_y, b_y).
\]
The existence of $f^{(k)}(z)$ as defined in
(\ref{eq:galerkin-density}) is guaranteed by Theorem 4.1 in
\cite{singler2008differentiability}. The result states that weak
solutions to parabolic problems are differentiable with respect to
parameters as long as the weak (Galerkin-form) operator
$\left< \mathcal{L} \psi_i(x,y), \psi_j(x,y) \right>$ is
differentiable with respect to the parameters. The condition certainly
holds for the normalized problem (\ref{eq:qq}) with our choice of
basis functions $S_k$, as they are infinitely differentiable with
repsect to $x,y$, the boundaries, and the diffusion parameters. 

At this point we will assume that for a sufficiently large $k$,
$f^{(k)}(z)$ is positivie for all $z \in Z$ and is integrable over
$Z$. As such, we may regard it as a proper probability density
function with the cumulative probability density and probability
measure over $Z$ being 
\begin{align}
  F^{(k)}(z | \theta) &= \displaystyle \int_{-\infty}^{a_x} \displaystyle \int_{-\infty}^{a_y} \displaystyle \int_{-\infty}^{b_x} \displaystyle \int_{-\infty}^{b_y} \displaystyle \int_{-\infty}^x \displaystyle \int_{-\infty}^y \frac{\partial^4}{\partial a'_x \partial b'_x \partial a'_y \partial b'_y} q^{(k)}(x', y', t | a'_x, b'_x, a'_y, b'_y)\,\,\, dx'\, dy'\, da'_x\, db'_x\, da'_y\, db'_y, \label{eq:approx-measure} \\
                          \Pr_{k}(A) &:= \displaystyle \int_{A} f^{(k)}(z)\, dz, \quad \mbox{ for any measurable } A \subset Z. \label{eq:approx-measure-2}
\end{align}
We will prove that for every $z \in Z$,
\[
  \lim_{k\to \infty} F^{(k)}(z | \theta) = F(z | \theta).
\]
First, we prove the Lemma
\begin{lemma}\label{lem:1}
  For $z = (x, y, a_x, b_x, a_y, b_y)$,
  \[
    \lim_{k\to \infty} \displaystyle \int_{a_x}^{b_x} \displaystyle
    \int_{a_y}^{b_y} f^{(k)}(x,y,a_x,b_x,a_y,b_y)\, dx\,dy =
    \displaystyle \int_{a_x}^{b_x} \displaystyle \int_{a_y}^{b_y}
    f(x,y,a_x,b_x,a_y,b_y)\, dx\,dy.
  \]
\end{lemma}
\begin{proof}
  Define the sets of form, for $z = (x,y,a_x,b_x,a_y,b_y)$,
  \begin{align*}
    B(a_x, b_x, a_y, b_y) &= \left\{ z' \in Z \left| z' \in [a_x, b_x]
        \times [a_y, b_y] \times [a_x, \infty) \times (-\infty, b_x]
                            \times [a_y, \infty) \times (-\infty, b_y] \right.\right\}.
  \end{align*}
  Elements within $B(a_x, b_x, a_y, b_y)$ are equivalent to sample paths that stay within the region $[a_x, b_x] \times [a_y, b_y]$:
  \begin{align*}
    B(a_x, b_x, a_y, b_y) &= \left\{ \omega \in W | X_\omega(t) \in [a_x, b_x], Y_\omega(t) \in [a_y, b_y], m_x \in [a_x,b_x), M_x \in (a_x, b_x] \right\}
  \end{align*}

  Then
  \begin{align}
    \Pr(B(a_x, b_x, a_y, b_y)) &= \displaystyle \int_{B(a_x, b_x, a_y, b_y)} f(z)\, dz \nonumber \\ 
                               &= \displaystyle \int_{a_x}^{\infty} \displaystyle \int_{-\infty}^{b_x} \displaystyle \int_{a_y}^{\infty} \displaystyle \int_{-\infty}^{b_y} \displaystyle \int_{-\infty}^{b_x} \displaystyle \int_{-\infty}^{b_y} f(x', y', a'_x, b'_x, a'_y, b'_y) dx' dy' da_x' db_x' da_y' db_y' \label{eq:full-form} \\
                               &= \displaystyle \int_{a_x}^{b_x} \displaystyle \int_{a_y}^{b_y} q(x,y,a_x,b_x,a_y,b_y)\, dx\, dy, \nonumber
  \end{align}
  
  where the last equality employs the interpretation of $q$ in
  (\ref{eq:CDF}) and we can freely change the order of integration as
  $f$ is bounded for all $z \in Z$. Similarly,
  \begin{align*}
    \Pr_k(B(a_x, b_x, a_y, b_y)) &= \displaystyle \int_{a_x}^{b_x} \displaystyle \int_{a_y}^{b_y} q^{(k)}(x,y,a_x,b_x,a_y,b_y)\, dx\, dy.
  \end{align*}
  From (\ref{eq:full-form}), the partial derivative of the probability
  $\Pr(B(a_x,b_x,a_y,b_y))$ is well-defined and can be written as
  \begin{align*}
    \frac{\partial^4}{\partial a_x \partial b_x \partial a_y \partial b_y} \Pr(B(a_x, b_x, a_y, b_y)) &= \displaystyle \int_{a_x}^{b_x} \displaystyle \int_{a_y}^{b_y} f(x,y,a_x,b_x,a_y,b_y)\, dx\, dy.
  \end{align*}
  The first-order \textbf{finite difference} approximation the above
  expression can be expressed as a linear combination of probabilities
  of perturbed sets $B(\cdot)$:
  \begin{align*}
    \lim_{\epsilon \to 0}\quad \frac{1}{\epsilon^4} \sum_{i=1}^{16} c(i) \Pr(B(a_x + k_1(i)\epsilon, b_x + k_2(i)\epsilon, a_y + k_3(i)\epsilon, b_y + k_4(i)\epsilon) = \frac{\partial^4}{\partial a_x \partial b_x \partial a_y \partial b_y} \Pr(B(a_x, b_x, a_y, b_y)),
  \end{align*}
  where $c$ and $k$ are functions from $i$ to the appropriate
  coefficients in the first-order finite difference approximation
  $c(i) \to \left\{-1, 1\right\}$, $k_j(i) \to \{-1,1\}$. Using Big-O
  notation, for a sufficiently small $\epsilon$
  \begin{equation}
    \sum_{i=1}^{16} c(i) \Pr(B(a_x + k_1(i)\epsilon, b_x +
    k_2(i)\epsilon, a_y + k_3(i)\epsilon, b_y + k_4(i)\epsilon) =
    \epsilon^4 \frac{\partial^4}{\partial a_x \partial b_x \partial
      a_y \partial b_y} \Pr(B(a_x, b_x, a_y, b_y)) + O(\epsilon^5 ;
    a_x, b_x, a_y, b_y). \label{eq:fin-diff}
  \end{equation}

  The convergence result in Section \ref{sec:error-bound} implies that
  \begin{align*}
    \Pr_k(B(a_x + k_1(i)\epsilon, b_x + k_2(i)\epsilon, a_y +
    k_3(i)\epsilon, b_y + k_4(i)\epsilon) = \displaystyle \int_{a_x +
    k_1(i)}^{b_x + k_2(i)} \displaystyle \int_{a_y + k_3(i)}^{b_y +
    k_4(i)} q^{(k)}(x,y,a_x,b_x,a_y,b_y)\, dx\, dy \to \\
    \displaystyle
    \int_{a_x + k_1(i)}^{b_x + k_2(i)} \displaystyle \int_{a_y +
    k_3(i)}^{b_y + k_4(i)} q(x,y,a_x,b_x,a_y,b_y)\, dx\, dy =
    \Pr(B(a_x + k_1(i)\epsilon, b_x + k_2(i)\epsilon, a_y +
    k_3(i)\epsilon, b_y + k_4(i)\epsilon) \\
    \mbox{ as } k \to \infty \mbox{ in } L_2(\Omega).
  \end{align*}
  Hence, for a sufficiently large $k$ dependent on the supremum over
  $i$, and given the error estimate in (\ref{eq:error-est}), we have
  the relation
  \begin{align*}
    \Pr_k(B(a_x + k_1(i)\epsilon, b_x + k_2(i)\epsilon, a_y +
    k_3(i)\epsilon, b_y + k_4(i)\epsilon) &= \Pr(B(a_x + k_1(i)\epsilon, b_x + k_2(i)\epsilon, a_y +
                                            k_3(i)\epsilon, b_y + k_4(i)\epsilon) \\
    & + O(h(k)^2; a_x + k_1(i)\epsilon, b_x + k_2(i)\epsilon, a_y + k_3(i)\epsilon, b_y + k_4(i)\epsilon)
  \end{align*}
  Note here that the dominating terms $O(h(k)^2; \cdot)$ are
  differentiable with respect to the boundary parameters
  $(a_x, b_x, a_y, b_y)$ since $q$ and $q^{(k)}$ have this
  property. Therefore, if we replace $\Pr(\cdot)$ in
  (\ref{eq:fin-diff}) with $\Pr_k(\cdot)$
  \begin{align*}
    \sum_{i=1}^{16} c(i) \Pr_k(B(a_x + k_1(i)\epsilon, b_x +
    k_2(i)\epsilon, a_y + k_3(i)\epsilon, b_y + k_4(i)\epsilon) &= \sum_{i=1}^{16} c(i) \Pr(B(a_x + k_1(i)\epsilon, b_x +
                                                                  k_2(i)\epsilon, a_y + k_3(i)\epsilon, b_y + k_4(i)\epsilon)  \\
                                                                & + \sum_{i=1}^{16} c(i) O(h(k)^2; a_x + k_1(i)\epsilon, b_x +
                                                                  k_2(i)\epsilon, a_y + k_3(i)\epsilon, b_y + k_4(i)\epsilon) \\
    &=     \epsilon^4 \frac{\partial^4}{\partial a_x \partial b_x \partial
      a_y \partial b_y} \Pr(B(a_x, b_x, a_y, b_y)) + O(\epsilon^5 ;
      a_x, b_x, a_y, b_y) \\
                                                                & + \sum_{i=1}^{16} c(i) O(h(k)^2; a_x + k_1(i)\epsilon, b_x +
                                                                  k_2(i)\epsilon, a_y + k_3(i)\epsilon, b_y + k_4(i)\epsilon).
  \end{align*}
  Dividing both sides by $\epsilon^4$ produces
  \begin{align*}
    \frac{1}{\epsilon^4} \sum_{i=1}^{16} c(i) \Pr_k(B(a_x + k_1(i)\epsilon, b_x +
    k_2(i)\epsilon, a_y + k_3(i)\epsilon, b_y + k_4(i)\epsilon) &= \frac{\partial^4}{\partial a_x \partial b_x \partial
                                                                  a_y \partial b_y} \Pr(B(a_x, b_x, a_y, b_y)) + O(\epsilon;
                                                                  a_x, b_x, a_y, b_y) \\
                                                                &+ \frac{1}{\epsilon^4}\sum_{i=1}^{16} c(i) O(h(k)^2; a_x + k_1(i)\epsilon, b_x +
                                                                  k_2(i)\epsilon, a_y + k_3(i)\epsilon, b_y + k_4(i)\epsilon) \\
    % = \frac{\partial^4}{\partial a_x \partial b_x \partial
    % a_y \partial b_y} \Pr(B(a_x, b_x, a_y, b_y)) + O(\epsilon;
    % a_x, b_x, a_y, b_y) + \frac{\partial^4}{\partial a_x \partial b_x \partial
    % a_y \partial b_y} O(h(k)^2 ; a_x, b_x, a_y, b_y)
  \end{align*}
  As mentioned above $O(h(k)^2; \cdot)$ is differentiable with respect
  to the boundary parameters, so that the right-most term is still
  $O(h(k)^2)$ as $\epsilon \to 0$. Taking the limit in $\epsilon$, we have
  \begin{align*}
    \displaystyle \int_{a_x}^{b_x} \displaystyle \int_{a_y}^{b_y}
    f^{(k)}(x,y,a_x,b_x,a_y,b_y)\, dx\, dy =
    \frac{\partial^4}{\partial a_x \partial b_x \partial a_y \partial
    b_y} \Pr_k(B(a_x, b_x, a_y, b_y)) \\
    = \frac{\partial^4}{\partial
      a_x \partial b_x \partial a_y \partial b_y} \Pr(B(a_x, b_x, a_y,
    b_y)) +  O(h(k)^2; a_x, b_x, a_y, b_y) \\
    =\displaystyle \int_{a_x}^{b_x} \displaystyle \int_{a_y}^{b_y}
    f(x,y,a_x,b_x,a_y,b_y)\, dx\, dy + O(h(k)^2; a_x, b_x, a_y, b_y).
  \end{align*}
  Therefore, we have the desired result:
  \[
    \lim_{k\to \infty} \displaystyle \int_{a_x}^{b_x} \displaystyle
    \int_{a_y}^{b_y} f^{(k)}(x,y,a_x,b_x,a_y,b_y)\, dx\,dy =
    \displaystyle \int_{a_x}^{b_x} \displaystyle \int_{a_y}^{b_y}
    f(x,y,a_x,b_x,a_y,b_y)\, dx\,dy.
  \]
\end{proof}

\begin{lemma}[Convergence in distribution] \label{lem:conv-dist}
  For any $z \in Z$,
  $ \lim_{k \to \infty} F^{(k)}(z | \theta) = F(z).$

\end{lemma}

\begin{proof}
  Let
  $I^{(k)}(z) = \displaystyle \int_{a_x}^{x} \displaystyle
  \int_{a_y}^{y} f^{(k)}(u,v,a_x,b_x,a_y,b_y)\, du\,dv$ and let
  $I(z) = \displaystyle \int_{a_x}^{x} \displaystyle \int_{a_y}^{y}
  f(u,v,a_x,b_x,a_y,b_y)\, du\,dv$. It is possible to show that
  $\lim_{k\to \infty} I^{(k)}(z) = I(z)$ as a consequence of Lemma
  \ref{lem:1} by considering some $\chi_k(z) \in S_k$ approximating
  the indicator $1(u \leq x, v \leq y)$ as $k \to \infty$ and setting up a triangle
  inequality. However, we will omit this technical detail here.

  Next, we know that $I(z)$ is integrable over $(a_x, b_x, a_y, b_y)$
  as $\Pr(Z) = \mathbb{P}_{W}(W) = 1$. The Dominated Convergence
  Theorem applies, and we therefore have
  \[
    \lim_{k \to \infty} \displaystyle \int_{-\infty}^{a_x} \displaystyle \int_{-\infty}^{b_x} \displaystyle \int_{-\infty}^{a_y} \displaystyle \int_{-\infty}^{b_y} I^{(k)}(z) da_x' db_x' da_y' db_y' = \displaystyle \int_{-\infty}^{a_x} \displaystyle \int_{-\infty}^{b_x} \displaystyle \int_{-\infty}^{a_y} \displaystyle \int_{-\infty}^{b_y} I(z) da_x' db_x' da_y' db_y',
  \]
  which implies the result of the Lemma.
\end{proof}
% \begin{align*}
%   q(x,y,t) = \\
%   \Pr\left(X(t) \in dx, Y(t) \in dy,  \min_{t'}X(t') \geq a_x,
%   \max_{t'}X(t')\leq b_x, \min_{t'} Y(t')\geq a_y, \max_{t'} Y(t')\leq b_y|  X(0)=x_0, Y(0)=y_0, \theta \right) = \\
%   \Pr_{W}\left(\left\{ \omega \in \Omega | X_\omega(t) \in dx,\,\, Y_\omega(t) \in dy,\,\,  \forall t' \in [0,t]\,\, a_x \leq X_\omega(t') \leq b_x,\,\, \forall t' \in [0,t] \,\, a_b \leq Y_\omega(t') \leq b_y,\,\, X_\omega(0) = Y_\omega(0) = 0 \right\} \right)
% \end{align*}
% Our strategy will be to define a family of sets $\left\{ \tilde{A}_m(z) \right\}_{m=0}^\infty$ for any $z=(x,y,a_x,b_x,a_y,b_y)$ such that
% \begin{align*}
%   \lim_{m \to \infty} \Pr_W \left(  \tilde{A}_m(z) \right) = 
%   \Pr \left( X(t) \leq x,
%   Y(t) \leq y, \min_{t' \leq t} X(t') \leq a_x, \max_{t' \leq t}
%   X(t') \leq b_x, \min_{t' \leq t} Y(t') \leq a_y, \max_{t' \leq t}
%   Y(t') \leq b_y \right) = F(z | \theta)
% \end{align*}

% \begin{lemma}
%   For $z = (x,y,a_x,b_x,a_y,b_y)$, 
%   \[\lim_{m \to \infty} \displaystyle \int_{-\infty}^x \displaystyle
%     \int_{-\infty}^y q(x',y',t, a_x-m, b_x, a_y-m, b_y)\,\, dx'\,dy' -
%     \displaystyle \int_{-\infty}^x \displaystyle \int_{-\infty}^y
%     q(x',y',t, a_x, b_x, a_y, b_y)\,\, dx'\,dy' = F(z | \theta)\]
% \end{lemma}
% \begin{proof}
% Given some $z = (x,y,a_x,b_x,a_y,b_y)$, define the sets
% \begin{align*}
%   A_m(z) &= \left\{ \omega \in \Omega | X_\omega(t) \leq x,\,\,
% Y_\omega(t) \leq y,\,\, \right. \\ & \forall t' \in [0,t]\,\, a_x-m
% \leq X_\omega(t') \leq b_x,\,\, \\ & \forall t' \in [0,t] \,\, a_y -
% m\leq Y_\omega(t') \leq b_y,\,\, \\ & \left.X_\omega(0) = Y_\omega(0)
% = 0 \right\}, \quad \quad m = 0,1,2,... \\
% \end{align*}
% Note here that $\Pr_{W}(A_m(z)) = \displaystyle \int_{-\infty}^x \displaystyle \int_{-\infty}^y q(x',y',t, a_x-m, b_x, a_y-m, b_y)\,\, dx'\,dy'$.  Next, we define
% \begin{align*}
%   \tilde{A}_{m}(z) &= A_m(z) \,\, \bigcap \,\, A_0^C(z) \\
%                    &= \left\{ \omega \in \Omega | X_\omega(t) \in dx,\,\, \right.\\
%                    & \quad \quad \forall t' \in [0,t]\,\, a_x-m \leq X_\omega(t') \leq b_x,\,\, \\
%                    & \quad \quad \forall t' \in [0,t] \,\, a_y - m\leq Y_\omega(t') \leq b_y,\,\, \\
%                    & \quad \quad \exists t_{a_x} \in [0,t] \,\, s.t. \,\, X_\omega(t_{a_x}) < a_x \,\, , \exists t_{a_y} \in [0,t] \,\, s.t. \,\, X_\omega(t_{a_y}) < a_y, \\
%                    & \quad \quad \left.X_\omega(0) = Y_\omega(0) = 0 \right\}
% \end{align*}
% It is easy to see that $A_{m}(z) \subset A_{m+1}(z)$ and that
% $\tilde{A}_{m}(z) \subset \tilde{A}_{m+1}(z)$. Because of the former
% relation,
% \[
%   \Pr_{W}(\tilde{A}_m(z)) = \Pr_W(A_m(z)) - \Pr_{W}(A_0(z)) = \displaystyle \int_{-\infty}^x \displaystyle \int_{-\infty}^y q(x',y',t, a_x-m, b_x, a_y-m, b_y)\,\, dx'\,dy' - \displaystyle \int_{-\infty}^x \displaystyle \int_{-\infty}^y q(x',y',t, a_x, b_x, a_y, b_y)\,\, dx'\,dy'.
% \]
% Finally,
% \begin{align*}
%   \bigcup_{m=0}^\infty \tilde{A}_m(z) &= \left\{ \omega \in \Omega | X_\omega(t) \in dx,\,\, \right.\\
%                    & \quad \quad \forall t' \in [0,t]\,\, -\infty \leq X_\omega(t') \leq b_x,\,\, \\
%                    & \quad \quad \forall t' \in [0,t] \,\, -\infty \leq Y_\omega(t') \leq b_y,\,\, \\
%                    & \quad \quad \exists t_{a_x} \in [0,t] \,\, s.t. \,\, X_\omega(t_{a_x}) < a_x \,\, , \exists t_{a_y} \in [0,t] \,\, s.t. \,\, X_\omega(t_{a_y}) < a_y, \\
%                    & \quad \quad \left.X_\omega(0) = Y_\omega(0) = 0 \right\},
% \end{align*}
% such that for $\omega \in \bigcup_{m=0}^\infty \tilde{A}_m(z)$,
% $X_\omega(t) \leq x, Y_\omega(t) \leq y, \min_{t'} X_{\omega}(t') \leq
% a_x, \min_{t'} Y_{\omega}(t') \leq a_y, \max_{t'} X_{\omega}(t') \leq
% b_x$, $\max_{t'} Y_{\omega}(t') \leq b_y$, and
% \[
%   \Pr_W\left( \bigcup_{m=0}^\infty \tilde{A}_m(z) \right) = \lim_{M \to \infty} \Pr_W\left( \bigcup_{m=0}^M \tilde{A}_M(z) \right) = F(z |
%   \theta).
% \]
% Since $\tilde{A}_{m}(z) \subset \tilde{A}_{m+1}(z)$,
% \[
%   \lim_{m\to \infty} \Pr_W(\tilde{A}_m(z)) = \lim_{m \to \infty} \displaystyle \int_{-\infty}^x \displaystyle \int_{-\infty}^y q(x',y',t, a_x-m, b_x, a_y-m, b_y)\,\, dx'\,dy' - \displaystyle \int_{-\infty}^x \displaystyle \int_{-\infty}^y q(x',y',t, a_x, b_x, a_y, b_y)\,\, dx'\,dy'  = F(z | \theta)
% \]
% \end{proof}
% For a shorthand, denote
% $
%   \mu_k(\tilde{A}_m(z)) = \displaystyle \int_{-\infty}^x \displaystyle \int_{-\infty}^y q_k(x',y',t, a_x-m, b_x, a_y-m, b_y)\,\, dx'\,dy' - \displaystyle \int_{-\infty}^x \displaystyle \int_{-\infty}^y q_k(x',y',t, a_x, b_x, a_y, b_y)\,\, dx'\,dy'.
% $

% \begin{lemma} \label{lem:weak}
%   For a fixed $m$, $\mu_k(\tilde{A}_m(z)) \to \Pr_W(\tilde{A}_m(z))$ as $k\to \infty$.
% \end{lemma}
% \begin{proof}
%   The proof for this lemma follows direcly from the convergence result
%   $p^{(k)} \to p$ as $k\to \infty$ under the $L_2(\Omega)$ norm from
%   Section \ref{sec:semidiscrete-galerkin}.
% \end{proof}
% With the result from Lemma \ref{lem:weak}, we can now define the
% function $k(m; z)$ as
% \[
%   k(m; z) := \inf_{k'}\left\{ \left| \mu_{k'}(\tilde{A}_m(z)) - \Pr_W(\tilde{A}_m(z)) \right| < 1/2m  \right\}
% \]

% \begin{lemma}\label{lem:convergence-in-dist}
%   For $F_{k(m;z)}$, $F$, and $z$ defined above,
%   \[ \lim_{m\to \infty} F_{k(m;z)}(z | \theta) = F(z | \theta). \]
% \end{lemma}

% \begin{proof}
%   For any $\epsilon > 0$, let $M_1 = 2/\epsilon$. By Lemma
%   \ref{lem:1}, we can find $M_2$ such that $\forall m > M_2$
%   $\left| \Pr_W(\tilde{A}_m(z) - F(z | \theta) \right| <
%   \epsilon/2$. Let $M = \max\{M_1, M_2\}$. For any $m > M$,
%   \begin{align*}
%     \left| F_{k(m)}(z) - F(z) \right | &= \left| \mu_{k(m)}(\tilde{A}_m(z)) - F(z) \right | \\
%                                        &\leq \left|  \mu_{k(m)}(\tilde{A}_m(z)) - \Pr_W(\tilde{A}_m(z)) \right| + \left|  \Pr_W(\tilde{A}_m(z)) - F(z | \theta) \right|. \\
%                                        &\leq \epsilon/2 + \epsilon/2
%   \end{align*}
%   The left term bound is defined by $k(m;z)$.
% \end{proof}

% % \begin{lemma}
% %   For a fixed $k$,
% %   $\lim_{m\to \infty} \mu_k(\tilde{A}_m(z)) = c \in \mathbb{R},$ i.e.
% %   $\left\{ \mu_k(\tilde{A}_m(z))\right\}_{m=0}^\infty$ is a convergent
% %   sequence.
% % \end{lemma}

% % \begin{proof}[Proof idea]
% %   Here we will present a formal argument, without giving all of the
% %   necessary technical details. As $m \to \infty$, the parameters
% %   $\tau_x, \tau_y$ in the normalized problem are $O(1/m)$.  Further,
% %   the initial condition coordinates tend to
% %   $((1-1/m),(1-1/m))$. Letting $\gamma = 1/m$ for $m >> 1$, the
% %   normalized problem becomes
% %   \begin{equation*}
% %     \frac{\partial}{\partial t} p(x,y,t) = \mathcal{L}p(x,y,t),\quad (x,y) \in = \Omega
% % \end{equation*}
% % with the initial condition
% % \begin{align*}
% %   p(x,y,0) &= \delta\left( x - (1-\gamma) \right) \delta\left(y-(1-\gamma)\right),
% % \end{align*}
% % where the differential operator $\mathcal{L}$ is of order
% % \[
% %   \mathcal{L} = \frac{1}{2} \gamma^2 \frac{\partial^2}{\partial x^2}
% %   + \rho\gamma^2 \frac{\partial^2}{\partial x \partial y} + \frac{1}{2}\gamma^2 \frac{\partial^2}{\partial y^2}.
% % \]
% % This means that fundamental solution to the unbounded problem above is
% % the Gaussian density
% % \[
% %   G(x,y,t'| \gamma) = \frac{1}{2\pi\,\,t'\,\, \gamma^2\sqrt{1-\rho^2}} \exp\left\{ -\frac{1}{2\, t' \, (1-\rho^2)} \left( \frac{(x - 1 + \gamma)^2}{\gamma^2} - 2\rho \frac{(x-1+\gamma)(y-1+\gamma)}{\gamma^2} + \frac{(y - 1 + \gamma)^2}{\gamma^2}\right) \right\}, t' \in (0,t]
% % \]
% % We claim that according to our method, we can find a constant $\Gamma$
% % such that for $\gamma < \Gamma$, $t_\epsilon > t$ for the small-time
% % solution $p(x,y,t_\epsilon)$, which means that $G(x,y,t | \gamma)$
% % solves the above problem. This means that, \textbf{for fixed} $k$, the
% % Galerkin approximation generated by our method is the projection of
% % $p(x,y,t)$ onto the basis family $S_k$ (there is no forward evolution
% % of the problem).

% % Further, because
% % $G(x,y,t | \gamma) \to \delta(x-1+\gamma)\delta(y-1+\gamma)$ weakly as
% % $\gamma \to 0$, each of the projections of $G(x,y,t | \gamma)$ onto $S_k$ converge to the basis functions evaluated at $((1-\gamma), (1-\gamma))$:
% % \[
% %   \displaystyle \int_\Omega G(x,y,t | \gamma) \psi_i(x,y) dx\,dy \to \psi_i(1-\gamma,1-\gamma).
% % \]
% % Given $S_k$, the above coefficients uniquely define $p^{(k)}$
% % projection is defined only by $p(x,y,t_\epsilon)$ and $S_k$ only on
% % $\mathbf{p}(t_\epsilon)$ (see equation
% % (\ref{eq:orthogonality-conditions-mat-2})) become the same. Because
% % $k$ is fixed, the mass $M$ stays the same, which means that the
% % initial condition vectors in the Galerkin method also converge at
% % $\gamma \to 0$. Fu
% % \end{proof}

\begin{lemma}
  The maximum likelihood estimator is consistent as $n \to \infty$ and $m \to \infty$:
  \[ \hat{\theta}_{n,k} \to \theta \].
\end{lemma}
\begin{proof}
  By Lemma \ref{lem:conv-dist}
    \[ Z_k \xrightarrow[]{d} Z \mbox { as } k \to \infty. \] Next,
    given Theorem 4.1 in \cite{singler2008differentiability}, we know
    that, for each $k$, $q_k$ is analytic in both the diffusion
    parameters and boundary parameters. Hence, the probability density
    function satisfies the criteria A1 - A6 in
    \cite{casella2002statistical} to guarantee that, for data
    $Z_{k} \sim F_k(\theta)$,
    \[ \hat{\theta}_{n,k}(Z_k) \xrightarrow[]{p} \theta \mbox{ as } n
      \to \infty. \]

    Now we need to show that the same holds for data sampled from $F$
    as $k \to \infty$. To do this, we will use Chebyshev's inequality:
  \[
    \Pr_{Z}\left( \left| \hat{\theta}_{n,k}(Z) - \theta \right| \geq
      \epsilon \right) \leq \frac{ \mbox{E}_{Z}\left[
        (\hat{\theta}_{n,k}(Z) - \theta)^2 \right] }{ \epsilon^2 }.
  \]
  By the Maximum theorem [REFERENCE], $\hat{\theta}_{n,k}(x)$ is a continuous
  function with respect to $x$, and further because we have bounded
  $\hat{\theta}$ from below and above,
  \[
    \mbox{E}_{Z_k}\left[ (\hat{\theta}_{n,k}(Z_k) - \theta)^2 \right]
    \to \mbox{E}_{Z}\left[ (\hat{\theta}_{n,k}(Z) - \theta)^2 \right]
    \mbox{ as } k \to \infty
  \]
  by the portmanteau lemma. Finally, we can show that
  \begin{equation}
    \mbox{E}_{Z_k}\left[ (\hat{\theta}_{n,k}(Z_k) - \theta)^2 \right]
    \to 0 \mbox{ as } n \to \infty, \label{eq:var-lim}
  \end{equation}
  since the expected value of the estimator tends to $\theta$ and its
  variance goes to 0 when $n \to \infty$. Therefore, given any
  $\epsilon > 0$ and $\delta > 0$, we can find a sufficiently large
  $n$ and $k$ such that
  \[
    \Pr_{Z}\left( \left| \hat{\theta}_{n,k}(Z) - \theta \right| \geq
      \epsilon \right) \leq \frac{ \mbox{E}_{Z}\left[
        (\hat{\theta}_{n,k}(Z) - \theta)^2 \right] }{ \epsilon^2 } < \delta    
  \]
\end{proof}


\bibliographystyle{plainnat}
\bibliography{master-bibliography}
\end{document}
