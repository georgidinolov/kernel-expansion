\message{ !name(solutions-to-advection-diffusion-PDEs-on-bounded-regular-domains.tex)}\documentclass[10pt]{article}
\pdfminorversion=4

\usepackage{amssymb}
\usepackage{pslatex,palatino,avant,graphicx,color}
\usepackage{colortbl}
\usepackage{fullpage}
\usepackage{mathtools}
\usepackage{natbib}
\usepackage{amsmath}
\usepackage{caption}
%\usepackage{subfigure}
\usepackage{subcaption}
\usepackage{bm}
\usepackage{wrapfig}
\usepackage{enumerate}
\usepackage{rotating}
\usepackage{multirow}
\usepackage{tabularx}
\usepackage{tikz}
\usepackage{pgfplots}
\usepackage{bbm}
\usepackage[titles,subfigure]{tocloft} % Alter the style of the Table of Contents

%\pdfminorversion=4
% NOTE: To produce blinded version, replace "0" with "1" below.
\newcommand{\blind}{0}

% DON'T change margins - should be 1 inch all around.
\addtolength{\oddsidemargin}{-.5in}%
\addtolength{\evensidemargin}{-.5in}%
\addtolength{\textwidth}{1in}%
\addtolength{\textheight}{1.3in}%
\addtolength{\topmargin}{-.8in}%


\renewcommand{\cftsecfont}{\rmfamily\mdseries\upshape}
\renewcommand{\cftsecpagefont}{\rmfamily\mdseries\upshape} % No bold!

\newcommand{\indicator}[1]{\mathbbm{1}\left( #1 \right) }
\newcommand{\hb}{\hat{b}}
\newcommand{\ha}{\hat{a}}
\newcommand{\htheta}{\hat{\theta}}
\newcommand{\halpha}{\hat{\alpha}}
\newcommand{\hmu}{\hat{\mu}}
\newcommand{\hsigma}{\hat{\sigma}}
\newcommand{\hphi}{\hat{\phi}}
\newcommand{\htau}{\hat{\tau}}
\newcommand{\heta}{\hat{\eta}}
\newcommand{\E}[1]{\mbox{E}\left[#1\right]}
\newcommand{\Var}[1]{\mbox{Var}\left[#1\right]}
\newcommand{\Indicator}[1]{\mathbbm{1}_{ \left( #1 \right) } }
\newcommand{\dNormal}[3]{ N\left( #1 \left| #2, #3 \right. \right) }
\newcommand{\Beta}[2]{\mbox{Beta}\left( #1, #2 \right)}
\newcommand{\alphaphi}{\alpha_{\hphi}}
\newcommand{\betaphi}{\beta_{\hphi}}
\newcommand{\expo}[1]{ \exp\left\{ #1 \right\}}
\newcommand{\tauSquareDelta}{\htau^2
  \left(\frac{1-\expo{-2\htheta\Delta}}{2\htheta} \right)}
\newcommand{\mumu}{\mu_{\hmu}}
\newcommand{\sigmamu}{\sigma^2_{\hmu}}
\newcommand{\sigmamuexpr}{\log\left( \frac{\VarX}{\EX^2} + 1 \right)}
\newcommand{\mumuexpr}{\log(\EX) -  \log\left( \frac{\VarX}{\EX^2} + 1 \right) /2 }

\newcommand{\EX}{\mbox{E}\left[ X \right] }
\newcommand{\VarX}{\mbox{Var}\left[ X \right] }
\newcommand{\mueta}{\mu_{\heta} }
\newcommand{\sigmaeta}{\sigma^2_{\heta}}
\newcommand{\sigmaetaexpr}{ \log\left( \frac{\VarX}{\EX^2} + 1 \right) }
\newcommand{\muetaexpr}{ \log(\EX) -  \sigmaetaexpr /2 }

\newcommand{\mualpha}{\mu_{\halpha} }
\newcommand{\sigmaalpha}{\sigma^2_{\halpha}}
\newcommand{\sigmaalphaexpr}{ \log\left( \frac{\VarX}{\EX^2} + 1 \right) }
\newcommand{\mualphaexpr}{ \log(\EX) -  \sigmaalphaexpr /2 }

\newcommand{\mutauexpr}{ \frac{2}{T} \EX }
\newcommand{\sigmatauexpr}{ \frac{4}{T^2} \Var{X}}

\newcommand{\alphatau}{\alpha_{\htau^2}}
\newcommand{\betatau}{\beta_{\htau^2}}

\newcommand{\Gam}[2]{\mbox{Gamma}\left( #1, #2 \right) }
\newcommand{\InvGam}[2]{\mbox{Inv-Gamma}\left( #1, #2 \right) }

%%% END Article customizations

%%% The "real" document content comes below...

% \newbox{\LegendeA}
% \savebox{\LegendeA}{
%    (\begin{pspicture}(0,0)(0.6,0)
%    \psline[linewidth=0.04,linecolor=red](0,0.1)(0.6,0.1)
%    \end{pspicture})}
% \newbox{\LegendeB}
%    \savebox{\LegendeB}{
%    (\begin{pspicture}(0,0)(0.6,0)
%    \psline[linestyle=dashed,dash=1pt 2pt,linewidth=0.04,linecolor=blue](0,0.1)(0.6,0.1)
%    \end{pspicture})}

\title{Solution to a Non-Seperable Diffusion Equation on a Regular Domain}
\author{Georgi Dinolov, Abel Rodriguez, Hongyun Wang}
\date{} % Activate to display a given date or no date (if empty),
         % otherwise the current date is printed 

\begin{document}

\message{ !name(solutions-to-advection-diffusion-PDEs-on-bounded-regular-domains.tex) !offset(-3) }

\def\spacingset#1{\renewcommand{\baselinestretch}%
{#1}\small\normalsize} \spacingset{1}

\bigskip

\abstract{}

\vspace{1cm}
\noindent
{\it Keywords:} Diffusion equation, regular bounded domain

\spacingset{1.00} % DON'T change the spacing! was set to 1.45
\section{Motivation}\label{se:motivation}

\section{Solution on $\Omega \subset \mathbb{R}$}

In this Section we will demonstrate the method outlined in Section
\ref{se:motivation} where the solution is defined on a bounded
interval on $\mathbb{R}$. In this case, we have the true solution to
the diffusion equation. We will compare the asymptotic
expansion to the true solutio.

The PDE we will solve is the following BC/IC problem
\begin{align}
  \frac{\partial}{\partial t} q(x,t) &= \frac{1}{2}\sigma^2 \frac{\partial^2}{\partial x^2} q(x,t), \label{eq:pde} \\
  q(x,0) &= \delta_{x_0}(x), \label{eq:ic} \\
  q(a,t) &= q(b,t) = 0. & (\mbox{i.e. } \Omega = [a,b]) \label{eq:bc}
\end{align}
\textbf{Without loss of generality we will assume}
\begin{align*}
  a &=0, & b &= 1.
\end{align*}

Problem (\ref{eq:pde}) - (\ref{eq:bc}) can be solved in a variety of
ways. We will use the method of images, which repeatedly reflects the
fundamental solution
\[
  q_{fundamental}(x,t) = \frac{1}{\sqrt{2\pi \sigma^2 t}} \exp\left\{ -\frac{1}{2\sigma^2t} (x-x_0)^2 \right\}
\]
about the boundary points $a$ and $b$. The steps for the full
solutions are as follows:

\begin{enumerate}[Step 1:]
\item Select a kernel $f(x|t)$ for the basis expansion,
\item Perform Gram-Schmidt orthogonalization on the polynomials basis,
\item Compute the weight for each basis element,
\item Profit.
\end{enumerate}

\subsection{A suitable kernel for the basis elements}
As noted in the motivating Section \ref{se:motivation}, the kernel we
will use must be in $C^{\infty}(a,b)$, and it must obey the boundary
conditions. Moreover, the basis kernel must be chosen such that
\begin{enumerate}[i)]
\item derivatives $f'(x)$ can be computed easily, 
\item integrals $\int_{\Omega} x^m f(x|t)^2\, dx$ can be computed easily
\end{enumerate}
Consideration i) suggests that $f(x|t)$ should be of polynomial
form. Consideration ii) suggests that $f(x|t)$ should be a known pdf
over $[a,b]$, taking on zero at $a$ and $b$.  Given these
requirements, the Beta distribution comes to mind:
\[
  f(x|t, \alpha, \beta) = \frac{1}{B(\alpha,\beta)} x^{\alpha-1}(1-x)^{\beta-1},
\]
where $B(\alpha,\beta)$ is the beta function. Our choice for $\alpha$
and $\beta$ is not very restricted. However, we will outline a few
heuristics by which we can choose these parameters. Note that there
may exist and optimal choise for $(\alpha, \beta)$ in terms of the
accuracy of the asymptotic expansion with respect to the true solution
$q(x,t)$. However, we will not prove anything in this vein here.

\textbf{First}, as long as
\begin{align}
  \alpha, \beta > 1, \label{eq:lower-bound}
\end{align}
the mode for the distribution is guaranteed to
exist, so that the boundary conditions are met.

Aside from $\alpha > 1$ and $\beta > 1$, we can pick any
$(\alpha,\beta)$ pair for our kernel. However, given that $f(x|t)$ can
be thought of as implicitly dependent upon $t$, and that the variance
of the fundamental solution is $\sigma^2t$, a first, reasonable guess
for $(\alpha,\beta)$ can be given by the solution to the equation:
\begin{align}
  \Var{X} &:= \frac{\alpha \beta}{(\alpha+\beta)^2(\alpha+\beta+1)} = \sigma^2t, \label{eq:var-restriction}\\
  p(X \in dx) &= f(x|t, \alpha, \beta). \nonumber
\end{align}

By the same logic, noting the mean for the fundamental solution, we can require
\begin{align}
  \E{X} &:= \frac{\alpha}{\alpha+\beta} = x_0, \label{eq:mean-restriction}\\
  p(X \in dx) &= f(x|t, \alpha, \beta). \nonumber
\end{align}

Finally, we may require that $\alpha, \beta \in \mathbb{Z}$, since
this will guarantee that
\[
  \frac{\partial^k}{\partial x^k} f(x|t,\alpha,\beta) = 0
\]
for large enough $k$. [georgid: This may not prove important, but I
will keep it here anyway]

Thus, to set $\alpha$ and $\beta$, we simultaneously solve
(\ref{eq:var-restriction}) and (\ref{eq:mean-restriction}), then round
$\alpha$ and $\beta$ to the closest integer greater than or equal to
2. Since $\alpha$ and $\beta$ are dependent upon $t$, we will keep $t$
in our notation for $f$, albeit implicitly. In other words, once we
choose $\alpha$ and $\beta$, we will not be able to take derivatives
of $f$ with respect to $t$. We will denote the kernel as
$f(x| \alpha, \beta ; t)$.

\subsection{Gram-Schmidt orthogonalization on the polynomials basis}
The family of (polynomial) functions
$\left\{ x^m f(x| \alpha, \beta; t) \right\}_{m=0}^\infty$ spans the
space of $L^2([a,b])$ functions. We generate the basis elements
$\left\{ u_m(x|\alpha,\beta;t) \right\}_{m=0}^\infty$ by setting
\begin{align}
  v_0(x| \alpha, \beta; t) &= f(x| \alpha, \beta; t), \nonumber \\
  u_0(x| \alpha, \beta; t) &= \frac{f(x| \alpha, \beta; t)}{\|f(x|\alpha, \beta; t\|}, \label{eq:u0} \\
  \|f(x|\alpha,\beta;t) \| &\equiv \left( \int_{\Omega} f(x|\alpha,\beta;t)^2 dx \right)^{1/2}. \label{eq:norm}
\end{align}
The integral in (\ref{eq:norm}) is easy to compute because of the form
we have chosen for the kernel $f:$
\begin{align*}
  \int_{\Omega} f(x|\alpha,\beta;t)^2 dx  &= \int_{\Omega} \left( \frac{1}{B(\alpha,\beta)} x^{\alpha-1}(1-x)^{\beta-1} \right)^{2} dx, \\
                                          &= \int_{\Omega} \frac{1}{B(\alpha,\beta)^2} x^{(2\alpha-1)-1}(1-x)^{(2\beta-1)-1} dx, \\
                                          &= \frac{B(2\alpha-1, 2\beta-1)}{B(\alpha,\beta)^2}, \\
  \left( \int_{\Omega} f(x|\alpha,\beta;t)^2 dx \right)^{1/2} &= \sqrt{\frac{B(2\alpha-1, 2\beta-1)}{B(\alpha,\beta)^2}}.
\end{align*}
Next, for $u_1(x;\alpha,\beta;t)$,
\begin{align}
  v_1(x| \alpha, \beta; t) &= x f(x|\alpha,\beta;t) - \left<x f(x|\alpha,\beta;t) | u_0(x|\alpha,\beta;t) \right> u_0(x|\alpha,\beta;t) \\
  &= \left(x - \frac{\left<x f(x|\alpha,\beta;t) | u_0(x|\alpha,\beta;t) \right>}{\| f(x|\alpha,\beta;t) \|} \right) f(x|\alpha,\beta;t) \\
  \left<x f(x|\alpha,\beta;t) | u_0(x|\alpha,\beta;t) \right> &= \int_\Omega \frac{x f(x|\alpha,\beta;t)^2}{\| f(x|\alpha,\beta;t) \|} = \frac{1}{\| f(x|\alpha,\beta;t) \|} \int_\Omega \frac{1}{B(\alpha,\beta)^2} x^{2\alpha-1}(1-x)^{(2\beta-1)-1} dx  \\
  u_1(x|\alpha,\beta;t) &= \frac{v_1(x|\alpha,\beta;t)}{\|v_1(x|\alpha,\beta;t)\|} \\
  \|v_1(x|\alpha,\beta;t)\| &= \int_\Omega \left(x - \frac{\left<x f(x|\alpha,\beta;t) | u_0(x|\alpha,\beta;t) \right>}{\| f(x|\alpha,\beta;t) \|} \right)^2 f(x|\alpha,\beta;t)^2 dx \\
  u_1(x|\alpha,\beta;t) &= \frac{v_1(x|\alpha,\beta;t)}{\|v_1(x|\alpha,\beta;t)\|} = p_1(x) f(x|\alpha,\beta;t), 
\end{align}
where $p_1(x)$ is a first-order polynomial. In general,
\begin{align*}
  v_{m}(x|\alpha,\beta;t) &= x^m f(x|\alpha,\beta;t) - \sum_{m'=0}^{m-1} \left< x^m f(x|\alpha,\beta;t) | u_{m'}(x|\alpha,\beta;t) \right>u_{m'}(x|\alpha,\beta;t) \\
  u_{m}(x|\alpha,\beta;t) &=  \frac{v_{m}(x|\alpha,\beta;t)}{\| v_{m}(x|\alpha,\beta;t) \|}  \equiv p_m(x|\alpha,\beta;t) f(x|\alpha,\beta;t)
\end{align*}
In finding the basis, we will have to perform two main types
calcluations:
\begin{enumerate}[1)]
\item polynomial multiplication: $p_m(x|\alpha,\beta;t)p_n(x|\alpha,\beta;t)$
\item integration of the form: $\int_\Omega x^m f(x|\alpha,\beta;t)^2 dx$
\end{enumerate}
In \texttt{R}, the package \texttt{mpoly} will be used to handle
1). Calculation 2) can be performed relatively easily due to the form
of $f(x|\alpha,\beta;t)$, as show in (\ref{eq:intpoly}).

\begin{align}
  \int_\Omega x^m f(x|\alpha,\beta;t)^2 dx &= \int_\Omega x^m \frac{1}{B(\alpha,\beta)^2} x^{2\alpha-2}(1-x)^{2\beta-2} dx = \frac{1}{B(\alpha,\beta)^2} \int_\Omega x^{2\alpha+m-2}(1-x)^{2\beta-2} dx = \frac{B(2\alpha+m-1, 2\beta-1)}{B(\alpha,\beta)^2} \label{eq:intpoly}
\end{align}

\subsection{Computing the Weights of the Basis Elements}
Given the set of orthonormal functions
$\{u_m(x|\alpha,\beta;t)\}_{m=0}^\infty$ spanning $L^2([a,b])$, and
assuming that $q(x,t) \in L^2([a,b])$, we can write down
\begin{align}
   q(x,t) &= \sum_{m=0}^\infty c_m u_m(x|\alpha,\beta;t), \\
  \mbox{with  } c_m &= \int_\Omega q(x,t) u_m(x|\alpha,\beta;t) dx. \label{eq:coef}
\end{align}
Since each $u_m$ is the product of two polynomials,
$u_m(x|\alpha,\beta;t) \in C^{\infty}([a,b])$ and is
square-integrable. Therefore, we can write
\begin{equation}
  \int_\Omega \frac{\partial^k \delta_{x_0}(x)}{\partial x^2}
  u_m(x|\alpha,\beta;t) dx = (-1)^{k} \int_\Omega \delta_{x_0}(x)
  \frac{\partial^k u_m(x|\alpha,\beta;t)}{\partial x^k} dx = (-1)^k
  \left. \frac{\partial^k u_m(x|\alpha,\beta;t)}{\partial x^k}
  \right|_{x=x_0} \label{eq:delta-deriv}
\end{equation}
Equipped with (\ref{eq:delta-deriv}) and that
$q(x,0) = \delta_{x_0}(x)$, we can compute the integrals in
(\ref{eq:coef}) by using the Taylor expansion of $q(x,t)$ about $t=0$:
\begin{align*}
  \int_\Omega q(x,t) u_m(x|\alpha,\beta;t) dx &= \int_\Omega \left[\underbrace{q(x,0)}_{\delta_{x_0}(x)} + \sum_{k=1}^\infty \frac{t^k}{k!} \frac{\partial^k q(x,t)}{\partial t^k} \right] u_m(x|\alpha,\beta;t) dx
\end{align*}

\bibliographystyle{plainnat}
% \bibliographystyle{bka}
\bibliography{master-bibliography}
\end{document}

\message{ !name(solutions-to-advection-diffusion-PDEs-on-bounded-regular-domains.tex) !offset(-297) }
